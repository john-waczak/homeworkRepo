\documentclass[a4paper, 11pt]{article}
\usepackage{geometry}
\geometry{letterpaper, margin=1in}
\usepackage{amsmath} 
\usepackage{ulem} 


\begin{document}
%Header-Make sure you update this information!!!!
\noindent
\large\textbf{Homework 1} \hfill \textbf{John Waczak} \\
\normalsize PH 431 \hfill  Date: \today \\
Prof. Bo Sun  \\


\section*{1. Image Charge}
\textit{A charge q is placed near two perpendicular grounded, infinitely large metal plates,
	what is the force exerted on q? Can you guess how the far field potential behaves?} \\ 

I shall denote $\Phi_{tot}$ to be the total electric potential caused by $q$ and the induced charge in the perpendicular grounded plates. There are two important considerations for this problem that help to define our boundary conditions. They are that: 
	\begin{align}
		&\Phi_{tot}(x=0) = \Phi_{tot}(y=0) = 0 \\ 
		&\Phi_{tot}(x^2+y^2>>d_1^2+d_2^2) \rightarrow 0
	\end{align}
In words: the potential on the plates must be zero because they are grounded and the potential at infinity must be zero in the first quadrant. These two equations will constitute the boundary conditions necessary to solve the problem. The total potential, $\Phi_{tot}$ does not obey Laplace's equation in the first quadrant which we need in order to apply the uniqueness theorem. However, the induced potential in the plates does as all of the charge is contained on the surfaces and not in the first quadrant. So we can say: 
	\begin{align}
		\nabla^2 \Phi_{ind} = 0
	\end{align}
Furthermore because of equation (1) and the fact that the only charged objects are $q$ and the induced charge on the plates, we can deduce the following: 
	\begin{align}
		\Phi_{total} &= \Phi_{ind}+\Phi{q} = 0\\ 
		\Phi_{ind} &= -\Phi_{q}
	\end{align}
In order to solve this problem we simply need to find a potential that satisfies the boundary conditions outlined above. The uniqueness theorem then guarantees that this potential is the unique potential for the system in the region of the first quadrant. So rather than reasoning about the grounded plates, we will analyze an analogous system of image charges. \\

\textbf{Claim:} The potential in the first quadrant can be determined by the charge q located at $(d_1,d_2,z)$ along with the image charges -q  at $(-d_1, d_2, z)$, q at $(-d_1, -d_2, z)$, and -q at $(d_1, -d_2, z)$. 

From here on I will suppress the z dependence due to the plates being infinite. The potential of this distribution, i.e. a quadrupole, is given by: 
	\begin{align*}
		\Phi_{tot} &= \Phi_{q} + \Phi_{Im} \\ 
				   &= \frac{1}{4\pi\epsilon_0} \Bigg\lbrack \frac{q}{\sqrt{(x-d_1)^2+(y-d_2)^2}}+\frac{-q}{\sqrt{(x+d_1)^2+(y-d_2)^2}} \\
				   &+\frac{q}{\sqrt{(x+d_1)^2+(y+d_2)^2}}+\frac{-q}{\sqrt{(x-d_1)^2+(y+d_2)^2}} \Bigg\rbrack
	\end{align*}
First, $\Phi_{im}$ satisfies Laplace's equation in the first quadrant as all of the image charges reside outside of that region. Thus (3) holds. Secondly, the boundary condition that $\Phi_{tot} \rightarrow 0$ as $r \rightarrow \infty$ also holds as our configuration is a collection of point charges that all have $1/r$ dependence. So our only remaining boundary conditions are that $\Phi_{tot}(x=0) = 0$ and $\Phi_{tot}(y=0)=0$ on the "grounded plates." This we will show in two steps: 
	\begin{align*}
		\Phi_{tot}(x=0) &= \frac{1}{4\pi\epsilon_0} \Bigg\lbrack \frac{q}{\sqrt{d_1^2+(y-d_2)^2}}+\frac{-q}{\sqrt{d_1^2+(y-d_2)^2}} \\
		&+\frac{q}{\sqrt{d_1^2+(y+d_2)^2}}+\frac{-q}{\sqrt{d_1^2+(y+d_2)^2}} \Bigg\rbrack \\ 
		&= 0 
	\end{align*} 
as the terms inside cancel. Similarly for the second case of $y=0$ we have: 
	\begin{align*}
		\Phi_{tot}(y=0)&= \frac{1}{4\pi\epsilon_0} \Bigg\lbrack \frac{q}{\sqrt{(x-d_1)^2+d_2^2}}+\frac{-q}{\sqrt{(x+d_1)^2+d_2^2}} \\
		&+\frac{q}{\sqrt{(x+d_1)^2+d_2^2}}+\frac{-q}{\sqrt{(x-d_1)^2+d_2^2}} \Bigg\rbrack \\
		&=0
	\end{align*}
And so because this potential meets the boundary conditions, the uniqueness theorem guarantees it is the potential for the system. Now, because we have reduced the problem to a collection of point charges, solving for the force is incredibly easy. First we will the difference vectors from each of the image charges to q: 
	\begin{align*}
		\mathbf{r_1} &= 2 \cdot d_1 \hat{x} \\ 
		\mathbf{r_2} &= 2 \cdot d_1 \hat{x} + 2 \cdot d_2 \hat{y} \\  
		\mathbf{r_3} &= 2 \cdot d_2 \hat{y}
	\end{align*}

Using this information and the superposition principle, we conclude that: 
	\begin{align*}
		\mathbf{F}_{pt} &= \frac{1}{4\pi\epsilon_0}\frac{q_1q_2}{\mathbf{r}^2}\hat{\mathbf{r}} \\
		\mathbf{F_{im,q}} &= \mathbf{F_1} + \mathbf{F_2} + \mathbf{F_3} \\ 
		&= \frac{q}{4\pi\epsilon_0} \Bigg\lbrack \frac{-q}{4d_1^2}\hat{x} + \frac{q}{(4d_1^2 + 4d_2^2)^{\frac{3}{2}}}\left(2d_1 \hat{x} + 2d_2 \hat{y} \right) + \frac{-q}{4d_2^2}\hat{y} \Bigg\rbrack
	\end{align*}

As the image charge's illustrate, the expected far field behavior for this situation is that of a quadrupole. \\
\section*{2. Green's reciprocity theorem}
\textit{a. Prove the Green's reciprocity theorem} \\

In order to prove this theorem, I am going to make an assumption that both charge distributions $\rho_1$ and $\rho_2$ are finite as this will allow me to apply suitable boundary conditions. Some useful facts will be the following: 

\begin{align}
	&\nabla^2 \phi = \frac{\rho}{\epsilon_0} \\ 
	&\mathbf{E} = -\nabla \phi \\ 
	&\nabla \cdot \mathbf{E} = \frac{\rho}{\epsilon_0} \\ 
	&\nabla \cdot (f\mathbf{A}) = (\nabla \cdot \mathbf{A})f + \mathbf{A} \cdot \nabla f \\
	&\int_V (\nabla \cdot \mathbf{A})d^3\mathbf{r} = \oint_S \mathbf{A} \cdot d\mathbf{s} 
\end{align}

Now because we want to compare the two different distributions, we will look at a combination involving the electric fields $\mathbf{E_1}$ and $\mathbf{E_2}$ and then go to town applying the above theorems. Here on out when I write $\int$ I mean to integrate over all space unless otherwise specified.

\begin{align*}
	&\int \mathbf{E_1} \cdot \mathbf{E_2}  d^3\mathbf{r} = \\ 
	&= - \int (\nabla \phi_1) \cdot \mathbf{E_2} d^3\mathbf{r} \\ 
	&= - \int [\nabla \cdot (\mathbf{E_2}\phi_1)-(\nabla \cdot \mathbf{E_2})\phi_1] d^3\mathbf{r} \quad \mbox{eqn 10} \\ 
	&= - \int (\nabla \cdot \mathbf{E_2})\phi_1 - \nabla \cdot (\mathbf{E_2}\phi_1) d^3\mathbf{r} \quad \mbox{rearranging}\\ 
	&= -\int \frac{\rho_2}{\epsilon_0}\phi_1 - \nabla \cdot (\mathbf{E_2}\phi_1) d^3\mathbf{r}  \quad \quad \quad \mbox{ eqn 8}\\ 
	&= - \int \frac{\rho_2}{\epsilon_0}\phi_1 d^3\mathbf{r} - \int \nabla \cdot (\mathbf{E_2}\phi_1) d^3 \mathbf{r} \\
	&= - \int \frac{\rho_2}{\epsilon_0}\phi_1 d^3\mathbf{r} - \int \nabla \cdot (-\nabla \phi_1 \phi_2) d^3\mathbf{r} \quad \mbox{eqn 7} \\ 
	&= -\int \frac{\rho_2}{\epsilon_0}\phi_1d^3\mathbf{r} + \oint_S \phi_1 \nabla\phi_2 d^3\mathbf{r} \quad \quad \quad \mbox{eqn 10}
\end{align*}
In the last line we employed Gauss's law in reverse to give us a surface argument. Now here I employ my assumption that our potentials are both finite. This means that at infinity, these potentials must decay to 0. Now because we were integrating over all space when we convert the volume integral to a closed surface integral, the corresponding surface is of infinite radius. Thus the second integral must shrink to zero, i.e. 

\begin{align*}
	&= -\int \frac{\rho_2}{\epsilon}\phi_1 d^3\mathbf{r} + \mbox{\sout{$\oint_S \phi_1 \nabla\phi_2 d^3\mathbf{r}$}} \\ 
	&= -\int \frac{\rho_2}{\epsilon}\phi_1 d^3\mathbf{r} 
\end{align*} 

Now our choice to begin by expanding $\mathbf{E_1}$ was arbitrary and we could have repeated the same exact process but instead starting with $\mathbf{E_2}$. Thus we have that: 

\begin{equation}
	\frac{-1}{\epsilon_0} \int \rho_2 \phi_1 d^3\mathbf{r} = \int \mathbf{E_1} \cdot \mathbf{E_2} d^3\mathbf{r} = \frac{-1}{\epsilon_0} \int \rho_1 \phi_2 d^3\mathbf{r} 
\end{equation}

and thus: 

\begin{equation}
	\int\limits_{\mbox{all space}} \rho_2 \phi_1 d^3\mathbf{r} = \int\limits_{\mbox{all space}} \rho_1 \phi_2 d^3\mathbf{r} 
\end{equation}\\

\textit{Now let us assume there are two separate, isolated conductors A and B. There is no constraint on their shapes and relative distance. If I put a total charge of Q on A, it generates electric field which has constant potential $V_{AB}$ on B and $V_{AA}$ on A (because B is a conductor). If I put the same total charge Q on B, while leaving A charged with Q, then A has potential $V_{BA}$. Prove that $V_{BA} = V_{AB} + V_{AA}$}\\

In order to apply the above theorem, we need to identify two distinct charge configurations and the corresponding potentials $\rho_1, \rho_2, \phi_1, \phi_2$. Fortunately, the problem gives us those configurations as the scenario with one Q on A and then the second scenario with a charge Q on A and B. Keeping in mind that we only know the potentials on the respective surfaces, I'll redefine the volume charge densities as surface densities (because we know charge resides on the surface of a conductor) in the following way: 

\begin{align}
	\rho_1 &= \sigma_{AA}\delta(A) + \sigma_{AB}\delta(B) \\ 
	\rho_2 &= \sigma_{BA}\delta(A) + \sigma_{BB}\delta(B) \\
	\phi_1 &= \begin{cases}
	V_{AA} & \text{if on surface A} \\
	V_{AB} & \text{if on surface B} \\
	\end{cases} \\ 
	\phi_2 &= \begin{cases}
	V_{BA} &= \text{if on surface A} \\
	V_{BB} &= \text{if on surface B} \\ 
	\end{cases}
\end{align}
Where $\delta$ denotes the delta function with respect the a particular surface. Now we can apply Green's reciprocity theorem to finish the proof: 
\begin{align*}
	\int\limits_{\text{all space}} \rho_1 \phi_2 d^3r &= 	\int\limits_{\text{all space}} \rho_2 \phi_1 d^3{r}\\
	\int\limits_{\text{all space}} \big(\sigma_{AA}\delta(A) + \sigma{AB}\delta(B) \big)\phi_2 d^3r &= \int\limits_{\text{all space}} \big(\sigma_{BA}\delta(A) + \sigma_{BB}\delta(B) \big)\phi_1 d^3r \\
	\int\limits_A \sigma_{AA}\phi_2 dS + \int\limits_B \sigma_{AB}\phi_2 dS &= \int\limits_A \sigma_{BA}\phi_1 dS + \int\limits_B \sigma_{BB} \phi_1 dS \\ 
\end{align*}
So the $\delta$-functions collapsed the volume integrals over all space to surface integrals over each surface where we know the potential is constant. Thus we can pull it through the sum. 

\begin{equation*}
	V_{BA}\int\limits_A \sigma_{AA}dS + V_{BB}\int\limits_B \sigma_{AB}dS = V_{AA}\int\limits_A \sigma_{BA}dS + V_{AB}\int\limits_B \sigma{BB}dS
\end{equation*}
Now that we have separated out the potentials from the integrals they simply are the sum of the surface charge densities along each surface which evaluates to the charge on that surface. Therefore we have that: 

\begin{equation*}
	V_{BA}Q + V_{BB}0 = V_{AA}Q + V_{AB}Q 
\end{equation*}
And so dividing away the Q's gives the desired result. 
\begin{equation}
	V_{BA} = V_{AA} + V_{AB} 
\end{equation}
\section*{3. Conducting half sphere in a uniform field}
\textit{A metal half-sphere conductor of radius a is placed in uniform electric field as shown. If we define the potential of the conductor to be zero, what is the potential at any given point outside the sphere? At far away from the sphere can you express the electric field by a dipole approximation?}

I will choose the bottom (flat) half of the sphere to be centered at the origin of coordinates (i.e. $z = 0$). Note that because we have a uniform field in the z direction, $\mathbf{E} = E_0 \hat{z}$, then $\phi_E = -E_0z$ as $\mathbf{E} = -\nabla \phi_E$

We are given the potential at the surface ($\phi_{tot}$) is 0 and so our boundary conditions are: 

\begin{align}
	\phi_{in}(r \rightarrow \infty) = 0 \\ 
	\phi_{tot}(r=a, \theta < \frac{\pi}{2}) = 0 \\ 
	\phi_{tot}(r\leq a , \theta = \frac{\pi}{2}) = 0 
\end{align}

Notice that Laplace's equation for $\phi_{in}$ is satisfied with azimuthal symmetry for outside of the half sphere. Thus, we can immediately conclude that the solution must be of the form: 
\begin{equation*}
	\phi_{in} = \sum\limits_{l = 0}^{\infty} \bigg( A_lr^l + \frac{B_l}{r^{l+1}} \bigg)P_l (\cos{\theta})
\end{equation*}

We only need to find $A_l, B_l$ such that the boundary conditions are met and then the uniqueness theorem means that this is \textit{the} solution. The first boundary condition, $\phi_{in}(r\rightarrow \infty) = 0$ means that $A_l = 0$  $\forall$ $l$ because the $r^l$ terms blow up at $\infty$. Thus we can reduce the equation to: 

\begin{equation*}
	\phi_{in} = \sum\limits_{l = 0}^{\infty} \bigg( \frac{B_l}{r^{l+1}} \bigg)P_l(\cos{\theta})
\end{equation*}

The problem set $\phi_{tot} = 0$ along the boundary of the half sphere but were are not given that it is grounded. Thus, we can expect that the electric field will lead to the separation of charge along the surface forming a dipole for $\phi_{in}$ with $\frac{1}{r^2}$ dependence. Therefore I claim that all $B_l = 0$ unless $l = 1$ so that: 
\begin{equation*}
	\phi_{in} = \bigg( \frac{B_1}{r^2} \bigg)\cos(\theta)
\end{equation*}

Now we can solve for $B_1$ using the other boundary conditions (19) and (20).
\begin{align*}
	1) \quad &\phi_{tot}(r=a, \theta < \pi/2) = 0\\ 
	&\Rightarrow \phi_{in} = -\phi_{\mathbf{E}}(r=a, \theta < \pi/2)\\
	&\frac{B_1}{a^2}\cos{\theta} = E_0z \\ 
	&\frac{B_1}{a^2}\cos{\theta} = E_0a\cos{\theta}\\ 
	&B_1 = E_0a^3 \\ 
	2) \quad &\phi_{tot}(r < a, \theta = \pi/2) \Rightarrow z = 0\\ 
	& \cos{\pi/2} = 0 \Rightarrow \phi_{in} = \phi_{\mathbf{E}} = 0
\end{align*}
Thus the have found an equation that meets the boundary conditions and satisfies Laplace's equation in the desired region so by the uniqueness theorem equation (16) is the induced potential outside the half-sphere. 
\begin{equation}
	\phi_{in}(r,\theta) = \frac{E_0a^3}{r^2}\cos{\theta}
\end{equation}
The general equation for an electric dipole is: $\phi = \frac{1}{4\pi\epsilon_0}\frac{qd\cos{\theta}}{r^2}$. If we recognize $qd$ as the $E_0a^3$ term then yes, we can clearly see that this solution is for a dipole. This makes sense as we are not given that the conductor is grounded so we expect the conductor to form a dipole due to the field inducing a separation of charge. It does seem strange that this is the same solution as for the full sphere but I think this is just an artifact of artificially forcing the total potential to be zero along the surface. 
\end{document}















