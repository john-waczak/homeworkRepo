\documentclass[a4paper, 11pt]{article}
\usepackage{geometry}
\geometry{letterpaper, margin=1in}
\usepackage{amsmath}
\usepackage{amssymb}  
\usepackage{amsthm}
\usepackage{ulem} 
\usepackage{graphicx}
\usepackage{enumitem} 
\graphicspath{ {images/} }


\newtheorem*{theorem}{Theorem}
\newtheorem*{corollary}{Corollary}
\newtheorem*{lemma}{Lemma}
\newtheorem*{definition}{Definition}
\newtheorem*{proposition}{Proposition}


% stop typing \mathbb a thousand times 
\newcommand{\R}{\mathbb{R}}
\newcommand{\C}{\mathbb{C}}


\begin{document}
%Header-Make sure you update this information!!!!
\noindent
\large\textbf{Complex Variables: Assignment 5} \hfill \textbf{John Waczak} \\
\normalsize MTH 483 \hfill  Date: \today \\
\par\noindent\rule{\textwidth}{0.4pt}

\subsection*{2}
	\begin{proof}[(a)]
		Observe that we can rewrite the function accordingly 
			\begin{equation*}
				(z^2+1)^{-3}(z-1)^{-4}=((z-i)(z+i))^{-3}(z-1)^{-4} = \frac{1}{(z-i)^3(z+i)^3(z-1)^4}
			\end{equation*}
		Then it is clear that the function has 3 poles $\{i, -i, 1\}$ with multiplicities $\{3, 3, 4\}$. 
	\end{proof}
	
	\begin{proof}[(b)]
		Consider $f(z) = z\cot z$ which can be rewritten as $f(z) = \frac{z\cos z}{\sin z}$. Because $\sin z$ and $\cos z$ are $\pi/2$ out of phase, they never share a zero, thus we have poles of order 1 whenever $\sin(z) = 0$ i.e. when $z=n\pi$ such that $n\in\mathbb{Z}\setminus\{0\}$. Here $0$ is a special case as we have the indeterminant form $\frac{0}{0}$. This singularity is removable as can be seen by the power series:
			\begin{align*}
				f(z) &= \frac{\cos z}{z^{-1}\sin(z)} \\ 
					&= \frac{\sum\limits_{n=0}^{\infty}(-1)^n\frac{z^{2n}}{(2n)!}}{z^{-1}\sum\limits_{n=0}^{\infty}(-1)^n\frac{z^{2n+1}}{(2n+1)!}} \\ 
					&= \frac{\sum\limits_{n=0}^{\infty}(-1)^n\frac{z^{2n}}{(2n)!}}{\sum\limits_{n=0}^{\infty}(-1)^n\frac{z^{2n}}{(2n+1)!}} \\ 
					&= \frac{1-\frac{z^2}{2!}+\frac{z^4}{4!}-...}{1-\frac{z^2}{3!}+\frac{z^4}{5!}+...}
			\end{align*}
		And from the last line it is easy to see that as $z\to 0$, $f(z)\to 1$ indicating that $z=0$ is in fact a removable singularity. 
	\end{proof}
	
	\begin{proof}[(c)]
		The function $f(z) = \sin(z)z^{-5}$ has a pole of order 4 when $z=0$ because (as we saw in the previous part) the power series for $\sin(z)$ is entire and has a power series looking like $z-\frac{z^3}{3!}+\frac{z^5}{5!}-...$. Thus dividing by $z^5$ cancels out the first z-term leaving $z^{-4}$.  
	\end{proof}
	
	
\par\noindent\rule{\textwidth}{0.4pt}
\subsection*{5}
	\begin{proof}[(a)]
		We can find $\oint_\gamma \cot(z) dz$ by using the argument principle. Observe that $\cot z = \frac{\cos z}{\sin z} = \frac{\sin'z}{\sin z}$. The argument principal gives that $\oint_\gamma \frac{f'(z)}{f(z)}dz =2\pi i \big[ Z(f, \gamma)-P(f,\gamma)\big]$. $\gamma = C[0,3]$ the circle of radius 3 centered about $z=0$. Therefore we have that $f(z)=\sin(z)$ has no poles inside of $\gamma$ and and one zero when $z=0$. Thus 
			\begin{equation*}
				\oint_\gamma \cot(z)dz = 2\pi i 
			\end{equation*}
	\end{proof}
	
	
	\begin{proof}[(c)]
		We want to find the value of $\oint_\gamma \frac{dz}{(z+4)(z^2+1)}$. We can do this using the Residue theorem. First observe that the integral can be rewritten as 	$\oint_\gamma \frac{dz}{(z+4)(z-i)(z+i)}$ which has 3 simple poles of order 1 at $\{-4,i,-i\}$. Only the poles at $i, -i$ lie within $\gamma$, thus
			\begin{align*}
				\oint_\gamma \frac{dz}{(z+4)(z-i)(z+i)} &= 2\pi i\sum_i Res[f, z_i] \\ 
					&= 2\pi i \big[\frac{1}{(i+4)(i+i)}+\frac{1}{(-i+4)(-i-i)}\big] \\ 
					&= -\frac{2\pi i}{17}
			\end{align*}
		
	\end{proof}
	
	
\par\noindent\rule{\textwidth}{0.4pt}
\subsection*{8}
	\begin{proof}[(c)]
		We want to evaluate the following integral for $\gamma = C[0, 2]$: $\oint_\gamma \frac{\exp(z)}{z^3+z}dz$. First rearrange the integral to become
			\begin{equation*}
				\oint_\gamma \frac{\exp(z)}{z(z^2+1)}dz = \oint_\gamma \frac{\exp(z)}{z(z-i)(z+i)}dz
			\end{equation*}
		From this last equation we see that the integrand has 3 simple poles at $z=0,i,-i$. All are included within $\gamma$. Therefore, our integral evaluates to
			\begin{align*}
				\oint_\gamma f(z)dz &= 2\pi i \sum_i Res[f, z_i] \\ 
					&= 2\pi i \Big[ \frac{\exp(0)}{-i(i)}+ \frac{\exp(i)}{i(2i)}+\frac{\exp(-i)}{-i(-2i)} \Big]  \\ 
					&= 2\pi i \big[ 1-\frac{\cos(1)+i\sin(1)}{2}-\frac{\cos(1)-i\sin(1)}{2} \big] \\
					&= 2\pi i (1-\cos(1))
			\end{align*}
	\end{proof}
	
	\begin{proof}[(d)]
		We want to evaluate the integral $\oint_\gamma \frac{dz}{z^2\sin z}$ where $\gamma = C[0,1]$. This function has only one pole inside of $\gamma$ when $z=0$ with order 3. Thus we have that 
			\begin{align*}
				\oint_\gamma f(z) dz &= 2\pi i Res(f, z=0) \\ 
					&= 2\pi i \frac{1}{2}\lim_{z\to 0}\frac{d^2}{dz^2}z^3f(z) \\ 
					&=\pi i \lim_{z\to 0} \frac{d^2}{dz^2} \frac{z}{\sin z} \\ 
					&= \pi i \lim_{z\to 0} \csc(z)[z\cot^2(z)-2\cot(z)+z\csc^2(z)] \\
					&= \frac{\pi i}{3}
			\end{align*}
	\end{proof}
	
	\begin{proof}[(e)]
		We want to evaluate the integral $\oint_\gamma f(z)dz$ for $f(z) = \frac{\exp(z)}{(z+2)^2\sin(z)}$ $\gamma = C[0,3]$. This function has a simple pole when $z=0$ and a pole of multiplicity 2 when $z=-2$. Both of these poles lie within $\gamma$ and therefore, 
			\begin{align*}
				\oint_\gamma f(z)dz &= 2\pi i [Res(f, z=0)+Res(f, z=-2)] \\ 
					&= 2\pi i \Big[ \lim_{z\to0}zf(z) + \lim_{z\to -2}\frac{d}{dz}(z+2)^2f(z)  \Big] \\ 
					&= 2\pi i \Big[ \frac{1}{4} + \frac{((-1 - \cot(2))\csc(2))}{e^2} \Big]
			\end{align*}
	\end{proof}
	
	
\par\noindent\rule{\textwidth}{0.4pt}
\subsection*{21}
	\begin{proof}[(a)]
		We want to find the number of zeros of $3\exp(z)-z$ in $\overline{D}[0,1]$. By Rouche's theorem we can see that $|-z| = 1$ is less than $|3\exp(z)|=3$ $\forall z \in \gamma$ (the boundary). So define $f(z) = 3\exp(z)$ and $g(z) = -z$. Then, 
			\begin{align*}
				Z(f+g, \gamma) = Z(f, \gamma)
			\end{align*} 
		However here we see that the function $f(z)=3\exp(z)$ has no zeros and therefore as we shrink the boundary, we do not find any zeros. Thus we conclude that the function $3\exp(z)-z$ has no zeros inside of $\overline{D}[0,1]$. 
	\end{proof}
	
	\begin{proof}[(b)]
			We want to find the number of zeros of $\frac{1}{3}\exp(z)-z$ in $\overline{D}[0,1]$. In analogy to part (a), observe that $|f(z)|=|\frac{1}{3}\exp(z)|=\frac{1}{3}e^{\Re(z)} \leq\frac{1}{3}e^{1} = \frac{1}{3}e$ and $|g(z)|=|-z|=1$ on the boundary. Thus by applying Rouche's theorem, we have that the number of zeros $Z(f+g, \gamma) = Z(g, \gamma)$ since $|g(z)|\geq |f(z)|$. Therefore, there are no zeros until we shrink $\gamma$ down to radius $0$ at which point $g(z)=-z$ has a zero. Therefore we conclude that $f(z)+g(z) = \frac{1}{3}\exp(z)-z$ has one zero inside of $\overline{D}[0,1]$. 
	\end{proof}
	
	\begin{proof}[(c)]
		We want to find the zeros of $z^4-5z+1$ inside of $\{z\in\C:1\leq |z| \leq 2 \}$. Define $f(z) = z^4$ and $g(z) = -5z+1$. The total number of zeros in our region should be the zeros inside of $C[0,2]$ minus those in $C[0,1]$. Thus for the first case observe that $|f(z)| = |z^4| \leq 2^4 = 16$ on the outer boundary. Then $|g(z)| = |-5z+1|\leq 5\cdot 2 + 1 = 11$ Therefore Rouche's theorem gives us that the number of zeros for $f(z)+g(z)$ inside of $C[0,2]$ is given by the number of zeros of $f(z)$ which is 4 by the fundamental theorem of algebra. \\
		
		\noindent For the second one we have that $|g(z)| = |-5z+1| \leq 5+1 = 6$ on the boundary of $C[0,1]$ and $|f(z)| = |z^4|\leq 1^4 = 1$ on the same boundary. Thus the number of zeros for $f(z)+g(z)$ in $C[0,1]$ is given by the number of zeros of $g(z)$ in $C[0,1]$. This number is 1 (when $z=1/5$). Therefore we conclude that that the total number of zeros in the annulus is $4-1 =$ \textbf{3}. 
	
	\end{proof} 
	
	
	
	
	
	
	
	
	
	
	
	
	
	
	
	
	
	
	
	
	
\end{document}