\documentclass[a4paper, 11pt]{article}
\usepackage{geometry}
\geometry{letterpaper, margin=1in}
\usepackage{amsmath}
\usepackage{amssymb}  
\usepackage{amsthm}
\usepackage{ulem} 
\usepackage{graphicx}
\usepackage{enumitem} 
\graphicspath{ {images/} }


\newtheorem*{theorem}{Theorem}

\begin{document}
%Header-Make sure you update this information!!!!
\noindent
\large\textbf{Complex Analysis: Day 19} \hfill \textbf{John Waczak} \\
\normalsize MTH 483 \hfill  Date: \today \\

\subsection*{Poles} 
	\par\noindent\rule{\textwidth}{0.4pt}
	
	\noindent \textit{Recall:} If $f:\Omega\to\mathbb{C}$ is holomorphic and $a\in\Omega$ then there exists some $R>0$ for which $f$ can be expressed in a power series:
		\begin{equation*}
			f(z) = \sum\limits_{n=0}^\infty c_n(z-a)^n 
		\end{equation*}
	for all $z\in D_R(a)$, and $D_R(a)\subseteq\Omega$. Where
		\begin{equation*}
			c_n = \frac{f^{(n)}(a)}{n!}
		\end{equation*}

	\noindent \textbf{Definition:} If $f(z)$ is holomorphic and $f(a)=0$, we call $a$ a \textit{zero} of $f$. \\
	
	\begin{theorem}
		Let $f:\Omega\to\mathbb{C}$ be a holomorphic function and $a\in\Omega$ be a zero of $f$. Then one of the following two cases occurs: 
			\begin{enumerate}[label=\alph*]
				\item There exists a disc $D$ centered at $a$ for which $f(z)=0$ $\forall z\in D$. 
				\item There exists an integer $m\geq1$ and a holomorphic function $g:\Omega\to\mathbb{C}$ such that $f(z)=(z-a)^mg(z)$ for all $z\in\Omega$ and $g(a)\neq 0$. 
			\end{enumerate}
		
		\noindent \textit{Example:} $f(z)=z^3-2z^2+z$, $a=1$. 
			\begin{align*}
				f(1) &= 1^3-2+1 = 0 \\ 
				f(z) &= z(z^2-2z+1) \\ 
					&= z(z-1)^2 \\ 
				\text{thus } m&=2 \quad g(z)=z 
			\end{align*}
	\end{theorem}


	\noindent\textbf{Definition} In case (b) we say that m is the \textit{multiplicity} of the zero $a$. 
	
	\begin{proof}
		Expand $f$ into its power series
			\begin{equation*}
				f(z) = \sum\limits_{n=0}^\infty c_n(z-a)^n
			\end{equation*}
		where $|z-a|<R$ with $R>0$. Let $D=D_R(a)$. If $c_n=0$ $\forall n$ then clearly case (a) holds. Otherwise we have $c_m\neq 0$ for some $m\geq 1$ and we make take $m$ minimal with this property. We already know that $c_0=0$ since $f(a)=c_0$ and we are assuming that $a$ is a zero. Thus the power series starts at m and looks like
			\begin{equation*}
				f(z) = \sum\limits_{n=m}^\infty c_n(z-a)^n 
			\end{equation*}
		and so we can observe that
			\begin{align*}
				&= (z-a)^m\Big\{ c_m + c_{m+1}(z-a) + c_{m+2}(z-a)^{2} + ... \Big\} \\
				&= (z-a)^m g(z) \\ 
			\text{where } g(z) &= \sum\limits_{k=0}^\infty c_{m+k}(z-a)^k 
			\end{align*}
		
		\noindent If instead $z\neq a$, also define 
			\begin{equation*}
				g(z) = \frac{f(z)}{(z-a)^m} \quad (z\in\Omega\setminus\{a\})
			\end{equation*}
		This defines $g:\Omega\to\mathbb{C}$. \textbf{Note:} points $z\in D\setminus\{a\}$ have two definitions of g(z), but the two definitions are equal.
	\end{proof}
	
	\noindent\textit{Example} $f(z) = \sin^2(z)$, $a=0$ 
		\begin{align*}
			\sin^2(z) &= \sum\limits_{n=0}^\infty c_n z^n \\ 
			f'(z) &= 2\sin(z)\cos(z) \\ 
			f''(z)&= 2\cos^2(z)-2\sin^2(z) \\ 
			\Rightarrow c_0 &= f(0) = 0 \\ 
				c_1 &= f'(0) = 0 \\ 
				c_2 &= f''(0)/2 = 1 \neq 0 \\ 
			\Rightarrow m&= 1  
		\end{align*}
	\noindent the function
		\begin{equation*}
			g(z) =	\begin{cases}
				\frac{\sin^2(z)}{z^2} & z\neq 0 \\ 
				1 & z=0 
			\end{cases}
		\end{equation*}
	\noindent is holomorphic on $\mathbb{C}$ and $\sin^2(z) = z^2g(z)$. 
	





















	
\end{document}
































