\documentclass[a4paper, 11pt]{article}
\usepackage{geometry}
\geometry{letterpaper, margin=1in}
\usepackage{amsmath}
\usepackage{amssymb}  
\usepackage{amsthm}
\usepackage{ulem} 
\usepackage{graphicx}
\graphicspath{ {images/} }
\usepackage{tikz} 
\begin{document}
%Header-Make sure you update this information!!!!
\noindent
\large\textbf{Complex Analysis - MTH 483} \hfill \textbf{John Waczak} \\
\normalsize Day 10 \hfill  Date: \today \\

\subsubsection*{More on integration}
If $f:\Omega \rightarrow \mathbb{C}$ is a function and $\gamma:[a,b]\rightarrow\Omega$ is a curve, then $\int_\gamma f(z)dz = \int_a^b f(\gamma(t))\gamma'(t)dt$\\

\noindent \textbf{Definition} we say two curves are \textit{equivalent} if
	\begin{align*}
		\gamma_1&:[a,b]\rightarrow\mathbb{C} \\ 
		\gamma_2&:[c,d]\rightarrow\mathbb{C} 
	\end{align*}
and there exists a continuous function $u:[a,b]\rightarrow[c,d]$ with $u(a)=c$ and $u(b)=d$ differentiable with $u'(t)>0$ such that $\gamma_1(t) = \gamma_2(u(t))$. In this situation, $\gamma_1$ and $\gamma_2$ define the same curves, just parametrized differently. \\

\noindent \textbf{Proposition} if $\gamma_1, \gamma_2$ are equivalent, then $\int_{\gamma_1}f(z)dz = \int_{\gamma_2}f(z)dz$. \\

\noindent \textbf{Definition} the length of a curve $\gamma$ is given by: 
	\begin{equation*}
		\text{length}\gamma = \int_a^b |\gamma'(t)|dt
	\end{equation*}

\noindent \textit{Example} 
	$\gamma:[0,2]\rightarrow\mathbb{C}$ such that $\gamma(t) = (1+3i)t$ (a straight line). 
		\begin{align*}
			\text{length}\gamma &= \int_0^2 |1+3i|dt \\ 
				&= \sqrt{10}t\Big|_0^2 \\ 
				&= 2\sqrt{10} 
		\end{align*}

\subsubsection*{Properties of path integrals} 
	\begin{enumerate}
		\item $\int_\gamma (af(z)+bg(z))dz = \int_\gamma af(z)dz + \int_\gamma bg(z)dz$ i.e. integration is linear
		
		\item if $-\gamma$ denotes the path $\gamma$ traversed in opposite direction, then $\int_{-\gamma}f(z)dz = -\int_\gamma f(z)dz$. 
		
		\item if $\gamma_1:[a,b]\rightarrow \mathbb{C}$ and $\gamma_2:[c,d]\rightarrow\mathbb{C}$ with $\gamma_1(b) = \gamma_2(c)$ then if $\gamma_3$ is obtained by $\gamma_1$ then $\gamma_2$ we have:
			\begin{equation*}
				\int_{\gamma_3}f(z)dz = \int_{\gamma_1}f(z)dz + \int_{\gamma_2}f(z)dz
			\end{equation*}
		
		\item $|\int_\gamma f(z)dz| \leq \max\{|f(z)|\:z\in\gamma\}\cdot\text{length}(\gamma)$. Think of bounding an integral by making a square with its maximum value i.e. $|\int_a^b f(x)dx|\leq M(b-a)$. 
	\end{enumerate}
	
\end{document}


















































