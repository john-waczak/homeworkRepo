\documentclass[a4paper, 11pt]{article}
\usepackage{geometry}
\geometry{letterpaper, margin=1in}
\usepackage{amsmath}
\usepackage{amssymb}  
\usepackage{amsthm}
\usepackage{ulem} 
\usepackage{graphicx}
\usepackage{enumitem} 
\graphicspath{ {images/} }


\newtheorem*{theorem}{Theorem}
\newtheorem*{corollary}{Corollary}
\newtheorem*{lemma}{Lemma}
\newtheorem*{definition}{Definition}
\newtheorem*{proposition}{Proposition}

\begin{document}
%Header-Make sure you update this information!!!!
\noindent
\large\textbf{Complex Analysis: Day 23} \hfill \textbf{John Waczak} \\
\normalsize MTH 483 \hfill  Date: \today \\


	
	
	
\subsection*{Residues} 
	If $f(z)$ has a simple pole at $a$ (a pole of multiplicity 1) then 
		\begin{equation*}
			Res_a(f) = \lim\limits_{z\to a}(z-a)f(z) 
		\end{equation*}
	
	\noindent More generally, suppose that $f(z)$ has a pole of multiplicity $m\geq 1$. Then 
		\begin{equation*}
			Res_a(f) = \lim\limits_{z\to a}\Big\{\frac{1}{(m-1)!}\Big(\frac{d}{dz}\Big)^{m-1}(z-a)^mf(z) \Big\}
		\end{equation*}
	
	
	
	
\end{document}
