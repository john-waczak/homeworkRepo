\documentclass[a4paper, 11pt]{article}
\usepackage{geometry}
\geometry{letterpaper, margin=1in}
\usepackage{amsmath}
\usepackage{amssymb}  
\usepackage{amsthm}
\usepackage{ulem} 
\usepackage{graphicx}
\usepackage{enumitem} 
\graphicspath{ {images/} }


\newtheorem*{theorem}{Theorem}
\newtheorem*{corollary}{Corollary}
\newtheorem*{lemma}{Lemma}
\newtheorem*{definition}{Definition}
\newtheorem*{proposition}{Proposition}

\begin{document}
%Header-Make sure you update this information!!!!
\noindent
\large\textbf{Complex Analysis: Day 22} \hfill \textbf{John Waczak} \\
\normalsize MTH 483 \hfill  Date: \today \\

\subsection*{Heading towards Residues} 
	Recall that last time we defined 3 types of singularities
		\begin{enumerate}
			\item Removable
			\item Pole 
			\item Essential
		\end{enumerate}
	\noindent Removable singularities mean it's possible to find an alternative function that agrees with $f$ and is defined at the singularity. A \textit{pole} had $\lim_{z\to z_0}=+\infty$. Essential was neither removable nor a pole.\\ 
	
	\begin{theorem}
		If $f:\Omega\setminus\{z_0\}\to\mathbb{C}$ is holomorphic and there exists $r>0$ for which $f(z)$ is bounded on $D_r(z_0)\setminus\{z_0\}$, then the singularity is removable. Using this we can show the following
	\end{theorem}
	
	\begin{proposition}
		If $f:\Omega\setminus\{z_0\}\to\mathbb{C}$ is holomorphic, then $f$ has a pole at $z_0$ if and only f $\frac{1}{f(z_0)}$ has a zero when $f(z)$ has a singularity. 
	\end{proposition}
	
	\noindent \textit{Example} $f(z) = \frac{z^2}{(z-1)(z+2)}$ $g(z) = \frac{(z-1)(z-2)}{z^2}$ \\ 
	f(z) has zero at $z=0$ and poles at $z=1, z=-2$. \\ 
	
	
	
	\begin{proposition}
		Let $f:\Omega\setminus\{a\}\to\mathbb{C}$ be holomorphic w/ a pole at $a$ of multiplicity $m\geq 1$. Then there exists $r>0$ such that for all $z\in D_r(a)\setminus\{a\}$ we have that:
			\begin{equation}
				f(z) \frac{a_{-m}}{(z-a)^m} + \frac{a_{-m+1}}{(z-a)^{m-1}}+...+\frac{a_{-1}}{z-a}+a_0+a_1(z-a)+... = \sum_{k=-m}^{\infty}a_k(z-a)^k 
			\end{equation}
	\end{proposition}
	
	
	
	
	
	
	
	
	
	
	
\end{document}




































