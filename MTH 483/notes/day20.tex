\documentclass[a4paper, 11pt]{article}
\usepackage{geometry}
\geometry{letterpaper, margin=1in}
\usepackage{amsmath}
\usepackage{amssymb}  
\usepackage{amsthm}
\usepackage{ulem} 
\usepackage{graphicx}
\usepackage{enumitem} 
\graphicspath{ {images/} }


\newtheorem*{theorem}{Theorem}
\newtheorem*{corollary}{Corollary}
\newtheorem*{lemma}{Lemma}
\newtheorem*{definition}{Definition}

\begin{document}
%Header-Make sure you update this information!!!!
\noindent
\large\textbf{Complex Analysis: Day 20} \hfill \textbf{John Waczak} \\
\normalsize MTH 483 \hfill  Date: \today \\


\subsection*{more stuff} 
	\par\noindent\rule{\textwidth}{0.4pt}
	\begin{theorem}[Identity Principle]
		Let $f:\Omega\to\mathbb{C}$ and $g:\Omega\to\mathbb{C}$ be holomorphic on a region $\Omega$ and suppose $\{\alpha_k\}$ is a sequence of distinct complex numbers in $\Omega$ converging to $\alpha \in \Omega$. Suppose that $f(\alpha_k)=g(\alpha_k)$ for all $k\geq 1$ Then $f(z)=g(z)$ for all $z\in\Omega$. This shows that holomorphic functions are very rigid. 	
	\end{theorem}
	
	\begin{proof}
		Define $h = f-g$. In other words $h(\alpha_k) = 0$ $\forall k$. Now define two sets
			\begin{align*}
				X &= \{a\in\Omega : \exists \text{ some } r>0 \text{ for which} h(a)=0 \quad \forall z\in D_R(a)\}\\ 
				Y &= \{a\in\Omega : \exists r>0 \text{ s.t. } h(a)\neq 0 \quad \forall z\in D_R(a)\setminus\{a\}\}
			\end{align*}
		Note that $X\cap Y = \emptyset$ and $X\cup Y = \Omega$. Also note that both $X$ and $Y$ are open. Given $a\in X \text{ or } Y$ points close enough to a are also in $X$ or $Y$. Thus $\Omega$ is the disjoint union of two open sets. By the definition of \textit{connectedness} one of these sets must be $\emptyset$. We know that $\alpha = \lim\limits_{k\to\infty} \alpha_k \in X$. Thus $X\neq \emptyset$. Therefore $\Omega = X$ and $h(z)=0 \quad \forall z\in\Omega$. This implies that $f(z)=g(z) \quad \forall z \in \Omega$. 
	\end{proof}

	\begin{corollary}
		Let $\Omega_1 \subseteq \Omega_2$ be regions, and let $f:\Omega_1\to\mathbb{C}$ be holomorphic. If $\exists$ holomorphic function $F:\Omega_2\to\mathbb{C}$ such that $F(z)=f(z)$ for all $z\in\Omega_1$ then $F$ is unique. 
	\end{corollary}
		
	\begin{proof}
		If $F,G:\Omega_2\to\mathbb{C}$ are holomorphic and $F(z)=G(z) = f(z)$ for all $z\in \Omega_1$ then $F(z)=G(z)$ $\forall z\in\Omega_2$ by the identity principle. 
	\end{proof}

	\begin{definition}
		If such a $F:\Omega_2\to\mathbb{C}$ exists it is called an \textit{analytic continuation} of  $f$. 
	\end{definition}

	\noindent \textit{Example} $\Omega_1 = \{z=x+iy : x>1\}$
		\begin{align*}
			f:\Omega_1&\to \mathbb{C} \\ 
			f(z) &= \sum\limits_{n=1}^\infty \frac{1}{n^z}
		\end{align*} 
	It turns out that this function $f(z)$ has an analytic continuation $F:\Omega_2\to\mathbb{C}$ where $\Omega_2 = \mathbb{C}\setminus\{1\}$. The \textit{Riemann Hypothesis} says that all of the zeros of $F(z)$ in the strip $0<x<1$ occur on the line $x=1/2$. 

	
\end{document}
















