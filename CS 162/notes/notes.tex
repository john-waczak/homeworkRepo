\documentclass[a4paper, 11pt]{article}
\usepackage{geometry}
\geometry{letterpaper, margin=1in}
\usepackage{graphicx}
\graphicspath{ {images/} }

\usepackage{amsmath}
\usepackage{amssymb}  
\usepackage{amsthm}
\usepackage{ulem}

\usepackage{enumitem}


\usepackage{pdfpages} % for including full pdf pages

\usepackage{empheq}

\usepackage{listings}


%format to allow bolded theorems, corollaries, etc...
\newtheorem*{theorem}{Theorem}
\newtheorem*{corollary}{Corollary}
\newtheorem*{lemma}{Lemma}
\newtheorem*{definition}{Definition}
\newtheorem*{Example}{Example} 
\newtheorem*{Remark}{Remark}

% stop typing \mathbb a thousand times 
\newcommand{\R}{\mathbb{R}}
\newcommand{\C}{\mathbb{C}}
\newcommand{\F}{\mathbb{F}}
\newcommand{\E}{\mathbb{E}}
\newcommand{\sphere}{\mathbb{S}}

% commands for bra-ket notation
\newcommand{\bra}[1]{\ensuremath{\left\langle#1\right|}}
\newcommand{\ket}[1]{\ensuremath{\left|#1\right\rangle}}
\newcommand{\bracket}[2]{\ensuremath{\left\langle #1 \middle| #2 \right\rangle}}
\newcommand{\matrixel}[3]{\ensuremath{\left\langle #1 \middle| #2 \middle| #3 \right\rangle}}
\newcommand{\expectation}[1]{\ensuremath{\left\langle #1 \right\rangle}}

% vector stuff
\newcommand{\basis}[1]{\hat{\mathbf{e}}_#1}
\newcommand{\unit}[1]{\hat{\boldsymbol{#1}}}
\newcommand{\bvec}[1]{\vec{\boldsymbol{#1}}}


% change margins for solution
\newenvironment{solution}{%
	\begin{list}{}{%
			\setlength{\topsep}{0pt}%
			\setlength{\leftmargin}{0.5cm}%
			\setlength{\rightmargin}{0.5cm}%
			\setlength{\listparindent}{\parindent}%
			\setlength{\itemindent}{\parindent}%
			\setlength{\parsep}{\parskip}%
		}%
		\item[]}{\end{list}}



\begin{document}
\noindent
\large\textbf{Week 1 Notes} \hfill \textbf{John Waczak} \\
\normalsize CS 162 \hfill  Date: \today 
\par\noindent\rule{\textwidth}{0.4pt} \\\\

% \begin{lstlisting}[language=C++]
% \end{lstlisting}

\section*{Objects \& Structures}
Objects are a way to think about grouping together similar items in order to
organize their structure and inter-relations. In programming, objects are a way
to group variables together (as in arrays). \textit{Arrays} allow us to group
together objects so long as they are the same type.

\begin{lstlisting}[language=C++]
  int grades[150]; 
\end{lstlisting}
But what if we want more flexibility? We need to move away from the
\textit{primitive} variable types and begin to group different variables
together in order to make a sort of container.\\ 

\textbf{Structs} are custom objects (structures) that allow us to mix and match
data types. Traditionally, structures contain only data and no member functions
i.e. a clump of related variables. The following shows an example. \\


\begin{lstlisting}[language=C++]
  struct book{
    int pages;
    unsigned in pub_date;
    string title; // a string inside the struct
    int num_authors;
    string* authors; // a pointer to a string
  };

  // declare a book struct
  book text_book; 
\end{lstlisting}
\vspace{1em}

Inside of \textit{book} we have defined a number of useful variables such as a
string for the title, an int for the number of pages, etc... How do we access
these?


\begin{lstlisting}[language=C++]
  book bookshelf[10];
  for (int i = 0; i < 10; i++){
    bookshelf[i].num_pages = 100;
    bookshelf[i].title = ``Place holder''
    bookshelf[i].authors = new string[2]; // dynamically allocate array 
  }
\end{lstlisting}


\section*{Pointers}
\begin{itemize}
\item Pointers $==$ memory addresses.
  
\item Variable declaration creates a variable on the stack

\begin{lstlisting}[language=C++]
int a = 5;
\end{lstlisting}
  
\item Pointer declaration
  
\begin{lstlisting}[language=C++]
int* b = &a; 
\end{lstlisting}
This creates a pointer variable of type \textbf{int} which points to the address
of \textbf{a} (using the address operator \&)

\item Dereferencing a pointer:
  
\begin{lstlisting}[language=C++]
cout << *b << endl; 
\end{lstlisting}
This will take the pointer b and grab the variable held at that memory location. 

\end{itemize}



\section*{Array}
\begin{itemize}
  \item An \textit{array} is a collection variables of one data type and its
    memory is stored contiguously

  \item \textit{static arrays} are created on the stack and are of a fixed size

\begin{lstlisting}[language=C++]
int stack_array[10]; 
\end{lstlisting}
    
  \item \textit{Dynamic arrays} are created on the heap and their size may
    change during runtime.

    
\begin{lstlisting}[language=C++]
int *heap_array = new int[10]; 
\end{lstlisting}

  \item Arrays can be of one ore more dimensions

    
\begin{lstlisting}[language=C++]
int stack_array_2d[5][7]; 
\end{lstlisting}


    

    
\end{itemize}




\end{document}

































