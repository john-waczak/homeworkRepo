\documentclass[a4paper, 11pt]{article}
\usepackage{geometry}
\geometry{letterpaper, margin=1in}
\usepackage{graphicx}
\graphicspath{ {images/} }

\usepackage{amsmath}
\usepackage{amssymb}  
\usepackage{amsthm}
\usepackage{ulem}

\usepackage{enumitem}


\usepackage{pdfpages} % for including full pdf pages

\usepackage{empheq}

\usepackage{listings}


%format to allow bolded theorems, corollaries, etc...
\newtheorem*{theorem}{Theorem}
\newtheorem*{corollary}{Corollary}
\newtheorem*{lemma}{Lemma}
\newtheorem*{definition}{Definition}
\newtheorem*{Example}{Example} 
\newtheorem*{Remark}{Remark}

% stop typing \mathbb a thousand times 
\newcommand{\R}{\mathbb{R}}
\newcommand{\C}{\mathbb{C}}
\newcommand{\F}{\mathbb{F}}
\newcommand{\E}{\mathcal{E}}
\newcommand{\prob}[2]{\mathcal{P}_{_{#1\rightarrow #2}}}
\newcommand{\M}{\mathbb{M}}
\newcommand{\sphere}{\mathbb{S}}

% commands for bra-ket notation
\newcommand{\bra}[1]{\ensuremath{\left\langle#1\right|}}
\newcommand{\ket}[1]{\ensuremath{\left|#1\right\rangle}}
\newcommand{\bracket}[2]{\ensuremath{\left\langle #1 \middle| #2 \right\rangle}}
\newcommand{\matrixel}[3]{\ensuremath{\left\langle #1 \middle| #2 \middle| #3 \right\rangle}}
\newcommand{\expectation}[1]{\ensuremath{\left\langle #1 \right\rangle}}

% vector stuff
\newcommand{\basis}[1]{\hat{\mathbf{e}}_#1}
\newcommand{\unit}[1]{\hat{\boldsymbol{#1}}}
\newcommand{\bvec}[1]{\vec{\boldsymbol{#1}}}
\newcommand{\threevec}[2]{\begin{pmatrix} #1 \\ #2 \end{pmatrix}}

% change margins for solution
\newenvironment{solution}{%
	\begin{list}{}{%
			\setlength{\topsep}{0pt}%
			\setlength{\leftmargin}{0.5cm}%
			\setlength{\rightmargin}{0.5cm}%
			\setlength{\listparindent}{\parindent}%
			\setlength{\itemindent}{\parindent}%
			\setlength{\parsep}{\parskip}%
		}%
		\item[]}{\end{list}}




\begin{document}
\noindent
\large\textbf{Homework 5} \hfill \textbf{John Waczak} \\
\normalsize PH 653 \hfill  Date: \today \\
Dr. Oksana Ostroverkhova \hfill worked w/ Ryan Tollefsen
\par\noindent\rule{\textwidth}{0.4pt} \\\\



\begin{enumerate}[leftmargin=0em, label=\textbf{\arabic*}]
  \item  Imagine a situation in which there are  three particles and only three
    states a, b, and c available to them. What is the total number of allowed,
    distinct configurations for the following systems: 
  
    \begin{enumerate}[leftmargin=2em, label=(\textbf{\alph*})]
      \item Labeled (i.e. distinguishable) particles \\
        \begin{solution}
          If we consider 3 distinguishable particles $\{1, 2, 3\}$ and three possible states
          $\{ \ket{a}, \ket{b}, \ket{c} \}$ then each particle can occupy any of
          the three states so that there are
          \begin{equation}
            3^3 = 27
          \end{equation}
          possible states in total. If we insist that states of the particles
          must be different, then there are
          \begin{equation}
            3! = 6
          \end{equation}
          possible states.
          
        \end{solution}

      \item identical bosons\\
        \begin{solution}
          If instead the particles are bosons, multiple particles may occupy the
          same state and the state vector describing the system must by
          \textit{symmetric} under exchange. That is,
          \begin{equation}
            \hat{P}_{ij}\ket{\psi_A} = \ket{\psi_A}
          \end{equation}
          where $\hat{P}_{ij}$ denotes the permutation operator which exchanges the
          states of the $i^{th}$ and $j^{th}$ particles. We will begin
          counting by first examining states for which each particle is in the
          same state and then will work up to the case for which each particle
          occupies a different state. \\

          Hereafter, we assume the notation $\ket{a; b; c}$ for the combined
          state vector which indicates particle one is in state a, particle two
          is in state b, and so forth. With this convention, we can easily see
          that there are only three symmetric cases for which each particle is
          in the same state, namely
          \begin{equation}
            \ket{a;a;a} \qquad \ket{b;b;b} \qquad \ket{c;c;c}
          \end{equation}

          Next, let us consider the possible symmetric state vectors for the
          case where exactly two particles occupy the same state. To account for
          all possible permutations, we must have
          \begin{align}
            &\ket{a;a;b} + \ket{a;b;a} + \ket{b;a;a} \\
            &\ket{b;b;a} + \ket{b;a;b} + \ket{a;b;b} \\
            \nonumber\\
            &\ket{c;c;a} + \ket{c;a;c} + \ket{a;c;c} \\
            &\ket{a;a;c} + \ket{a;c;a} + \ket{c;a;a} \\
            \nonumber\\
            &\ket{b;b;c} + \ket{b;c;b} + \ket{c;b;b} \\
            &\ket{c;c;b} + \ket{c;b;c} + \ket{b;c;c}
          \end{align}
          Where we can think of this as taking a state (for example $\ket{a}$)
          and multiplying by the a symmetric combination of that state with
          another (e.g. $\ket{ab}+\ket{ba}$). 

          Finally, we have the case for which every particle in a different
          state. The symmetric state vector in this configuration is given by
          \begin{equation}
            \ket{a;b;c}+ \ket{a;c;b} + \ket{b;c;a} + \ket{b;a;c}+\ket{c;b;a}+\ket{c;a;b}
          \end{equation}
          Therefore, in total we have 10 possible states. 


          Again, if we insist that the particles must occupy separate states, 
          then the only option is equation (11) so that there is 1 symmetric
          state vector. 
        \end{solution}
        
       
      \item identical fermions
        \begin{solution}
          For the case of three fermions, we expect that the system will behave
          altogether as a fermion and therefore require that the state vector
          describing this system is antisymmetric. That is
          \begin{equation}
            \hat{P}_{ij}\ket{\psi_A} = -\ket{\psi_A}
          \end{equation}
          No state vector for which each particle occupies the same state can be
          antisymmetric. To see if there are any possible antisymmetric states
          where two particles occupy the same sate, we can construct the so
          called Slater determinant

          \begin{align}
            \ket{\psi_{A, 2}} &= \begin{vmatrix} \ket{a} & \ket{a} &\ket{b}\\  \ket{a} & \ket{a} &\ket{b}\\  \ket{a} & \ket{a} &\ket{b}\\ \end{vmatrix} \\
                          & = \ket{a}\Big(\ket{a;b}-\ket{b;a}\Big)-\ket{a}\Big(\ket{a;b}-\ket{b;a} \Big)+\ket{b}\Big(\ket{a;a}-\ket{a;a}\Big)\\
                          &= 0
          \end{align}
          therefore we can clearly see that there is no way we can make an
          antisymmetric state where two particles occupy the same state. This
          result is simply a restatement of the Pauli Exclusion Principle. Thus,
          the only possible way to construct an antisymmetric state for three
          particle with three possible states is by taking the following Slater
          determinant.

          \begin{align}
            \ket{\psi_A} &= \begin{vmatrix}\ket{a}&\ket{b}&\ket{c}\\ \ket{a}&\ket{b}&\ket{c}\\ \ket{a}&\ket{b}&\ket{c}\end{vmatrix} \\
                         &= \ket{a}\Big(\ket{b;c}-\ket{c;b}\Big)-\ket{b}\Big(\ket{a;c}-\ket{c;a}\Big)+\ket{c}\Big(\ket{a;b}-\ket{b;a}\Big)\\
                         &= \ket{a;b;c}-\ket{a;c;b}-\ket{b;a;c}+\ket{b;c;a}+\ket{c;a;b}-\ket{c;b;a}
          \end{align}
          Thus, we conclude that there is exactly one antisymmetric state for
          a three particle, three state system.
        \end{solution}
        
    \end{enumerate}


  \item Two non-interacting particles, with the same mass $m$, are in a 1D box
    of length $2a$. \\
    
    \begin{enumerate}[leftmargin=2em, label=(\textbf{\alph*})]
      \item What are the values of the three lowest energies of the system?
        \begin{solution}
          If the two particles-in-a-box are non-interacting, the Hamiltonian is
          separable in terms of each individual particle so that we may write:
          \begin{equation}
            H = H_1 + H_2
          \end{equation}
          This leads to energies given by 
          \begin{equation}
            E_{n_1,n_2} = \frac{\pi^2\hbar^2}{8ma^2}\left( n_1^2+n_2^2 \right)
          \end{equation}
          The ground state of such a system occurs for $n_1=n_2=0$ with energy
          \begin{equation}
            E_{gs}=E_{11} = \frac{1}{4}\frac{\pi^2\hbar^2}{ma^2}
          \end{equation}

          The first excited state then occurs for either $n_1=1$ $n_2=2$
          \textit{or} $n_1=2$ $n_2=1$  with energy
          \begin{equation}
            E_{12} = E_{21} = \frac{5}{8}\frac{\pi^2\hbar^2}{ma^2}
          \end{equation}

          The second excited state occurs when $n_1=n_2=2$ with energy
          \begin{equation}
            E_{22} = 1\frac{\pi^2\hbar^2}{ma^2}
          \end{equation}

          As a sanity check, the state $n_1=1, n_2=3$ has energy
          \begin{equation}
            E_{13}=E_{31} = \frac{5}{4}\frac{\pi^2\hbar^2}{ma^2} > E_{22}
          \end{equation}
         Thus, we have identified the three lowest energy levels.  

          
        \end{solution}
      \item What are the degeneracies of these energy levels if the two
        particles are:
        \begin{enumerate}[leftmargin=2em, label=(\textbf{\roman*})]
          \item identical, with spin $1/2$;
            \begin{solution}
              A system of two identical fermions obeys Fermi-Dirac statistics
              and must therefore have an antisymmetric overall state vector under
              particle exchange. We must construct all possible
              antisymmetric state vectors for each of the energy levels
              identified in part (a). The total state vector is the tensor
              product of the spatial state (relating to the $n$ quantum number)
              with the spin state (relating to the $m_s$ quantum number). \\ 
              Therefore either the spacial or the spin part must be antiymmetric
              while the opposite is symmetric as discussed in McIntyre (pg 413
              eq 13.10). To that end, we adopt the notation
              \begin{equation}
                \ket{n_1, n_2; m_{s1},m_{s2}}
              \end{equation}
              for the combined, two-particle state. \\

              In the ground state $n_1=n_2$ and therefore, we must have an
              antisymmetric spin component. There is only one way to do this
              \begin{equation}
                \ket{1,1;+,-} - \ket{1,1;-,+}
              \end{equation}
              For the first excited state we have degeneracy in the energy and
              spin so that we have either $\ket{n_1,n_2}_S\ket{m_{s1},m_{s2}}_A$
              or $\ket{n_1,n_2}_A\ket{m_{s1},m_{s2}}_S$. This results in four
              possible states
              \begin{align}
                \big(\ket{1,2}-\ket{2,1}\big)\ket{++}&=\ket{1,2;++}-\ket{2,1;++} \\
                \big(\ket{1,2}-\ket{2,1}\big)\ket{--}&=\ket{1,2;--}-\ket{2,1;--} \\
                \ket{1,2}\big(\ket{+,-}-\ket{-,+}\big)&=\ket{1,2;+,-}-\ket{1,2;-+} \\
                \ket{2,1}\big(\ket{+,-}-\ket{-,+}\big)&=\ket{2,1;+-}-\ket{2,1;-+} 
              \end{align}

              For the second excited state, we again have $n_1=n_2$ so that
              there is only one way to make the overall state antisymmetric:
              \begin{equation}
                \ket{2,2;+,-} - \ket{2,2;-,+}
              \end{equation}
 
            \end{solution}

          \item not identical, but both have spin $1/2$;
            \begin{solution}
              As discussed in class (see page 8 of the lecture 11 notes) a
              system of two non-identical fermions (like a hydrogen atom)
              behaves as a boson. Therefore, we require that the overall state
              vector must be symmetric under particle exchange and consequently,
              both the spatial and spin components of the state must be
              symmetric. \\

              For the ground state where $n_1=n_2$, the spacial component is
              necessarily symmetric. There are 3 ways to make a symmetric spin
              state and therefore, we have 3 total states possible
              \begin{align}
                &\ket{1,1;+,+} \\
                &\ket{1,1;-,-} \\
                &\ket{1,1}\Big(\ket{+,-}+\ket{-,+}\Big) = \ket{1,1;+,-}+\ket{1,1;-,+}
              \end{align}

              For the first excited state we have exactly one way to make a
              symmetric spatial state and (again) three ways to make a symmetric
              spin state. Thus, the options are
              \begin{align}
                \Big(\ket{1,2}+\ket{2,1}\Big)\ket{++} &= \ket{1,2;++}+\ket{2,1;++} \\
                \Big(\ket{1,2}+\ket{2,1}\Big)\ket{--} &= \ket{1,2;--}+\ket{2,1;--} \\
                \Big(\ket{1,2}+\ket{2,1}\Big)\Big(\ket{+,-}+\ket{-,+}\Big) &= \ket{1,2;+,-}+\ket{1,2;-,+} \nonumber \\
                &\quad+\ket{2,1;+,-}+\ket{2,1;-,+} 
              \end{align}

              However, we can also construct an overall symmetric state by
              combining an antisymmetric space state with an antisymmetric spin
              state. The antisymmetric space state is
              \begin{equation}
                \ket{1,2}-\ket{2,1}
              \end{equation}
              and the possible anti-symmetric spine 1 states are
              \begin{align}
                \ket{1,-1}-\ket{-1,1} \\
                \ket{1,0} - \ket{0,1} \\
                \ket{0,-1} - \ket{-1,0}
              \end{align}
              yielding three more possible overall symmetric excited states.
              \begin{align}
                \big(\ket{1,2}-\ket{2,1} \big) \big(\ket{1,-1}-\ket{-1,1}\big)&= \ket{1,2;1,-1}-\ket{1,2;-1,1}\nonumber\\
                &\quad-\ket{2,1;1,-1}+\ket{2,1;-1,1} \\ 
                \big(\ket{1,2}-\ket{2,1} \big) \big(\ket{1,0} - \ket{0,1}\big)&= \ket{1,2;1,0}-\ket{1,2;0,1}\nonumber\\
                &\quad-\ket{2,1;1,0}+\ket{2,1;0,1}\\
                \big(\ket{1,2}-\ket{2,1} \big) \big(\ket{0,-1} - \ket{-1,0}\big)&= \ket{1,2;0,-1}-\ket{1,2;-1,0}\nonumber\\
                &\quad-\ket{2,1;0,-1}+\ket{2,1;-1,0}
              \end{align}
              For a total of 9 possible excited states.  \\
              
              Finally, for the second excited state, we have the same scenario
              as the ground state but with $n_1 = n_2=2$. Therefore, there are
              three more states
              \begin{align}
                &\ket{2,2;+,+} \\
                &\ket{2,2;-,-} \\
                &\ket{2,2}\Big(\ket{+,-}+\ket{-,+}\Big) = \ket{2,2;+,-}+\ket{2,2;-,+}
              \end{align}

            \end{solution}
            
          \item identical, with spin 1;
            \begin{solution}
              We now consider what happens if the two particles are identical
              bosons with spin 1. Because spin 1 particles may have
              z-projections of either $\hbar, 0,$ or $-\hbar$, there are more
              ways to write a symmetric spin state vector. They are:

              \begin{align}
                \ket{1,1} \\
                \ket{0,0} \\
                \ket{-1,-1} \\
                \ket{1,0}+\ket{0,1} \\
                \ket{1,-1} + \ket{-1,1} \\
                \ket{0,-1} + \ket{-1,0} 
              \end{align}

              For the ground state $n_1=n_2=1$ so that the spatial component is
              already symmetric. Therefore, the possible overall symmetric state vectors
              are just
              
              \begin{align}
                \ket{1,1}\ket{1,1} &= \ket{1,1;1,1}\\
                \ket{1,1}\ket{0,0} &= \ket{1,1;0,0}\\
                \ket{1,1}\ket{-1,-1} &= \ket{1,1;-1,-1}\\
                \ket{1,1}\Big(\ket{1,0}+\ket{0,1}\Big) &=\ket{1,1;1,0}+\ket{1,1;0,1}\\
                \ket{1,1}\Big(\ket{1,-1} + \ket{-1,1} \Big) &= \ket{1,1;1,-1}+\ket{1,1;-1,1}\\
                \ket{1,1}\Big(\ket{0,-1} + \ket{-1,0}\Big)&= \ket{1,1;0,-1}+\ket{1,1;-1,0}
              \end{align}

              For the first excited state we have exactly one way to make a
              symmetric spatial state and (again) six ways to make a symmetric
              spin state. Therefore, the possible states are
               
              \begin{align}
                \Big(\ket{1,2}+\ket{2,1}\Big)\ket{1,1} &= \ket{1,2;1,1}+\ket{2,1;1,1}\\
                \Big(\ket{1,2}+\ket{2,1}\Big)\ket{0,0} &= \ket{1,2;0,0}+\ket{2,1;0,0}\\
                \Big(\ket{1,2}+\ket{2,1}\Big)\ket{-1,-1} &= \ket{1,2;-1,-1}+\ket{2,1;-1,-1}\\
                \Big(\ket{1,2}+\ket{2,1}\Big)\Big(\ket{1,0}+\ket{0,1}\Big) &=\ket{1,2;1,0}+\ket{2,1;1,0} \nonumber\\
                &\quad+\ket{1,2;0,1}+\ket{2,1;0,1}\\
                \Big(\ket{1,2}+\ket{2,1}\Big)\Big(\ket{1,-1} + \ket{-1,1} \Big) &= \ket{1,2;1,-1}+\ket{2,1;1,-1}\nonumber\\
                &+\quad\ket{1,2;-1,1}+\ket{2,1;-1,1}\\
                \Big(\ket{1,2}+\ket{2,1}\Big)\Big(\ket{0,-1} + \ket{-1,0}\Big)&= \ket{1,2;0,-1}+\ket{2,1;0,-1}\nonumber\\
                &+\quad\ket{1,2;-1,0}+\ket{2,1;-1,0}
              \end{align}

              Finally, the second excited state is the same as the ground state
              but with $n_1=n_2=2$. This leads to
               
              \begin{align}
                \ket{2,2}\ket{1,1} &= \ket{2,2;1,1}\\
                \ket{2,2}\ket{0,0} &= \ket{2,2;0,0}\\
                \ket{2,2}\ket{-1,-1} &= \ket{2,2;-1,-1}\\
                \ket{2,2}\Big(\ket{1,0}+\ket{0,1}\Big) &=\ket{2,2;1,0}+\ket{2,2;0,1}\\
                \ket{2,2}\Big(\ket{1,-1} + \ket{-1,1} \Big) &= \ket{2,2;1,-1}+\ket{2,2;-1,1}\\
                \ket{2,2}\Big(\ket{0,-1} + \ket{-1,0}\Big)&= \ket{2,2;0,-1}+\ket{2,2;-1,0}
              \end{align}



            \end{solution}
            
        \end{enumerate}
        

    \end{enumerate}
    
  
\end{enumerate}

  
\end{document}



 


























