\documentclass[a4paper, 11pt]{article}
\usepackage{geometry}
\geometry{letterpaper, margin=1in}
\usepackage{graphicx}
\graphicspath{ {images/} }

\usepackage{amsmath}
\usepackage{amssymb}  
\usepackage{amsthm}
\usepackage{ulem}

\usepackage{enumitem}


\usepackage{pdfpages} % for including full pdf pages

\usepackage{empheq}
% format to allow bolded theorems, corollaries, etc... 
\newtheorem*{theorem}{Theorem}
\newtheorem*{corollary}{Corollary}
\newtheorem*{lemma}{Lemma}
\newtheorem*{definition}{Definition}
\newtheorem*{Example}{Example} 
\newtheorem*{Remark}{Remark}

% stop typing \mathbb a thousand times 
\newcommand{\R}{\mathbb{R}}
\newcommand{\C}{\mathbb{C}}
\newcommand{\F}{\mathbb{F}}
\newcommand{\E}{\mathbb{E}}

% commands for bra-ket notation
\newcommand{\bra}[1]{\ensuremath{\left\langle#1\right|}}
\newcommand{\ket}[1]{\ensuremath{\left|#1\right\rangle}}
\newcommand{\bracket}[2]{\ensuremath{\left\langle #1 \middle| #2 \right\rangle}}
\newcommand{\matrixel}[3]{\ensuremath{\left\langle #1 \middle| #2 \middle| #3 \right\rangle}}
\newcommand{\expectation}[1]{\ensuremath{\left\langle #1 \right\rangle}}

% vector stuff
\newcommand{\basis}[1]{\hat{\mathbf{e}}_#1}
\newcommand{\unit}[1]{\hat{\boldsymbol{#1}}}
\newcommand{\bvec}[1]{\vec{\boldsymbol{#1}}}


% change margins for solution
\newenvironment{solution}{%
	\begin{list}{}{%
			\setlength{\topsep}{0pt}%
			\setlength{\leftmargin}{0.5cm}%
			\setlength{\rightmargin}{0.5cm}%
			\setlength{\listparindent}{\parindent}%
			\setlength{\itemindent}{\parindent}%
			\setlength{\parsep}{\parskip}%
		}%
		\item[]}{\end{list}}



\begin{document}
\noindent
\large\textbf{Homework 6} \hfill \textbf{John Waczak} \\
\normalsize MTH 434 \hfill  Date: \today \\
Dr. Tevian Dray %\hfill worked w/ Ryan Tollefsen
\par\noindent\rule{\textwidth}{0.4pt} \\\\



\begin{enumerate}[leftmargin=0em, label=\textbf{\arabic*}]
\item \textbf{SPHERICAL COORDINATES II}
  Consider spherical coordinates $\{r, \theta, \phi\}$ and the adapted
  orthonormal basis
  \begin{equation}
    \left\{ \basis{1}, \basis{2}, \basis{3} \right\} = \left\{ \unit{r}, \unit{\theta}, \unit{\phi} \right\}
  \end{equation}
  The ``infinitesimal displacement vector'' $d\vec{\boldsymbol{r}}$ relates this
  basis to an orthonormal basis of 1-forms via
  \begin{equation}
    \vec{\boldsymbol{r}} = dr\;\unit{r} + rd\theta\;\unit{\theta} + r\sin\theta d\phi \;\unit{\phi}
  \end{equation}
  WARNING: \textit{these conventions imply $\tan\phi = \frac{y}{x}$}


  \begin{enumerate}[leftmargin=2em, label=(\textbf{\alph*})]
    \item Determine the exterior derivative of each basis vector (not 1-form)
      above, that is, compute $d\unit{r}, d\unit{\theta}$, and $d\unit{\phi}$. \\
      \begin{solution}
        We begin by recalling the definition of \textit{connection} 1-forms
        given by
        \begin{equation}
          d\unit{e}_j = \omega^i{}_j\; \unit{e}_i
        \end{equation}
        i.e. the 1-form coefficients of the expansion of $d\unit{e}_j$ in the
        regular basis. In class, we also defined the metric compatibility and
        torsion free requirements in terms of connections as
        \begin{align}
          \omega_{ij}+\omega_{ji} = 0 \\
          d\sigma^{i}+\omega^i{}_j\wedge \sigma^j = 0
        \end{align}
        where a connection with both indices downstairs is given by
        \begin{equation}
          \omega_{ij} = \unit{e}_i\cdot d\unit{e}_j
        \end{equation}
        Equipped with these definitions we now can solve for the connection
        1-forms. To evade a brute force calculation, recall the line element for
        $\E^3$ in spherical coordinates is given by
        \begin{equation}
          ds^2 = dr^2 + r^2d\theta^2+r^2\sin^2\theta d\phi^2
        \end{equation}
        so that
        \begin{equation}
          d\bvec{r} = dr\unit{r}+rd\theta\unit{\theta}+r\sin\theta d\phi\unit{\phi} 
        \end{equation}
        however, we also know that
        \begin{equation}
          d\bvec{r} = d(r\unit{r}) = dr \unit{r} + rd\unit{r}
        \end{equation}
        so that direct comparison gives
        \begin{equation}
          \boxed{ d\unit{r} = d\theta\;\unit{\theta}+\sin\theta d\phi\;\unit{\phi}}
        \end{equation}
        Knowing (10) significantly simplifies our job. For the other two
        derivatives we have
        \begin{align}
          d\unit{\theta} &= \omega^r{}_\theta\;\unit{r}+\omega^\theta{}_\theta \;\unit{\theta}+\omega^\phi{}_\theta \;\unit{\phi} \\
          d\unit{\phi} &= \omega^r{}_\phi\;\unit{r}+\omega^\theta{}_\phi\;\unit{\theta}+\omega^\phi{}_\phi \;\unit{\phi}\notag 
        \end{align}
        However, equation (4) further simplifies by allowing us to remove the
        diagonal terms. 
        \begin{align}
          d\unit{\theta} &= \omega^r{}_\theta\;\unit{r}+\omega^\phi{}_\theta \;\unit{\phi} \\
          d\unit{\phi} &= \omega^r{}_\phi\;\unit{r}+\omega^\theta{}_\phi\;\unit{\theta}\notag 
        \end{align}
        The vector version of (4) states that
        \begin{equation}
          d(\unit{e}_i\cdot\unit{e}_j) = \unit{e}_i\cdot d\unit{e}_j + d\unit{e}_i\cdot\unit{e}_j = 0
        \end{equation}
        Thus, equation (13) yields the following:
        \begin{align}
          \omega^r{}_\theta &= -\unit{\theta}\cdot d\unit{r} = -d\theta \\
          \omega^r{}_\phi &= -\unit{\phi}\cdot d\unit{r} = -\sin\theta d\phi
        \end{align}
        Now only two connections remain. However, we have that
        $\omega^\phi{}_\theta = -\omega^{\theta}{}_\phi$. It is therefore
        sufficient to solve for either one alone. To do this, consider the
        torsion free condition (eq 5). We have that
        \begin{align}
          0 &= d(rd\theta)+\omega^\theta{}_r\wedge dr + \omega^\theta{}_\theta\wedge rd\theta + \omega^\theta{}_\phi \wedge r\sin\theta d\phi \\
            &= dr\wedge d\theta +\omega^\theta{}_r\wedge dr + \omega^\theta{}_\phi\wedge r\sin\theta d\phi \\
            &= dr\wedge d\theta +d\theta\wedge dr +\omega^\theta{}_\phi \\
          \Rightarrow &\omega^\theta{}_\phi\wedge r\sin\theta d\phi = 0
        \end{align}
        Equation (19) tells us that $\omega^\theta{}_\phi$ must only include
        $d\phi$ and no other 1-forms in order for (19) to hold. Equation (5)
        also gives
        \begin{align}
          0 &= d(r\sin\theta d\phi) + \omega^\phi{}_r\wedge dr +\omega^\phi{}_\theta \wedge rd\theta + \omega^\phi{}_\phi\wedge r\sin\theta d\phi \\
            &= \sin\theta dr\wedge d\phi + r\cos\theta d\phi\wedge d\phi + \omega^\phi{}_r\wedge dr +\omega^\phi{}_\theta \wedge rd\theta \\
            &= \sin\theta dr \wedge d\phi + \sin\theta d\phi \wedge dr + r\cos\theta d\theta \wedge d\phi + \omega^{\phi}{}_\theta \wedge rd\theta \\
            &= r\cos\theta d\theta \wedge d\phi - rd\theta \wedge \omega^\phi{}_\theta\\
              \Rightarrow \omega^\phi{}_\theta &= \cos\theta d\phi \\
          \text{and } \omega^{\theta}{}_\phi &= -\cos\theta d\phi 
        \end{align}
        ...and that's all there is to it! In summary, we have
        \begin{empheq}[box=\fbox]{align}
          d\unit{r} &= d\theta\;\unit{\theta} +\sin\theta d\phi\;\unit{\phi}\\
          d\unit{\theta} &= -d\theta\;\unit{r} +\cos\theta d\phi\;\unit{\phi}\notag  \\
          d\unit{\phi} &= -\sin\theta d\phi\;\unit{r} -\cos\theta
          d\phi\;\unit{\theta} \notag
        \end{empheq}
      \end{solution}

    \item Compute $\omega_{ij}= \unit{e}_i\cdot d\unit{e}_j$ for $i,j=1,2,3$.
      \textit{What sort of beast should you get?}
      \begin{solution}
        This question asks us to identify the connection 1-forms. We can easily
        read these off from our solution to part $(a)$ by comparing with
        equation (11). They are
        \begin{empheq}[box=\fbox]{align}
          \omega_{rr} &= 0 &\omega_{\theta r}&=d\theta
          &\omega_{\phi r}&=\sin\theta d\phi \\
          \omega_{r \theta}&=-d\theta &\omega_{\theta \theta} &= 0
          &\omega_{\phi \theta} &= \cos\theta d\phi\ \notag\\
          \omega_{r\phi}&= -sin\theta d\phi & \omega_{\theta \phi}&= -\cos\theta
          d\phi & \omega_{\phi \phi} &= 0 \notag
        \end{empheq}
        
      \end{solution}
      
      
    \item Compute $\Omega_{ij}= d\omega_{ij}+\omega_{ik}\wedge\omega_{kj}$ for $i,j=1, 2, 3$
      (and where there is an implicit sum over k). \textit{What sort of beast
        should you get?} \\
      \begin{solution}
        Inspection of the equation for each $\Omega_{ij}$ reveals some
        interesting structure given our solution to part (b) of the problem.
        Notice that the table in equation 27 in antisymmetric as a result of
        equation 4. Therefore, we have that.
        
        \begin{equation}
          \omega_{ik}\wedge\omega_{ki} = -\omega_{ik}\wedge\omega_{ik}
        \end{equation}
        but each of our $\omega_{ik}$ are in $\bigwedge^1$ and therefore
        \begin{equation}
          \omega_{ik}\wedge\omega_{ik} = 0 \quad \forall\; i,k
        \end{equation}
        Thus, because $d(0)=0$ and because of (29) we have that $\Omega_{ii} =
        0$ $\forall$ $i$. If we zap equation (4) with d, we find that
        \begin{align}
          d\omega_{ij}+d\omega_{ji}= 0 \\
          \Rightarrow d\omega_{ji}=-d\omega_{ij}
        \end{align}
        For the second half of the $\Omega_{ij}$ equation, we have that
        \begin{align}
          \omega_{ik}\wedge \omega_{kj} &= -\omega_{kj}\wedge\omega_{ik} \\
                                        &= \omega_{jk}\wedge\omega_{ik} \\
                                        &= -\omega_{jk}\wedge\omega_{ki}
        \end{align}
        Putting (31) together with (34) gives
        \begin{equation}
          \Omega_{ij} = -\Omega_{ji}
        \end{equation}
        Therefore, it is sufficient to calculate elements with indices corresponding
        to the upper right half of the table in equation (27).  Let's begin with
        $i=\theta$ and $j = r$.
        \begin{align}
          d\omega_{\theta r} &= d(d\theta) = 0 \\
          \omega_{\theta k}\wedge \omega_{k r} &= \omega_{\theta r}\wedge \omega_{r r} + \omega_{\theta \theta} \wedge \omega_{\theta r} + \omega_{\theta \phi} \wedge \omega_{\phi r}  \\
                             &= 0 + 0 + \omega_{\theta \phi} \wedge \omega_{\phi r} \\
                             &= -\cos\theta d\phi \wedge \sin\theta d\phi \\
                             &= -\cos\theta\sin\theta d\phi \wedge d\phi \\
                             &= 0  \\ 
                              \Rightarrow  \Omega_{\theta r} &= 0 
        \end{align}
        For the next pair, we have 
        \begin{align}
          d\omega_{\phi r} &= d(\sin\theta d\phi) = \cos\theta d\theta \wedge d\phi \\
          \omega_{\phi k}\wedge \omega_{k r} &= \omega_{\phi r} \wedge \omega_{r r} + \omega_{\phi \theta} \wedge \omega_{\theta r} + \omega_{\phi \phi}\wedge \omega_{\phi r} \\
                           &= \omega_{\phi \theta} \wedge \omega_{\theta r} \\
                           &= \cos\theta d\phi \wedge d\theta  \\
                           &= -\cos\theta d\theta \wedge d\phi\\
          \Rightarrow \Omega_{\phi r} &= 0
        \end{align}
        finally,
        \begin{align}
          d\omega_{\phi \theta} &= d(cos\theta d\phi) = -\sin\theta d\theta \wedge d\phi \\
          \omega_{\phi k}\wedge\omega_{k\theta} &= \omega_{\phi r} \wedge \omega_{r\theta} + \omega_{phi \theta }\wedge\omega_{\theta \theta} + \wedge_{\phi \phi} \wedge \omega_{\theta \phi}  \\
                                &= \omega_{\phi r}\wedge \omega_{r\theta} \\
                                &= \sin\theta d\phi \wedge -d\theta \\
                                &= \sin\theta d\theta \wedge d\phi \\ 
          \Rightarrow \Omega_{\phi \theta} &= 0
        \end{align}
        Thus we conclude that every $\Omega_{ij} = 0$ for $\E^3$ described in
        spherical coordinates. The $\Omega_{ij}$ are related to curvature, and
        therefore, this makes sense as we are considering regular
        Euclidean space which is supposed to be flat.
        
        
        

        
        

        
      \end{solution}
  \end{enumerate}
\end{enumerate}


\end{document}

































