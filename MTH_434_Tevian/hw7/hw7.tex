\documentclass[a4paper, 11pt]{article}
\usepackage{geometry}
\geometry{letterpaper, margin=1in}
\usepackage{graphicx}
\graphicspath{ {images/} }

\usepackage{amsmath}
\usepackage{amssymb}  
\usepackage{amsthm}
\usepackage{ulem}

\usepackage{enumitem}


\usepackage{pdfpages} % for including full pdf pages

\usepackage{empheq}
% format to allow bolded theorems, corollaries, etc... 
\newtheorem*{theorem}{Theorem}
\newtheorem*{corollary}{Corollary}
\newtheorem*{lemma}{Lemma}
\newtheorem*{definition}{Definition}
\newtheorem*{Example}{Example} 
\newtheorem*{Remark}{Remark}

% stop typing \mathbb a thousand times 
\newcommand{\R}{\mathbb{R}}
\newcommand{\C}{\mathbb{C}}
\newcommand{\F}{\mathbb{F}}
\newcommand{\E}{\mathbb{E}}
\newcommand{\sphere}{\mathbb{S}}

% commands for bra-ket notation
\newcommand{\bra}[1]{\ensuremath{\left\langle#1\right|}}
\newcommand{\ket}[1]{\ensuremath{\left|#1\right\rangle}}
\newcommand{\bracket}[2]{\ensuremath{\left\langle #1 \middle| #2 \right\rangle}}
\newcommand{\matrixel}[3]{\ensuremath{\left\langle #1 \middle| #2 \middle| #3 \right\rangle}}
\newcommand{\expectation}[1]{\ensuremath{\left\langle #1 \right\rangle}}

% vector stuff
\newcommand{\basis}[1]{\hat{\mathbf{e}}_#1}
\newcommand{\unit}[1]{\hat{\boldsymbol{#1}}}
\newcommand{\bvec}[1]{\vec{\boldsymbol{#1}}}


% change margins for solution
\newenvironment{solution}{%
	\begin{list}{}{%
			\setlength{\topsep}{0pt}%
			\setlength{\leftmargin}{0.5cm}%
			\setlength{\rightmargin}{0.5cm}%
			\setlength{\listparindent}{\parindent}%
			\setlength{\itemindent}{\parindent}%
			\setlength{\parsep}{\parskip}%
		}%
		\item[]}{\end{list}}



\begin{document}
\noindent
\large\textbf{Homework 7} \hfill \textbf{John Waczak} \\
\normalsize MTH 434 \hfill  Date: \today \\
Dr. Tevian Dray %\hfill worked w/ Ryan Tollefsen
\par\noindent\rule{\textwidth}{0.4pt} \\\\



\begin{enumerate}[leftmargin=0em, label=\textbf{\arabic*}]
\item \textbf{SPHERICAL COORDINATES II}
  Consider the sphere of radius $r$, in spherical coordinates $(\theta, \phi)$,
  with line element
  \begin{equation}
    ds^2 = r^2(d\theta^2+\sin^2\theta d\phi^2)
  \end{equation}
  

  \begin{enumerate}[leftmargin=2em, label=(\textbf{\alph*})]
  \item Find the connection 1-forms $\omega_{ij}$ in this basis. \\ 
    \begin{solution}
      Because we are confined to the surface of the sphere, $d\bvec{r}$ vector
      will not aid in our calculation. However, The metric compatibility
      requirement
      \begin{equation}
        \omega_{ij}+\omega_{ji} = 0 
      \end{equation}
      Implies that every $\omega_{ii}$ is identically zero. Furthermore, because
      r is constant on the sphere, there are only 4 connection 1-forms to
      calculate. The above asymmetry argument implies we need only explicitly
      calculate 1 of them. To do this we will consider the torsion free
      requirement which states
      \begin{equation}
        d\sigma^i + \omega^i{}_j\wedge \sigma^j = 0 
      \end{equation}
      The leads to the following system of linear equations
      \begin{align}
        &\omega^\theta{}_\phi \wedge r\sin\theta d\phi = 0  \\
        &r\cos\theta d\theta \wedge d\phi + rd\theta\wedge\omega^\theta{}_\phi = 0
      \end{align}
      where in equation (4) we note that $dr = 0$ for the sphere. Recalling that
      $\omega^i{}_j$ are 1-forms and are expanded in the usual basis as
      \begin{equation}
        \omega^i{}_j = \Gamma^i{}_{jk}\;\sigma^k
      \end{equation}
      equation (4) implies that $\Gamma^\theta{}_{\phi\theta}=0$ so that
      $\omega^\theta{}_\phi$ has only a $d\phi$ component. With this, equation
      (5) becomes
      \begin{align}
        0 &= r\cos\theta d\theta \wedge d\phi + rd\theta \wedge \Gamma^{\theta}{}_{\phi\phi}d\phi \notag \\
          &= r\cos\theta d\theta \wedge d\phi + r\Gamma^{\theta}{}_{\phi\phi}d\theta\wedge d\phi \notag \\
        \Rightarrow \quad\Aboxed{&\Gamma^\theta{}_{\phi\phi} = -\cos\theta}
      \end{align}
      Where I have done the calculation in a coordinate basis for simplicity.
      Thus, we can conclude that the connection 1-forms for $\sphere^2$ (in the
      orthonormal basis) are
      \begin{empheq}[box=\fbox]{align}
            \omega_{\theta\theta} &=0 \hspace{12.5em}\omega_{\theta\phi}=-\cos\theta\;d\phi =-\frac{\cot\theta}{r}r\sin\theta \;d\phi
            \\
            \omega_{\phi\theta} &= \cos\theta\;d\phi =\frac{\cot\theta}{r}r\sin\theta \;d\phi \qquad \omega_{\phi\phi}=0
      \end{empheq} \\ 
    \end{solution}
    
  \item Compute $\Omega_{ij} = d\omega_{ij}+\omega_{ik}\wedge\omega_{kj}$ for
    $i, j = 1, 2$ (and where there is an implicit sum over k) \\ 

    \begin{solution}
      As in the last homework set, equation (2) combined with the fact that
      $d\alpha = 0$ for any $\alpha\in\bigwedge^1$ implies $\Omega_{ii} = 0$ for
      all $i$. We also have that
      \begin{equation}
        \Omega_{ji} = -d\omega_{ij}-\omega_{ik}\wedge\omega_{kj} = -\Omega_{ij}\\ 
      \end{equation}
      Therefore, we need only explicitly calculate one of the curvature 2-forms.
      For ease of calculation, I will begin in the coordinate basis and then
      convert back into the orthonormal basis to make the Gauss curvature
      obvious. 
      \begin{align}
        \Omega_{\theta\phi} &= d\omega_{\theta \phi} +\omega_{\theta\ k}\wedge\omega_{k\phi} \\
                            &= d\omega_{\theta\phi} +\omega_{\theta\theta}\wedge\omega_{\theta\phi}+\omega_{\theta\phi}\wedge\omega_{\phi\phi} \\
                            &= d\omega_{\theta\phi} \\
                            &= d(-\cos\theta\; d\phi) \\
                            &= -\sin\theta\; d\theta\wedge d\phi \\
                            &= -\frac{\sin\theta}{r^2\sin\theta}\; rd\theta\wedge r\sin\theta\; d\phi \\
                            &= -\frac{1}{r^2}\; rd\theta\wedge r\sin\theta\;d\phi \\
                            &= -\frac{1}{r^2}\omega
      \end{align}
      where $\omega$  is the orientation of $\sphere^2$.  \\

      In summary, we have found the following
      \begin{empheq}[box=\fbox]{align}
        \Omega_{\theta\theta} &= 0 \hspace{3em}\; \Omega_{\theta \phi} =
        -\frac{1}{r^2}\omega \\
        \Omega_{\phi\theta} &= \frac{1}{r^2}\omega \hspace{2em} \Omega_{\phi\phi}=0
      \end{empheq} \\

      From class, we know that the Gaussian curvature of a two dimensional
      surface is related to these curvature 2-forms by
      \begin{equation}
        \Omega^1{}_2= K\omega
      \end{equation}
      and $\frac{1}{r^2}$ is precisely the Gaussian curvature of $\sphere^2$ \\ 
    \end{solution}





    
  \item (Optional) Compare your answers (and your computations) with those from
    the previous homework assignment. \\
    
    \begin{solution}
      Looking at the previous homework, we see that the connection 1-forms are
      exactly the same for our computation in $\E^3$ despite the fact that the
      computation on $\sphere^2$ does not include any $dr$ components. It is
      interesting to note that the $dr$ components of the 1-forms in the first structure equation from
      $d\sigma^i$ exactly cancel out the $dr$ components of the 1-forms
      from $\omega^i{}_j\wedge\sigma^j$ leaving the $\omega_{ij}$ unchanged as
      we go from $\E^3$ to $\sphere^2$. \\

      The exact opposite happens for the curvature 2-forms for which the $dr$
      components of the 2-forms in the second structure equation exactly cancel
      in 3-dimensions to give zero for each $\Omega_{ij}$. On the sphere, $dr=0$
      which removes these canceling terms results in non-zero curvature 2-forms.
      \\

      This whole process illustrates how we can derive the curvature for
      surfaces in $\E^3$ by considering curvilinear coordinate systems and then
      setting a particular basis 1-form to 0. 
    \end{solution}
    
  \end{enumerate}
\end{enumerate}


\end{document}

































