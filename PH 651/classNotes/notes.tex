\documentclass[a4paper, 11pt]{article}
\usepackage{geometry}
\geometry{letterpaper, margin=1in}
\usepackage{amsmath}
\usepackage{amssymb}  
\usepackage{amsthm}
\usepackage{ulem} 
\usepackage{graphicx}
\usepackage{enumitem} % use for making lettered list 
\usepackage{bbm} % use for making the 1 identity operator EX: \mathbbm{1}
\usepackage{subfig} 
\graphicspath{ {images/} }

% format to allow bolded theorems, corollaries, etc... 
\newtheorem*{theorem}{Theorem}
\newtheorem*{corollary}{Corollary}
\newtheorem*{lemma}{Lemma}
\newtheorem*{definition}{Definition}
\newtheorem*{Example}{Example} 

% stop typing \mathbb a thousand times 
\newcommand{\R}{\mathbb{R}}
\newcommand{\C}{\mathbb{C}}


\begin{document}
%Header-Make sure you update this information!!!!
\noindent
\large\textbf{Class notes} \hfill \textbf{John Waczak} \\
\normalsize PH 651 \hfill  Last Updated: \today \\
	
	
\subsection*{Measurements} 
\par\noindent\rule{\textwidth}{0.4pt}
Given $A|\phi_n\rangle = a_n |\phi_n\rangle$ ; $\{|\phi_n\rangle\} $ \\ 
If our system is non-degenerate then our possible measurements are the eigenvalues $\{a_n\}$. Mathematically, we say that the way this works is via projection, i.e. 
	\begin{align}
		P_n &= |\phi_n\rangle\langle \phi_n| \\ 
		P_n |\phi_n\rangle &= \langle\phi_n|\psi\rangle |\phi_n \rangle 
	\end{align}
	
\begin{Example}[$L_x$ operator]
	\begin{align*}
		L_x \doteq \frac{1}{\sqrt{2}} \begin{pmatrix}
		0 & 1 & 0 \\ 
		1 & 0 & 1 \\ 
		0 & 1 & 0 
		\end{pmatrix} \\
		L_z \doteq \begin{pmatrix}
		1 & 0 & 0 \\ 
		0 & 0 & 0 \\ 
		0 & 0 & -1 
		\end{pmatrix}
	\end{align*}
	Our eigenvalues are given by the roots of the characteristic polynomial: 
	\begin{align*}
		det[L_x-\lambda \mathbf{I}] &= 0 \\ 
		-\lambda(\lambda^2-\frac{1}{2})-\frac{1}{\sqrt{2}}(-\frac{\lambda}{\sqrt{2}}) &= 0 \\
		\Rightarrow &\lambda \in \{0, \pm 1\}
	\end{align*}
	Then we put these back into the matrix to solve for the Eigenvectors. Some linear algebra we can get something like: 
		\begin{equation*}
			|L_x=0\rangle \doteq \frac{1}{\sqrt{2}}\begin{pmatrix}1 \\ 0 \\ -1\end{pmatrix}
		\end{equation*}
\end{Example}

\begin{Example}[$L_z^2$ operator]
	\begin{align*}
		L_z^2 &= \begin{pmatrix}1 & 0 & 0 \\ 0 & 0 & 0 \\ 0 & 0 & -1\end{pmatrix}\begin{pmatrix}1 & 0 & 0 \\ 0 & 0 & 0 \\ 0 & 0 & -1\end{pmatrix} \\ 
			&= \begin{pmatrix}1 & 0 & 0 \\ 0 & 0 & 0 \\ 0 & 0 & 1\end{pmatrix}
	\end{align*}
	Thus our characteristic equation is: 
		\begin{align*}
			-\lambda(1-\lambda)^2 &= 0 \\ 
			\rightarrow \lambda = 0 &\text{ and } \lambda = 1 \text { with 2 fold degeneracy} 
		\end{align*}
	Our $\lambda = 0$ eigenvector comes from: 
		\begin{equation*}
			\begin{pmatrix}1 & 0 & 0 \\ 0 & 0 & 0 \\ 0 & 0 & 1\end{pmatrix}\begin{pmatrix}x \\ y \\ z\end{pmatrix} = \begin{pmatrix} 0 \\ 0 \\ 0 \end{pmatrix}
		\end{equation*}
	Unfortunately the second solution leads to degeneracy... how do we solve it? 
\end{Example}



\subsection*{Degeneracy} 
\par\noindent\rule{\textwidth}{0.4pt}
Given some $a_n$ degenerate, we write: 
	\begin{equation*}
		A|\phi_n^i\rangle = a_n |\phi_n^i\rangle 
	\end{equation*}
And therefore we have: 
	\begin{equation}
		|\psi\rangle = \sum_n \sum\limits_{i=1}^{g_n}c_n^i|\phi_n^i\rangle; c_n^i = \langle\phi_n^i|\phi_n^i\rangle
	\end{equation}
	
Then our probabilities change as well! 
	\begin{align}
		P(a_n) &= \sum\limits_{i=1}^{g_n} |c_n^i|^2 \\ 
		P_n &= \sum\limits_{i=1}^{g_n} |\phi_n^i\rangle\langle \phi_n^i | 
	\end{align}
Previously, when we made a measurement, we knew what our final state would be with certainty. Now, our operator projects from a larger space to a smaller subspace. If $|\phi_i\rangle$ and $|\phi_j\rangle$ are degenerate, how do you know which you have projected onto? To figure this out, we need to look at our initial state. Say we have:
	\begin{equation*}
		|\psi\rangle = \frac{\alpha}{\sqrt{|\alpha|^2+|\beta|^2}}|\phi_n^1\rangle + \frac{\beta}{\sqrt{|\alpha|^2+|\beta|^2}}|\phi_n^2\rangle
	\end{equation*}

We can only determine the $\alpha, \beta$ given $|\psi\rangle$. i.e. 
	\begin{align*}
		\alpha = \langle \phi_n^1|\psi\rangle &\text{   } \beta = \langle\phi_n^1|\psi\rangle 
	\end{align*}

For the case of $L_z^2$ we had some degeneracy for the $\lambda =1$ eigenvalue. Using $\lambda=0$ gave 
	\begin{equation*}
		|L_z^2 = 0\rangle \doteq \begin{pmatrix}0 \\ 1 \\ 0 \end{pmatrix}
	\end{equation*}
For the degenerate case we have: 
	\begin{align*}
		\begin{pmatrix}
			0 & 0 & 0 \\ 
			0 & -1 & 0 \\ 
			0 & 0 & 0
		\end{pmatrix} \begin{pmatrix}c_1 \\ c_2 \\ c_3\end{pmatrix} = \mathbf{0}
	\end{align*}
We are constrained to have $c_2=0$ but we are free to make a choice for $c_1, c_3$. This allows us to pick nice choices for the subspace so long as we obey orhtonormality and completeness. 



\subsection*{Phases} 
\par\noindent\rule{\textwidth}{0.4pt}

Given $|\psi\rangle$ and $|\psi'\rangle = e^{i\phi}|\psi\rangle$ are these physically equivalent? We say \textbf{yes} because $e^{i\phi}$ is an \textbf{overall} phase and is not measurable. Whenever we \textit{measure} we will have $e^{i\phi}e^{-i\phi}$ which will disappear.  \\ 

Alternatively if we have a state $|\psi\rangle = \lambda_1 e^{i\phi_1}|\psi_1\rangle + \lambda_2 e^{i\phi_2}$ has a \textbf{relative} phase $e^{i(\theta_2-\theta_1)}$ which will not disappear when we perform measurements on the state. 
\end{document}

































