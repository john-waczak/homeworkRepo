\documentclass[a4paper, 11pt]{article}
\usepackage{geometry}
\geometry{letterpaper, margin=1in}
\usepackage{graphicx}
\graphicspath{ {images/} }

\usepackage{amsmath}
\usepackage{amssymb}  
\usepackage{amsthm}
\usepackage{ulem}

\usepackage{enumitem}


\usepackage{pdfpages} % for including full pdf pages

\usepackage{empheq}

\usepackage{listings}
\usepackage{hyperref}

%format to allow bolded theorems, corollaries, etc...
\newtheorem*{theorem}{Theorem}
\newtheorem*{corollary}{Corollary}
\newtheorem*{lemma}{Lemma}
\newtheorem*{definition}{Definition}
\newtheorem*{Example}{Example} 
\newtheorem*{Remark}{Remark}

% stop typing \mathbb a thousand times 
\newcommand{\R}{\mathbb{R}}
\newcommand{\C}{\mathbb{C}}
\newcommand{\F}{\mathbb{F}}
\newcommand{\E}{\mathbb{E}}
\newcommand{\M}{\mathbb{M}}
\newcommand{\sphere}{\mathbb{S}}

% commands for bra-ket notation
\newcommand{\bra}[1]{\ensuremath{\left\langle#1\right|}}
\newcommand{\ket}[1]{\ensuremath{\left|#1\right\rangle}}
\newcommand{\bracket}[2]{\ensuremath{\left\langle #1 \middle| #2 \right\rangle}}
\newcommand{\matrixel}[3]{\ensuremath{\left\langle #1 \middle| #2 \middle| #3 \right\rangle}}
\newcommand{\expectation}[1]{\ensuremath{\left\langle #1 \right\rangle}}

% vector stuff
\newcommand{\basis}[1]{\hat{\mathbf{e}}_#1}
\newcommand{\unit}[1]{\hat{\boldsymbol{#1}}}
\newcommand{\bvec}[1]{\vec{\boldsymbol{#1}}}
\newcommand{\threevec}[2]{\begin{pmatrix} #1 \\ #2 \end{pmatrix}}
\newcommand{\del}{\overrightarrow{\nabla}}

% change margins for solution
\newenvironment{solution}{%
	\begin{list}{}{%
			\setlength{\topsep}{0pt}%
			\setlength{\leftmargin}{0.5cm}%
			\setlength{\rightmargin}{0.5cm}%
			\setlength{\listparindent}{\parindent}%
			\setlength{\itemindent}{\parindent}%
			\setlength{\parsep}{\parskip}%
		}%
		\item[]}{\end{list}}




\begin{document}
\noindent
\large\textbf{Homework 6} \hfill \textbf{John Waczak} \\
\normalsize MTH 437 \hfill  Date: \today \\
Dr. Tevian Dray %\hfill worked w/ Ryan Tollefsen
\par\noindent\rule{\textwidth}{0.4pt} \\\\


\begin{enumerate}[leftmargin=0em, label=\textbf{\arabic*}.]
\item \textbf{INDEX GYMNASTICS}\\
  In a coordinate basis $\{dx^i\}$ of 1-forms, the components $g_{ij}$ of the
  metric are defined by $ds^2=g_{ij}dx^idx^j$. The dual basis $\{\bvec{e}_i\}$
  of vectors satisfies $\bvec{e}_i\cdot\bvec{e}_j = g_{ij}$. \\
  \noindent \textit{These bases are \textbf{not} necessarily orthogonal. It is
    however still true that $d\bvec{r} = dx^i\bvec{e}_i$}

  \begin{enumerate}[leftmargin=2em, label=(\textbf{\alph*})]
      \item Determine an expression for $\bvec{e}_i\cdot\del f$ in terms
        of partial derivatives.\\
        \begin{solution}
          Expanding the total derivative of f, we have
          \begin{align}
            df = \frac{\partial f}{\partial x^i}dx^i &= \del f\cdot d\bvec{r} \\
                                                     &= \del f \cdot dx^i\bvec{e}_i \\
                                                     &= \big(\bvec{e}_i\cdot\del f\big)dx^i 
          \end{align}
          \begin{equation}
            \Rightarrow \bvec{e}_i\cdot\del f = \frac{\partial f}{\partial x^i}
          \end{equation}
        \end{solution}
        
      \item Acting on 1-forms $F=\bvec{F}\cdot d\bvec{r}$, $G = \bvec{G}\cdot
        d\bvec{r}$, the metric satisfies $g(F,G)=\bvec{F}\cdot\bvec{G}$ for any
        vectors $\bvec{F}, \bvec{G}$. Express the components
        $g^{ij}=g(dx^i,dx^j)$ in terms of the components $g_{ij}$. \\
        \noindent\textit{A derivation in 2 dimensions is acceptable if you
          don't see how to handle the general case.}\\
        \begin{solution}
          In order to expand $g^{ij}=g(dx^i,dx^j)$, we need to find the fields
          related to $dx^i$ so that we can take advantage of the above fact. To
          do this, consider the following inner products
          \begin{align}
            \bvec{e}_i\cdot d\bvec{r} &= \bvec{e}_i\cdot dx^k\bvec{e}_k\\
                                      &= dx^kg_{ik} \\
            \bvec{e}_j\cdot d\bvec{r} &= \bvec{e}_j\cdot dx^\ell \bvec{e}_\ell \\
                                      & = dx^\ell g_{j\ell}
          \end{align}
          This suggest that we should look at the inner product on these two
          fields. \\
          \begin{align}
            g(dx^kg_{ik}, dx^\ell g_{j\ell}) &= \bvec{e}_i\cdot\bvec{e}_j = g_{ij} \\
            \Rightarrow g_{ik}\;g_{j\ell}\;g(dx^k,dx^\ell) &= g_{ij}  \\
            g_{ik}\;g_{j\ell}\;g^{k\ell} &= g_{ij} 
          \end{align}
          Now we note that this index equation may be reinterpreted as a matrix
          equation. If we take
          \begin{equation}
            \Big( g_{ij} \Big) \equiv G \qquad  \Big(g^{ij}\Big) \equiv \tilde{G} 
          \end{equation}
          Then equation 11 reinterpreted in terms of these matrices says
          \begin{equation}
            G\Big(G\tilde{G}\Big) = G
          \end{equation}
          Such an equation has a nontrivial solution only if
          \begin{equation}
            G\tilde{G} = \mathcal{I}
          \end{equation}
          where $\mathcal{I}$  is the identity matrix. This further implies that
          we must have $\tilde{G}=G^{-1}$. Therefore each $g^{ij}$ are the
          elements of the inverse matrix to $G$ corresponding to all of the
          $g_{ij}$. There is no simple way to write this out as an inverse
          matrix is the adjugate matrix divided by the determinant. We would therefore
          expect each $g^{ij}$ depends the determinant of many submatrices of
          $G$ resulting in 16 linear equations that we can, in principle, solve.
        \end{solution}
        
  \end{enumerate}

\item \textbf{DOUBLE-NULL COORDINATES}\\\\
  \noindent In 2-dimensional Minkowski space, let $u=t-x$, $v=t+x$.
  \begin{enumerate}[leftmargin=2em, label=(\textbf{\alph*})]
  \item Express the line element $ds^2 = -dt^2+dx^2$ in terms of the coordinate
    basis $\{du,dv\}$.
    \begin{solution}
      Taking the exterior derivative of the above, we find
      \begin{equation}
        du = dt-dx \qquad dv = dt+dx
      \end{equation}
      so that
      \begin{align}
        dt &= \frac{1}{2}(dv+du) \\
        dx &= \frac{1}{2}(dv-du)
      \end{align}
      Therefore, the re-expressed line element is
      \begin{align}
        ds^2 &= -dt^2+dx^2 \\
             &= -\left[ \frac{1}{2}(dv+du) \right]^2+\left[ \frac{1}{2}(dv-du) \right]^2\\
        &= -\frac{1}{2}dv\;du - \frac{1}{2}du\; dv
      \end{align}
      I chose to leave equation (20) as it is to make identifying elements of
      the metric easier. \\
    \end{solution}
  \item Determine the components $g^{ij}$ in this basis.\\
    \begin{solution}
      Recall from problem 1 that $ds^2=g_{ij}dx^idx^j$. Therefore, by
      inspection, we have that
      \begin{equation}
        \begin{cases}
          g_{uu} = g_{vv} = 0 \\
          g_{uv} = g_{vu} = -\frac{1}{2}
        \end{cases}
      \end{equation}
      so that as a matrix, the metric is
      \begin{equation}
        \Big(g_{ij}\Big) = \begin{pmatrix}0 & -\frac{1}{2} \\ -\frac{1}{2} & 0 \end{pmatrix}
      \end{equation}
      Now we wish to find all of the $g^{ij}$. Recall from problem 1 that
      $g^{ij}\equiv g(dx^i,dx^j)$. Therefore,
      \begin{align}
        g^{uu} &= g(du,du) = g(dt-dx, dt-dx) \\
               &= g(dt,dt) + g(dx,dx) \\
               &= -1+1 = 0 \\
        g^{uv} &= g(du,dv) = g(dt-dx,dt+dx) \\
               &= g(dt,dt)-g(dx,dx) \\
               &= -1-1 = -2\\
        g^{vu} &= g(dv, du) = g(dt+dx, dt-dx) \\
               &= g(dt,dt)-g(dx,dx) \\
               &= -1-1 = -2 \\
        g^{vv} &= g(dv,dv) = g(dt+dx, dt+dx) \\
               &= g(dt,dt)+g(dx,dx) \\
               &= -1+1 = 0
      \end{align}
      To summarize, the matrix corresponding to $g^{ij}$ is
      \begin{equation}
        \Big(g^{ij}\Big) = \begin{pmatrix}0 & -2 \\ -2 & 0 \end{pmatrix}
      \end{equation}
      which is in fact the inverse matrix to $\Big(g_{ij}\Big)$ as we predicted
      in problem 1.
    \end{solution}
    
  \item Compute $g_{ij}g^{jk}$. What sort of beast is it the components of?
    \begin{solution}
      There are four terms we must compute. Using our results from (b), they are
      \begin{align}
        g_{uu}g^{uu}+g_{uv}g^{vu} = 1 &= \delta_u{}^u  \\
        g_{uu}g^{vu}+g_{uv}g^{vv} = 0 &= \delta_u{}^v \\
        g_{vu}g^{uu}+g_{vv}g^{uv} = 0 &= \delta_v{}^u \\
        g_{vu}g^{uv}+g_{vv}g^{vv} = 1 &= \delta_v{}^v  
      \end{align}

      Apparently, $g_{ij}g^{jk}=\delta_i{}^k$ but we know this to be the case
      from problem 1 where we concluded that this kind of ``beast'' corresponds
      to an identity matrix. In other words,
      \begin{equation}
        \Big(g_{ij}g^{jk}\Big) = \begin{pmatrix}1&0\\0&1\end{pmatrix} 
      \end{equation}
      
    \end{solution}
    
  \end{enumerate}

\item \textbf{TRACES}\\\\
  \noindent Suppose that two vector-valued 1-forms
  $\bvec{G}=G^i{}_j\sigma^j\bvec{e}_i$ and $\bvec{R}=R^i{}_j\sigma^j\bvec{e}_i$
  have components that are related by
  \begin{equation}
    G^i{}_j = R^i{}_j - \frac{1}{2}\delta^i{}_jR, 
  \end{equation}
  where $R=R^i{}_i$. Find an expression for the trace $G=G^i{}_i$ of $\bvec{G}$
  in terms of R. \\
  \noindent\textit{$R$ is called the \textbf{trace} of $\bvec{R}$; more
    precisely, it is the trace of the matrix of components
    $\left(R^i{}_j\right)$. You may assume if desired that the underlying
    geometry is 4-dimensional, with signature 1.}\\
  \begin{solution}
    To find the trace, we simply let $j=i$. Doing this yields
    \begin{align}
      G^i{}_i &= R^i{}_i-\frac{1}{2}\delta^i{}_iR\\
              &= R^i{}_i-\frac{1}{2}\left(\sum_i 1\right) R \\
      &= R^{i}{}_i - \frac{N}{2}R = R\left(1-\frac{N}{2}\right)
    \end{align}
    where $N$ is the dimensionality of the space. For the particular case of
    4-dimensional spacetime with signature 1, we have
    \begin{equation}
      G^i{}_i = R(1-2) = -R 
    \end{equation}
    which a quick internet search confirms. Apparently, equation (45) is the
    reason why $\bvec{G}$, the Einstein tensor, is also known as the
    trace-reversed Ricci tensor.
  \end{solution}
  
\end{enumerate}

 
\end{document}






























