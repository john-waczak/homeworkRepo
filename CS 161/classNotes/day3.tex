% Created 2018-04-06 Fri 11:46
\documentclass[11pt]{article}
\usepackage[utf8]{inputenc}
\usepackage[T1]{fontenc}
\usepackage{fixltx2e}
\usepackage{graphicx}
\usepackage{longtable}
\usepackage{float}
\usepackage{wrapfig}
\usepackage{rotating}
\usepackage[normalem]{ulem}
\usepackage{amsmath}
\usepackage{textcomp}
\usepackage{marvosym}
\usepackage{wasysym}
\usepackage{amssymb}
\usepackage{hyperref}
\tolerance=1000
\author{John Waczak}
\date{4/6/2018}
\title{Day 3 Notes}
\hypersetup{
  pdfkeywords={},
  pdfsubject={},
  pdfcreator={Emacs 24.5.1 (Org mode 8.2.10)}}
\begin{document}

\maketitle

\section{Some tips:}
\label{sec-1}
It is common that the first program everyone learns is to print
"Hello world!" to the screen.

\begin{verbatim}
#include <iostream>
using namespace std;

int main()
{
  cout << "Hello world!" << endl;
  return 0;
}
\end{verbatim}

Be carefull with the "using namespace std;" line. This imports many
functions which can cause conflicts with user defined functions in
large programs. Typically it is better to use smaller statements
such as:

\begin{verbatim}
#include <iostream>

using std::cout;
using std::cin;
using std::endl;

int main()
{
  cout << "Hello world!" << endl;
  return 0;
}
\end{verbatim}

\section{Data types}
\label{sec-2}
There are multiple data types to consier. They all have different
ways to be declared.

\begin{verbatim}
#include <iostream

using std::cout;
using std::cin;
using std::string;
using std::endl;

int main()
{
  int myInt = 7;
  char myChar = '7';
  string myString = "7";
  bool myBool = true;

  cout << myInt << endl;
  cout << myChar << endl;
  cout << myString << endl;
  cout << myBool << endl;

  return 0;
}
\end{verbatim}
% Emacs 24.5.1 (Org mode 8.2.10)
\end{document}
