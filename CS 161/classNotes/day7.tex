% Created 2018-04-16 Mon 11:56
\documentclass[11pt]{article}
\usepackage[utf8]{inputenc}
\usepackage[T1]{fontenc}
\usepackage{fixltx2e}
\usepackage{graphicx}
\usepackage{longtable}
\usepackage{float}
\usepackage{wrapfig}
\usepackage{rotating}
\usepackage[normalem]{ulem}
\usepackage{amsmath}
\usepackage{textcomp}
\usepackage{marvosym}
\usepackage{wasysym}
\usepackage{amssymb}
\usepackage{hyperref}
\tolerance=1000
\author{John Waczak}
\date{4/16/2018}
\title{CS 161 -- Day 7 Notes}
\hypersetup{
  pdfkeywords={},
  pdfsubject={},
  pdfcreator={Emacs 24.5.1 (Org mode 8.2.10)}}
\begin{document}

\maketitle


\section{Named constants}
\label{sec-1}
Most of the time people just call them constants. They are different
from variables (named chunks of memory we can control) and literals
(literal values). A constant is a reserved, named piece of memeroy
but the value never changes. When you declare you have to say what
the value is. \\

Constants are easy to change (think for example sales tax that might
change year to year) and much better for catching typographical
errors than if you manually type out the same thing over and over
again. \\

\begin{verbatim}
#include <iostream>
#include <string>

using std::cin;
using std::cout;
using std::endl;
using std::string;

int main()
{
  const double SALES_TAX_RATE = 7.25;
  cout << SALES_TAX_RATE << endl;
  return 0;
}
\end{verbatim}

\begin{verbatim}
7.25
\end{verbatim}

Be careful about declaring global variables (variables outside of
functions and the main loop). This has to do with scope. Global
variables are BAD however global constants are acceptible and make \\
  sense.

\section{Decision makeing (If-statements)}
\label{sec-2}
An If-statement allows us to evaluate boolean conditions\ldots{} \\
\begin{verbatim}
#include <iostream>
#include <string>

using std::cin;
using std::cout;
using std::endl;
using std::string;

int main()
{
  const double SALES_TAX_RATE = 7.25;
  if (SALES_TAX_RATE > 5.0)
  {
    cout << "true\n";
  }
  else
  {
    cout << "false\n";
  }
  return 0;
}
\end{verbatim}

\begin{verbatim}
true
\end{verbatim}

You can do without the curly braces if you only have one line in
your statement. You can also nest your if statements to make a
decision tree.
\begin{verbatim}
#include <iostream>
#include <string>

using std::cin;
using std::cout;
using std::endl;
using std::string;

int main()
{
  int num = 12;

  if (num > 10)
    cout << " num > 10\n";
  else if (num == 9)
    cout << " num is exactly 9\n";
  else
    cout << " num < 9";



  return 0;
}
\end{verbatim}

\begin{verbatim}
num > 10
\end{verbatim}

\section{Loops}
\label{sec-3}
The simplest type of loop is probably the while loop. It uses a
conditional expression which, when true, continues to loop throug
the curly braces. \\

\begin{verbatim}
#include <iostream>
#include <string>

using std::cin;
using std::cout;
using std::endl;
using std::string;

int main()
{
  int num = 0;

  while (num <= 10)
    {
      cout << num << endl;
      num ++ ;
    }
  return 0;
}
\end{verbatim}

\begin{center}
\begin{tabular}{r}
0\\
1\\
2\\
3\\
4\\
5\\
6\\
7\\
8\\
9\\
10\\
\end{tabular}
\end{center}
So we see that the everything starts with the condition. If that
condition argument evaluates to true, we then we enter the
loop. Once finished with the while-loop's code block, the condition
is reevaluated. This continues until the argument evaluates to
false. \\

If you accidentally run an infinite loop (i.e. it gets stuck on
always true), type Ctrl-c or Ctrl-d. One of those should force the
program to quit.
% Emacs 24.5.1 (Org mode 8.2.10)
\end{document}
