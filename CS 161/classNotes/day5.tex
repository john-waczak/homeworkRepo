% Created 2018-04-11 Wed 11:23
\documentclass[11pt]{article}
\usepackage[utf8]{inputenc}
\usepackage[T1]{fontenc}
\usepackage{fixltx2e}
\usepackage{graphicx}
\usepackage{longtable}
\usepackage{float}
\usepackage{wrapfig}
\usepackage{rotating}
\usepackage[normalem]{ulem}
\usepackage{amsmath}
\usepackage{textcomp}
\usepackage{marvosym}
\usepackage{wasysym}
\usepackage{amssymb}
\usepackage{hyperref}
\tolerance=1000
\author{John Waczak}
\date{4/11/2018}
\title{CS 161 Day 5 Notes}
\hypersetup{
  pdfkeywords={},
  pdfsubject={},
  pdfcreator={Emacs 24.5.1 (Org mode 8.2.10)}}
\begin{document}

\maketitle

\section*{Some usefull math tricks}
\label{sec-1}
\begin{verbatim}
#include <iostream>
#include <string>
#include <cmath>

using std::cout;
using std::endl;
using std::string;
using std::pow;

int main()
{
  int num = 6;
  num++;
  cout <<"num++ does this: "<< num << endl;
  num--;
  cout <<"num-- does this: " << num << endl;
  cout <<"The pow command for 2^3 does this: "<< pow(2,3) << endl;
  return 0;
}
\end{verbatim}

\begin{center}
\begin{tabular}{lllrllll}
num++ & does & this: & 7 &  &  &  & \\
num-- & does & this: & 6 &  &  &  & \\
The & pow & command & for & 2$^{\text{3}}$ & does & this: & 8\\
\end{tabular}
\end{center}
% Emacs 24.5.1 (Org mode 8.2.10)
\end{document}
