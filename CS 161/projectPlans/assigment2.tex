\documentclass[a4paper, 11pt]{article}
\usepackage{geometry}
\geometry{letterpaper, margin=1in}
\usepackage{amsmath}
\usepackage{amssymb}  
\usepackage{amsthm}
\usepackage{ulem} 
\usepackage{graphicx}
\usepackage{listings}
\graphicspath{ {images/} }

\begin{document}
%Header-Make sure you update this information!!!!
\noindent
\large\textbf{Assigment 2: Project Plan} \hfill \textbf{John Waczak} \\
\normalsize CS 161 \hfill  Date: \today \\
Tim Alcon \\

	
\subsection*{2.a Min-Max}
	\textbf{Pseudocode:} 
	\begin{lstlisting}
		declare "ints" string for storing integers 
		declare "max", "min" and "numInts" int-variables
		
		prompt user to enter a number of intergers
		capture user input and intialize to "numInts" variable 
		
		Initialize "max" and "min" to zero.
		loop over integers from 1 to "numInts" 
			declare "integer" int 
			prompt user to enter number
			initialize "integer" to value user enters
			
			if "integer" is greater than "max"			
				set "max" to "integer"
			else if "integer" is less than "min"
				set "min" to "integer" 
			else 
				do nothing 
		
		once loop is finished, display max and min		  	
	\end{lstlisting}
	
	\textbf{Test Plan}
		\begin{center}
			\begin{tabular}{|c|c|}
				\hline
				\textbf{test} & \textbf{expected output} \\ 
				\hline 
				user enters all 1's (or integers of same value) & output max = 1, min = 1\\
				\hline 
				user enters 1,2,3,4,5 & output max = 5, min = 1 \\ 
				\hline 
				user enters 100, 101, 200, 1000 & output max = 1000, min = 100 \\ 
				\hline 
				user enters 10, 5, 0, -5, -10 & output max = 10, min = -10 \\ 
				\hline 
				user only enters 1 integer, say 10 & output max = 10, min = 10 \\ 
				\hline 
			\end{tabular}
		\end{center}
	
\subsection*{2.b File Adder} 
	\textbf{Pseudocode} 
	\begin{lstlisting}
		declare "fileName" string variable 
		prompt user to enter a file name
		intialize "fileName" to user's value 
		check to see if "fileName" is actually a file 
			if it isn't 
				print out "Not a file" 
			else, if it is a file 
				print out "Found file" 
		Open file 
		declare "numLines" int variable 
		declare "sum" variable and intialize to 0
		set the "numLines" to the number of lines in file 
		
		if the file is empty 
			print "empty file" and quit program 
		otherwise, 
			continue 
			
		for each line from 0 to "numLines" 
			read in line and cast to int variable "lineNum" 
			add "lineNum" to "sum" variable 
		
		Close the file 
		Create a file called "sum.txt" and open it 
		write the value "sum" to the file 
		close "sum.txt" 
	\end{lstlisting}

	\textbf{Test plan} 
	\begin{center}
		\begin{tabular}{|c|c|}
			\hline
			\textbf{test (input)} & \textbf{expected output} \\ 
			\hline 
			empty file & create empty sum file \\ 
			\hline 
			file with single integer 1& write 1 to sum.txt \\ 
			\hline 
			file with 1,2,3,4,5 & write 15 to sum.txt \\ 
			\hline 
			file with 10, 100, 1000 &  write 1110 to sum.txt \\ 
			\hline 
			file with 10, 5, 0, -5, -20 & write -10 to sum.txt \\ 
			\hline 
		\end{tabular}
	\end{center}

\subsection*{2.c Num Guess} 
	\textbf{Pseudocode} 
	\begin{lstlisting}
		declare "userNum", "playerGuess", and "guessCount" ints 
		prompt user to enter an integer for player to guess 
		initialize "userNum" to this value 
		
		declare boolean "isCorrect" and initialize to false
		initialize "guessCount" to 0
		
		begin loop (probably a do-while)
			prompt player to input a guess 
			initialize "playerGuess" to player's value 
			
			if the guess is too low, print out "Too low--try again"
			if the guess is too high, print out "Too high--try again" 
			otherwise, if the guess is correct
				 change "isCorrect" to true
				 
			increment "guessCount" 
			check if "isCorrect" is true. If it is, exit the loop
		print out "you guessed it in guessCount tries" 		
	\end{lstlisting}
	
	\textbf{Test Plan} 
	\begin{center}
		\begin{tabular}{|c|c|}
			\hline
			\textbf{test (input)} & \textbf{expected output} \\ 
			\hline 
			val: 1, guess: 1 & 1 try \\ 
			\hline 
			val: 1, guess: 10, -10, 1 & too high, too low, 3 tries \\
			\hline 
			val: -10, guess: 100, 10, 0, -1, -10 & too high, too high ..., 5 tries \\ 
			\hline 
			val: 1000, guess: 1, 2, 100, 10000, 1000 & too low, too low, too low, too high, 5 tries \\ 
			\hline 
		\end{tabular}
	\end{center}
	




















\end{document}







































