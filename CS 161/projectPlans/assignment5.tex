\documentclass[a4paper, 11pt]{article}
\usepackage{geometry}
\geometry{letterpaper, margin=0.5in}
\usepackage{amsmath}
\usepackage{amssymb}  
\usepackage{amsthm}
\usepackage{ulem} 
\usepackage{graphicx}
\usepackage{listings}
\graphicspath{ {images/} }

 \lstset{breaklines=true}



\begin{document}
%Header-Make sure you update this information!!!!
\noindent
\large\textbf{Assigment 3: Project Plan} \hfill \textbf{John Waczak} \\
\normalsize CS 161 \hfill  Date: \today \\
Tim Alcon \\


\subsection*{5a}
	\paragraph{}
		\textit{Testing Plan}
		
		\begin{center}
			\begin{tabular}{|c|c|}
				\hline
				\textbf{test} & \textbf{expected output} \\ 
				\hline 
				input: 2, -19, 4 & output -19, 2, 4 \\ 
				\hline
				input: valgrind ./transformArray & output No memory leaks \\ 
				\hline
				input: 1, 1, 1 & output 1, 1, 1 \\ 
				\hline
			\end{tabular}
		\end{center}
	\paragraph{}
		\textit{Pseudo-code}
		\begin{lstlisting}	
			define void function called transformArray that takes
			two arguments a reference to a pointer to an array of 
			integers and an integer for the size of the array 
			
			decalare a new dynamic array twice the size of the old 
			
			loop through the indices of the new array 
				if the index is less than the size 
					add the old array element to the new one 
				when the index is past the old size 
					add the old value at index-size to array 
					plus 1
			
			delete the old array to free the memory 
			set the old array equal to the new array
				
		\end{lstlisting} 

\subsection*{5b} 
	\paragraph{}
		\textit{Testing Plan} 
			\begin{center}
				\begin{tabular}{|c|c|}
					\hline
					\textbf{test} & \textbf{expected output} \\ 
					\hline 
					input: place ship at (0,9) R  & output: invalid move \\ 
					\hline
					input: place ship at (9, 0) C & output: invalid move  \\ 
					\hline
					input: place ship at (0, 9) C & output: valid move \\ 
					\hline
					input: place ship at (9,0) R & output: valid move \\ 
					\hline 
					input: attack (anything within bounds) & output true \\ 
					\hline
				\end{tabular}
			\end{center}


	\paragraph{}
		\textit{Psuedo-code} 
			\begin{lstlisting}
			create ship class with the following attributes
				string name 
				int length 
				int damage 
			with the following methods 
				constructor that sets name, length, and puts the damage to 0 
				
				get methods for all of the attributes 
				
				take hit method that increments damage by one
				
			create a board class with the following attributes
				10x10 bool array for keeping track of what has been attacked 
					
				10x10 ship pointer array that stores address of ship occupying each position on board
					
				int shipsUnsunk 
					
			with the follwing methods 
				constructor that initializes bools to false, pointers to NULL
				
				getAttacksArrayElement that returns element in bool array for a given (i,j) pair of int coordinates
					
				getShipsArrayElement that returns pointer to ship given a specific (i,j) pair of int coordinates. Should return NULL if nothing occupies that position 
					
				getNumShipsRemaining that returns shipsUnsunk 
					
				placeShip -- takes a ship object, (i,j) corresponding to position closest to (0,0) top left corner of board, and either R for row or C for column to specify orientation of the board. first make sure (i,j) are valid positions on 10x10 board second we want to check if the ship will fit with given choice of orientation and position. Notice that it is sufficient to check whether or not the opposite side of the ship is on the board to determine if the move is legal as the ship either fits or it doesn't. 
					if user picks R for row check to see if (i + length of ship) is valid position on board. 
							
					if user picks C for column check if (j + length of ship) is  a valid position on board. 
						
					if that all checks out then define a pointer that points to the ship. Loop from 0 to length of ship-1 and add poniter for corresponding position to the pointer array then add one to shipsUnsunk and return true 
					return false if any of the previous checks fail 
					
				attack -- takes an (i, j) coordinate pair. Check to see if that position has been hit before in bool array. If it has, that's an  invalid move so don't do anything, just return true. Otherwise, turn that position in bool array to true.
					
					When the bool is set to true, get the call the takeHit method  for the corresponding ship using the pointer in the pointer array. 
					
					If that ship's damage is equal to it's length print out "you sank (ship name)!". Decrease shipsUnsunk by 1. 
						
				allShipsSunk -- loop through 10x10 pointer array. If the pointer is not null and that ship's length is equal to it's damage continue. If this fails ever return false. If we make it through the whole array then return true (the game is over).  
							
			\end{lstlisting}
			
			General question: Since we can't have a main, can we add attributes to these classes if we want to? For example I am thinking I want to add something like an isSunken boolean to the ship class to keep track of whether or not the ship has been sunk before. 

\end{document}





































