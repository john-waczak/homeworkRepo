\documentclass[a4paper, 11pt]{article}
\usepackage{geometry}
\geometry{letterpaper, margin=1in}
\usepackage{amsmath}
\usepackage{amssymb}  
\usepackage{amsthm}
\usepackage{ulem} 
\usepackage{graphicx}
\usepackage{physics}
\graphicspath{ {images/} }

\begin{document}
%Header-Make sure you update this information!!!!
\noindent
\large\textbf{Assignment 3 Reflection} \hfill \textbf{John Waczak} \\
\normalsize CS 161 \hfill  Date: \today \\

\subsection*{Understanding}
	\paragraph{}
	I felt that the first problem was pretty straight forward and I didn't have trouble implementing a solution for using the pass-by-reference function to order three integers. The dice-game however was much more difficult and I found that the process of writing a testing plan and then pseudo-code and then actual code helped me learn how the game was actually supposed to work. For example I was confused at first and thought that we 
	
\subsection*{Testing Plan}
	\paragraph{}
	My testing plan was overall pretty good. I found that I had difficulty figuring out a comprehensive testing plan  for the dice game as I couldn't predict which dice would be rolled without hard-coding the random number seed but I think my tests gave me confidence that my solutions worked. 
\subsection*{Design}
	\paragraph{}
	As I mentioned the only thing I had to update in my design was reorganizing the dice-game logic as I slowly figured out how the rules worked. I was definitely confused about how the holding should work and also the difference between the turn total and the player score. 
 
\subsection*{Improvement} 
	 To improve my process for working on this problem I would try and come up with some better test cases. The ones I envisioned didn't involve fixing the random number seed which in hindsight I realize is a great way to test the game. Knowing that the sequence of random numbers is fixed would help me come up with some decision trees (hold/no hold) that I could try out to make sure that the program works. 
		
		
\end{document}