\documentclass[a4paper, 11pt]{article}
\usepackage{geometry}
\geometry{letterpaper, margin=1in}
\usepackage{graphicx}
\graphicspath{ {images/} }

\usepackage{amsmath}
\usepackage{amssymb}  
\usepackage{amsthm}
\usepackage{ulem}
\usepackage{mathtools}
\usepackage{enumitem}


\usepackage{pdfpages} % for including full pdf pages

% format to allow bolded theorems, corollaries, etc... 
\newtheorem*{theorem}{Theorem}
\newtheorem*{corollary}{Corollary}
\newtheorem*{lemma}{Lemma}
\newtheorem*{definition}{Definition}
\newtheorem*{Example}{Example} 
\newtheorem*{Remark}{Remark}

% stop typing \mathbb a thousand times 
\newcommand{\R}{\mathbb{R}}
\newcommand{\C}{\mathbb{C}}
\newcommand{\F}{\mathbb{F}}

% commands for bra-ket notation
\newcommand{\bra}[1]{\ensuremath{\left\langle#1\right|}}
\newcommand{\ket}[1]{\ensuremath{\left|#1\right\rangle}}
\newcommand{\bracket}[2]{\ensuremath{\left\langle #1 \middle| #2 \right\rangle}}
\newcommand{\matrixel}[3]{\ensuremath{\left\langle #1 \middle| #2 \middle| #3 \right\rangle}}
\newcommand{\expectation}[1]{\ensuremath{\left\langle #1 \right\rangle}}

% change margins for solution
\newenvironment{solution}{%
	\begin{list}{}{%
			\setlength{\topsep}{0pt}%
			\setlength{\leftmargin}{0.5cm}%
			\setlength{\rightmargin}{0.5cm}%
			\setlength{\listparindent}{\parindent}%
			\setlength{\itemindent}{\parindent}%
			\setlength{\parsep}{\parskip}%
		}%
		\item[]}{\end{list}}



\begin{document}
\noindent
\large\textbf{Homework 7} \hfill \textbf{John Waczak} \\
\normalsize PH 652 \hfill  Date: \today \\
Dr. Oksana Ostroverkhova
\par\noindent\rule{\textwidth}{0.4pt} \\\\



\begin{enumerate}[leftmargin=0em]
  \item Calculate the following Clebsch-Gordon coefficients
  
    \begin{enumerate}[leftmargin=3em, label=(\alph*)]
    \item $\bracket{1,1;-1,1}{2,0}$
    \item $\bracket{1,1;1,-1}{2,0}$
    \item $\bracket{1,1;0,0}{2,0}$
    \item $\bracket{1,1;-1,1}{1,0}$
    \item $\bracket{1,1;1,-1}{1,0}$
    \item $\bracket{1,1;-1,1}{0,0}$
    \item $\bracket{1,1;0,0}{0,0}$
    \end{enumerate}
    
    \begin{solution}
      I will solve each of these parts by filling out the Clebsch-Gordon table.
      Observe that from the bra in each of the above expressions, we can see
      that we are expanding for states with $j_1=j_2=1$. I will use the notation
      $\ket{j,m}$ for the coupled basis states and $\ket{j_1,j_2;m_1,m_2}$ for
      the uncoupled states. First, recall the following definitions from class
      \begin{align}
        J_-\ket{j,m} = \hbar\sqrt{j(j+1)-m(m-1)}\ket{j, m-1}\\
        J_- = J_{1-}\otimes I_2 + J_{2-}\otimes I_1 \equiv J_{1-}+J_{2-}
      \end{align}
      Using these definitions, we can then begin with the stretched state and
      write
      \begin{equation}
        \ket{2,2} = \ket{1,1;1,1}
      \end{equation}
      \begin{align}
        J_-\ket{2,2} &= (J_{1-}+J_{2-})\ket{1,1;1,1} \\
        \hbar2\ket{2,1} &= J_{1-}\ket{1,1;1,1}+J_{2-}\ket{1,1;1,1}\\
        2\ket{2,1} &= \sqrt{2-1(1-1)}\ket{1,1;0,1}+\sqrt{2-1(1-1)}\ket{1,1;1,0} \\
        \Rightarrow \ket{2,1} &= \frac{\sqrt{2}}{2}\left( \ket{1,1;0,1}+\ket{1,1;1,0} \right)
      \end{align}
      Applying the lowering operator once more gives
      \begin{align}
          \hbar\sqrt{6}\ket{2,0} &= \frac{\sqrt{2}}{2} \Big(J_{1-}\ket{1,1;0,1}+J_{2-}\ket{1,1;0,1}\\
          &\qquad +J_{1-}\ket{1,1;1,0}+J_{2-}\ket{1,1;1,0} \Big) \notag\\
                                 &= \hbar\frac{\sqrt{2}}{2}\Big(\sqrt{2}\ket{1,1;-1,1} +\sqrt{2}\ket{1,1;0,0} \\
                                 &\qquad + \sqrt{2}\ket{1,1;0,0}+\sqrt{2}\ket{1,1;1,-1}\Big) \notag\\
        \Rightarrow \Aboxed{\ket{2,0} &= \frac{1}{\sqrt{6}}\ket{1,1;-1,1}+\frac{2}{\sqrt{6}}\ket{1,1;0,0}+\frac{1}{\sqrt{6}}\ket{1,1;1,-1}   }
      \end{align}
      We can see from this that $(a), (b),$ and $(c)$ are given by the
      coefficients in (10). To continue to the next parts, we need to continue
      applying the ladder operator.
      \begin{align}
        \hbar\sqrt{6}\ket{2,-1} &= \frac{1}{\sqrt{6}}\Big(J_{1-}\ket{1,1;-1,1} + J_{2-}\ket{1,1;-1,1}\\
                                &\qquad\quad + 2J_{1-}\ket{1,1;0,0}+2J_{2-}\ket{1,1;0,0} \notag\\
                                &\qquad\quad + J_{1-}\ket{1,1;1,-1}+J_{2-}\ket{1,1;1,-1}\Big) \notag\\
        \Rightarrow \sqrt{6}\ket{2,-1} &= \frac{1}{\sqrt{6}}\Big(0 + \sqrt{2}\ket{1,1;-1,0} \\
                                &\qquad\quad + 2\sqrt{2}\ket{1,1;-1,0} + 2\sqrt{2}\ket{1,1;0,-1}\notag\\
                                &\qquad\quad + \sqrt{2}\ket{1,1;0,-1} + 0 \Big) \notag\\
        \Rightarrow \ket{2,-1} &= \frac{\sqrt{2}}{2}\Big(\ket{1,1;-1,0}+\ket{1,1;0,-1}\Big)
      \end{align}
      \begin{align}
        \hbar2\ket{2,-2} &= \frac{\sqrt{2}}{2}\Big(J_{1-}\ket{1,1;-1,0}+J_{2-}\ket{1,1;-1,0}\\
                     &\qquad\quad + J_{1-}\ket{1,1;0,-1}+J_{2-}\ket{1,1;0,-1}\Big) \notag \\
        \Rightarrow \ket{2,-2} &= \frac{\sqrt{2}}{4}\Big(0+\sqrt{2}\ket{1,1;-1,-1}+\sqrt{2}\ket{1,1;-1,-1}+0   \Big)
      \end{align}
      \begin{equation}
        \boxed{\ket{2,-2}= \ket{1,1;-1,-1}}
      \end{equation}
      There are two ways to make $m=1$ if $j=1$: either $j_1=0, j_2=1$, or $j_1=1, j_2=0$.
      Therefore we can take advantage of orthogonality to find the
      next state.
      \begin{align}
        0 &= \bracket{2,1}{1,1} \\
          &= \frac{\sqrt{2}}{2}\Big(\bra{1,1;0,1}+\bra{1,1;1,0}\Big)\Big(a\ket{1,1;0,1}+b\ket{1,1;1,0}\Big)\\
          &= \frac{\sqrt{2}}{2}(a+b)\\
        \Rightarrow a&=-b \\
      \end{align}
      \begin{equation}
        \Rightarrow \boxed{\ket{1,1}= \frac{\sqrt{2}}{2}\Big(\ket{1,1;1,0}-\ket{1,1;0,1}\Big)}
      \end{equation}
      Applying the lowering operator gives
      \begin{align}
        \hbar\sqrt{2}\ket{1,0}&= \frac{\sqrt{2}}{2}\Big( J_{1-}\ket{1,1;1,0}+J_{2-}\ket{1,1;1,0}\\
                              &\qquad\quad - J_{1-}\ket{1,1;0,1} - J_{2-}\ket{1,1;0,1}\Big)\notag\\
        \sqrt{2}\ket{1,0}&= \frac{\sqrt{2}}{2}\Big( \sqrt{2}\ket{1,1;0,0}+\sqrt{2}\ket{1,1;1,-1} \\
                              &\qquad\quad -\sqrt{2}\ket{1,1;-1,1}-\sqrt{2}\ket{1,1;0,0}\Big)\notag \\
        \Rightarrow \Aboxed{\ket{1,0}&= \frac{\sqrt{2}}{2}\Big(\ket{1,1;1,-1}-\ket{1,1;-1,1}\Big)  }
      \end{align}
      \begin{align}
        \hbar\sqrt{2}\ket{1,-1} &= \frac{\sqrt{2}}{2}\Big(J_{1-}\ket{1,1;1,-1}+J_{2-}\ket{1,1;1,-1}\\
                                &\qquad\quad -J_{1-}\ket{1,1;-1,1}-J_{2-}\ket{1,1;-1,1}\Big)\notag\\
        \sqrt{2}\ket{1,-1} &= \frac{\sqrt{2}}{2}\Big(0+\sqrt{2}\ket{1,1;0,-1} - 0 -\sqrt{2}\ket{1,1;-1,0}  \Big)\\
        \Rightarrow \Aboxed{\ket{1,-1} &= \frac{\sqrt{2}}{2}\Big(\ket{1,1;0,-1}-\ket{1,1;-1,0}\Big)}
      \end{align}
      Finally, we can observe that the only way to make $j=0, m=0$ is by
      some combination of $m_1=1$, $m_2=-1$, or $m_1=-1$, $m_2=1$, or
      $m_1=m_2=0$. Therefore we can expand the final state by using the
      following orthogonality condition
      \begin{align}
        0 &= \bracket{2,0}{0,0} \\
          &= \frac{1}{\sqrt{6}}\Big(\bra{1,1;1,-1}+2\bra{1,1;0,0}+\bra{1,1;-1,1}\Big)\cdot \\
          &\qquad\qquad\quad \Big(a\ket{1,1;1,-1}+b\ket{1,1;0,0}+c\ket{1,1;1,-1}\Big)\notag \\
        \Rightarrow 0 &= a+2b+c \\
        \Rightarrow \Aboxed{\ket{0,0}&= \frac{1}{\sqrt{3}}\ket{1,1;1,-1}-\frac{1}{\sqrt{3}}\ket{1,1;0,0}+\frac{1}{\sqrt{3}}\ket{1,1;-1,1}}
      \end{align}
     From these expansions, we can see that the solutions to the problem is as
     follows:
     \begin{enumerate}[leftmargin=3em, label=(\alph*)]
     \item $\bracket{1,1;-1,1}{2,0}= \frac{1}{\sqrt{6}}$
     \item $\bracket{1,1;1,-1}{2,0}= \frac{1}{\sqrt{6}}$
     \item $\bracket{1,1;0,0}{2,0} = \frac{2}{\sqrt{6}}$
     \item $\bracket{1,1;-1,1}{1,0} = -\frac{\sqrt{2}}{2}$
     \item $\bracket{1,1;1,-1}{1,0} = \frac{\sqrt{2}}{2}$
     \item $\bracket{1,1;-1,1}{0,0} = \frac{1}{\sqrt{3}}$
     \item $\bracket{1,1;0,0}{0,0} = -\frac{1}{\sqrt{3}}$
    \end{enumerate}
  \end{solution}

  
  \item A hydrogen atom is in the $^2P_{3/2}$ state with the projection of the
    total angular momentum on the z-axis $m=-1/2$. What is the probability to
    find the electron with spin up (i.e. $m_s=1/2$)?
    \begin{solution}
      From the spectroscopic notation given, we have that $s=1/2$, $j=3/2$, and
      therefore $\ell = 1$. Furthermore, because we know the projection of the
      total angular momentum is $-1/2$  we have that $m = -1/2$.
      Therefore, in the coupled basis, we have that the state can be represented
      by
      \begin{equation}
        \ket{\frac{3}{2}, -\frac{1}{2}}
      \end{equation}
      Using the Clebsch-Gordon coefficients for $j_1=1$, $j_2=1/2$ (page 375
      Table 11.3 in McIntyre) we find that the uncoupled representation is
      \begin{equation}
        \ket{\frac{3}{2},-\frac{1}{2}} = \sqrt{\frac{2}{3}}\ket{0,-\frac{1}{2}}+\frac{1}{\sqrt{3}}\ket{-1,\frac{1}{2}}
      \end{equation}
      where the second index is the $m_s$ quantum number. Therefore the
      probability to find the electron with spin up is
      \begin{equation}
        P(S_z=\hbar/2) = \left|\bra{m_\ell, \frac{1}{2}}\Big(\sqrt{\frac{2}{3}}\ket{0,-\frac{1}{2}}+\frac{1}{\sqrt{3}}\ket{-1,\frac{1}{2}}\Big)\right|^2 = \frac{1}{3}
      \end{equation}
    \end{solution}
    
  \item Consider a system of three spin $1/2$ particles whose interaction is
    described by the following Hamiltonian
    \begin{equation}
      H = -A\left( \mathbf{S}_1\cdot\mathbf{S}_3 +\mathbf{S}_2\cdot\mathbf{S}_3\right)
    \end{equation}
    Find the system's energy levels and degeneracies.
    \begin{solution}
      In order to choose an appropriate CSCO for this problem, let's try to
      simplify the Hamiltonian in (36) to be in terms of operators we understand
      how to act on eigenstates. We have that
      \begin{align}
        \mathbf{S}_T^2 &\equiv (\mathbf{S}_1^2+\mathbf{S}_2^2+\mathbf{S}_3^2)^2 \\
                       &= \mathbf{S}_1^2+\mathbf{S}_2^2+\mathbf{S}_3^2+2\mathbf{S}_1\cdot\mathbf{S}_2 +2\mathbf{S}_2\cdot\mathbf{S}_3+2\mathbf{S}_1\cdot\mathbf{S}_3 \\
        \Rightarrow 2\left( \mathbf{S}_1\cdot\mathbf{S}_3 +\mathbf{S}_2\cdot\mathbf{S}_3\right)&= \mathbf{S}_T^2-\mathbf{S}_1^2-\mathbf{S}_2^2-\mathbf{S}_3^2-2\mathbf{S}_1\cdot\mathbf{S}_2 \\
                       &\text{and furthermore,} \\
        \mathbf{S}_{12}^2 &\equiv \left(\mathbf{S}_1+\mathbf{S}_2\right)^2 \\
        &= \mathbf{S}_1^2+\mathbf{S}_2^2+2\mathbf{S}_1\cdot\mathbf{S}_2
      \end{align}
      and therefore, we may rewrite the Hamiltonian as 
      \begin{align}
        H = -\frac{A}{2}\Big(\mathbf{S}_T^2-\mathbf{S}_{12}^2-\mathbf{S}_3^2\Big)
      \end{align}
      Therefore, we can take our CSCO to be the set of operators $\{H,
      \mathbf{S}_1^2, \mathbf{S}_3^2, \mathbf{S}_{12}^2, \mathbf{S}_T^2,
      \mathbf{S}_{12z}, \mathbf{S}_{Tz}\}$ with the energy eigenstates
      identified by the kets
      \begin{equation}
        \ket{s_1,\; s_3,\; s_{12},\; s_T,\; m_{12},\; m_T}
      \end{equation}
      As $s_1=s_3=1/2$, we can further simplify to the more compact notation
      \begin{equation}
        \ket{s_{12}, s_T ; m_{12}, m_T}
      \end{equation}
      From this representation we can see that
      \begin{align}
        H\ket{s_{12}, s_T ; m_{12}, m_T} &= -\frac{A}{2}\Big(\hbar^2s_T(s_T+1)-\hbar^2s_{12}(s_{12}+1)-\hbar^2\frac{3}{4}\Big) \ket{s_{12}, s_T ; m_{12}, m_T}\\
        \Rightarrow E_{s_{12},s_T} &= -\frac{A\hbar^2}{2}\Big(s_T(s_T+1)-s_{12}(s_{12}+1)-\frac{3}{4}\Big)
      \end{align}
      Now we must find the possible energy levels and their degeneracies. First,
      let's consider a system of only two spin-1/2 particles with spins $s_1$,
      $s_2$ and total spin $s_{12}$. The possible values for $s_{12}$ are given
      by
      \begin{equation}
        |s_1-s_2|\leq s_{12}\leq s_1+s_2 \Rightarrow s_{12}\in\{0, 1\}
      \end{equation}
      Furthermore, the projection of the total spin must obey
      \begin{equation}
        -s_{12}\leq m_{12}\leq s_{12} \rightarrow m_{12}\in\{-s_{12}, -s_{12}+1, ..., s_{12}-1, s_{12}\}
        \end{equation}
        Using the coupled basis $\ket{s_{12},m_{12}}$, this leads to the singlet and triplet states given by
        \begin{equation}
          \begin{cases}
            \ket{0,0} &\text{singlet} \\
            \ket{1,1},\;\ket{1,0},\;\ket{1,-1} &\text{triplet}
          \end{cases}
        \end{equation}
       Now we may consider what will happen if we add another spin-1/2 particle.
       We have that
       \begin{equation}
         s_T = s_1+s_2+s_3 = s_{12}+s_3
       \end{equation}
       From this we can see that
       \begin{equation}
         |s_{12}-s_3|\leq s_T \leq s_{12}+s_3 \Rightarrow s_{12}\in\left\{\frac{1}{2},\;\frac{3}{2}\right\}
       \end{equation}
       Putting all of this together, we can write the new states in the
       $\ket{s_{12},s_T;m_{12},m_T}$ coupled basis.
       \begin{align}
         \ket{0,0}&\longrightarrow \ket{0, \frac{1}{2};0, \frac{1}{2}},\quad \ket{0,\frac{1}{2};0,-\frac{1}{2} }\\
         \ket{1,0}&\longrightarrow \ket{1, \frac{1}{2};0, \frac{1}{2}},\quad \ket{1,\frac{1}{2};0,-\frac{1}{2}}\\
         \ket{1,1}&\longrightarrow \ket{1,\frac{3}{2}; 1,\frac{3}{2}},\quad \ket{1,\frac{3}{2};1,\frac{1}{2}}\\
         \ket{1,-1}&\longrightarrow \ket{1,\frac{3}{2};-1,-\frac{1}{2}},\quad\ket{1,\frac{3}{2};-1,-\frac{3}{2}}
       \end{align}
       Our energies only depend on the first two indices of our kets and
       therefore, we can see that the levels have split into two doublets and a
       quadruplet.
       \begin{align}
         E_{0,\frac{1}{2}} &= 0 &\text{multiplicity }2\\
         E_{1,\frac{1}{2}} &= -\frac{A\hbar^2}{2}\left(-2 \right) = A\hbar^2 &\text{multiplicity }2\\
         E_{1,\frac{3}{2}} &= -\frac{A\hbar^2}{2}\left( \frac{9}{4}-2-\frac{3}{4} \right)= \frac{A\hbar^2}{4} &\text{multiplicity }4
       \end{align}
       Looking back, our result makes sense if we think about how these spaces
       behave as tensor products and direct sums. We know that
       \begin{equation}
         j_1\otimes j_2 = (j_1+j_2)\oplus j_1+j_2-1\oplus |j_1-j_2|
       \end{equation}
       Therefore, we are dealing with the following
       \begin{align}
         \frac{1}{2}\otimes(1\oplus 0) &= \frac{1}{2}\otimes 1\oplus \frac{1}{2}\otimes 0 \\
         &= \frac{3}{2}\oplus \frac{1}{2}\oplus\frac{1}{2}
       \end{align}
       This tells us that we expect to have a space with $s_T=\frac{3}{2}$ of
       dimension $2s_T+1=4$ (i.e. a quadruplet) and two spaces corresponding to
       $s_T=\frac{1}{2}$ of dimension 2 (i.e. doublets). 
         
       
    \end{solution}
    
  \item  Consider a system of four spin-$1/2$ particles. Find the possible
    values of the total spin S of the system and specify the number of angular
    momentum eigenstates corresponding to each value of S.
    \begin{solution}
        We can quickly solve this problem using the ``irreducibility'' notation
        from the end of the previous problem. We know are taking the tensor
        product of (62) with another spin-1/2 space.
        \begin{align}
          j_1\otimes j_2 &= \frac{1}{2}\otimes\Big(\frac{3}{2}\oplus \frac{1}{2}\oplus\frac{1}{2}\Big) \\
                         &= \left(\frac{1}{2}\otimes\frac{3}{2}\right)\oplus\left( \frac{1}{2}\otimes\frac{1}{2} \right)\oplus \left( \frac{1}{2}\otimes\frac{1}{2} \right) \\
                         &= 2\oplus 1 \oplus 1 \oplus 0 \oplus 1 \oplus 0  \\
        \end{align}
        From this we see that the possible values for the total spin are
        $S\in\{2,1,0 \}$. For $S=2$ there are $2(2)+1=5$ possible states (i.e. a
        quintuplet). For each $S=1$ there are $2(1)+1=3$ possibles states (i.e
        3 triplets). Finally, for each $S=0$ there is $2(0)+1=1$ possible states (i.e.
        2 singlets). Adding these up gives a total of $5+3*3+2*1= 16$ total basis
        states.

        In conclusion, there are 5 states corresponding to $S=2$, there are $9$
        states corresponding to $S=1$, and there are $2$ states corresponding to
        $S=0$. 
    \end{solution}
    

    
\end{enumerate}
































\end{document}










