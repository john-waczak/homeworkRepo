\documentclass[a4paper, 11pt]{article}
\usepackage{geometry}
\geometry{letterpaper, margin=1in}
\usepackage{amsmath}
\usepackage{amssymb}  
\usepackage{amsthm}
\usepackage{ulem} 


\begin{document}
%Header-Make sure you update this information!!!!
\noindent
\large\textbf{Homework 1} \hfill \textbf{John Waczak} \\
\normalsize MTH 343 \hfill  Date: \today \\
Prof. Ren Guo  \\


\section*{1.3.15}
\textit{Prove $A \cap (B \ C) = (A \cap B) \ (A \cap C)$}\\

Recall $R \cap S = \{x : x \in R, x \in S\}$ and $R\setminus S = \{x:x\in R, x \notin S\}$. Thus: 
\begin{align*}
	A \cap (B \setminus C) &= \{x:x \in A, x \in B \setminus C \} \\ 
	&= \{x: x \in A, x \in B, x \notin C \}  \\ 
	&= \{x: x \in A \cap B, x \notin C \} \\ 
	x \notin c &\rightarrow x \notin A \cap C \quad \mbox{since} \quad A \cap C \subseteq C \\
	\mbox{Thus:} &= \{x:x \in A \cap B , x \notin A \cap C\} \\
	&= (A \cap B) \cap (A \cap C)' \\ 
	&= (A \cap B) \setminus (A \cap C) \qed
\end{align*}

\section*{1.3.19}
\textit{Given $f:A \mapsto B$ and $g:B \mapsto C$ are invertible, show $(g \circ f)^{-1} = (f^{-1} \circ g^{-1})$} \\


Recall that the composition of mappings is associative i.e. $h \circ (g \circ f) = (h \circ g) \circ f$. Thus: 

\begin{align*}
	(g \circ f) \circ (f^{-1} \circ g^{-1}) &= g \circ (f \circ f^{-1}) \circ g^{-1} \\ 
	&= g \circ Id_B \circ g^{-1} \\ 
	&= g \circ g^{-1} \\ 
	&= Id_C \\
	(f^{-1} \circ g^{-1}) \circ (g \circ f) &= f^{-1} \circ (g^{-1} \circ g) \circ f \\ 
	&= f^{-1} \circ Id_B \circ f \\ 
	&= f^{-1} \circ f \\ 
	&= Id_c \\ 
	\mbox{Thus:} \quad \quad (g \circ f)^{-1} &= (f^{-1} \circ g^{-1}) \qed
\end{align*}

\section*{1.3.26}
\textit{define $(a,b) \sim (c,d)$ if $a^2 + b^2 \leq c^2 + d^2$. Show $\sim$ is reflexive and transitive but not symmetric.}

\begin{align*}
	\mbox{Reflexive:} \quad \mbox{w.t.s} \quad (a,b) &\sim (a,b) \\ 
	\mbox{observe that:} \quad a^2 + b^2 &= a^2 + b^2 \\
	&\mbox{Thus $\sim$ is reflexive} \\ \\ 
	\mbox{Transitive:} \quad (a,b) \sim (c,d), (c,d) &\sim (e,f)  \Rightarrow (a,b) \sim (e,f) \\ 
	\mbox{observe that:} \quad (a,b) \sim (c,d) &\Rightarrow a^2 + b^2 \leq c^2 + d^2 \\ 
	(c,d) \sim (e,f) &\Rightarrow c^2 + d^2 \leq e^2 + f^2 \\ 
	\mbox{thus} \quad a^2+b^2 &\leq c^2 + d^2 \leq e^2 + f^2 \\ 
	&\Rightarrow a^2 + b^2 \leq e^2 + f^2 \\ 
	&\mbox{and so $\sim$ is transitive} \\ 
	\mbox{Not symmetric:} \quad (a,b) \sim (c,d) &\Rightarrow (c,d) \sim (a,b) \\ 
	(a,b) \sim (c,d) &\Rightarrow a^2 + b^2 \leq c^2 + d^2 \\ 
	(c,d) \sim (a,b) &\Rightarrow c^2 + b^2 \leq a^2 + d^2 \\ 
	a^2 + b^2 \leq c^2 + d^2 &\Rightarrow \Leftarrow c^2 + d^2 \leq a^2 + b^2
\end{align*}
thus we have a contradiction and so $\sim$ is not symmetric \qed

\section*{2.3.6}
\textit{prove $4 \cdot 10^{2n} + 0 \cdot 10 ^{2n-1} +5$ is divisible by 99 $\forall n \in \mathbb{N}$} \\ 
Proof by mathematical induction: 
\begin{align*}
	\mbox{let} \quad n=1, & \mbox{then} \quad 4\cdot 10^2 + 9\cdot 10 + 5 = \\ 
	&= 400 + 90 + 5 \\ 
	&= 495  \\ 
	&= 5 \cdot 99
\end{align*}

Thus the base step is true. Now assuming n=k is true, w.t.s. that n=k+1 is true. 

\begin{align*}
	4 \cdot 10^{2(k+1)}	+ 9 \cdot 10^{2(k+1)-1} + 5 &= \\ 
	&= 4 \cdot 10^{2k+2} + 0 \cdot 10^{2k+1} + 5 \\ 
	&= 100(4*10^{2k}) + 100(9*10^{2k-1}) + 5 \\ 
	&= 100(4*10^{2k} + 9*10^{2k-1}) + 5) -500 + 5 \\ 
	&= 100(4*10^{2k} + 9*10^{2k-1}) + 5) -495 \\ 
	&= 100(99*a) -495, a\in \mathbb{Z} \quad \mbox{because n=k is assumed true} \\
	&= 100a \cdot 99 - 5\cdot 99 \\ 
	&= (100a - 5)99 \\ 
	&= 99b, b\in \mathbb{Z}
\end{align*}
Thus by mathematical induction, the hypothesis $4 \cdot 10^{2n} + 0 \cdot 10 ^{2n-1} +5$ is divisible by 99 $\forall n \in \mathbb{N}$. \qed


\section*{2.3.15} 
\textit{find r and s s.t. $gcd(r,s) = ra + sb$ given $a=234$ and $b=165$} \\ 

First, we need to find the gcd of a and b which we will do using the Euclidean algorithm. Then working backwards, we will determine r and s. 

\begin{align*}
	234 &= 165\cdot 1 + 69 \\ 
	165 &= 69 \cdot 2 + 27 \\ 
	27 &= 15 \cdot 1 + 12 \\ 
	15 &= 12 \cdot 1 + 3 \\ 
	12 &= 3\cdot 4
\end{align*}

Thus, the gcd(a,b) is 3. Now  we will find $r,s \in \mathbb{Z}$

\begin{align*}
	3 &= 15 -12 \\ 
	&= (69-(2)27) - (27-15)\\
	&= 69-(2)27-27+15\\
	&= 69 - (3)27 + 15 \\ 
	&= 234-165-(3)(165-(2)69) + 69-(2)27 \\
	&=234-165-(3)165+(6)69+69-(2)27\\
	&= 234-(4)165+(7)69-(2)27\\
	&= 234-(4)165+(7)(234-165)-(2)(165-(2)69)\\
	&= 234-(4)165+(7)234-(7)165-(2)(165-(2)69)\\ 
	&= 234-(4)165+(7)234-(7)165-(2)165+(4)69 \\ 
	&= 234-(4)165+(7)234-(7)165-(2)165+(4)234-4(165)\\
	&= (1+7+4)234 + (-4-7-2-4)165 \\ 
	&=(12)234 + (-17)165 \\
	&= 2808-2805\\
	&= 3 \Rightarrow r=12, s=-17 \qed
\end{align*}

\section*{2.3.19}
\textit{Let $x,y \in \mathbb{N}$ be relatively prime. If xy is a perfect square, prove that x and y must be perfect sqaures} \\ 

I will prove this proposition by first proving a \textbf{lemma:} \textit{if n is a perfect square then each of the factors in its prime factorization must have an even power.} \\

Because n is a perfect square we can say $\exists m \in \mathbb{Z}_+$ such that $ n = m^2 $. By the fundamental theorem of arithmetic (FTA), both n and m have a unique prime factorization up to order of factors. Thus we can say: 
\begin{equation*}
	m = p_1^{a_1} p_2^{a_2} ... p_k^{a_k} 
\end{equation*}
Now since $n = m^2$ we have 
\begin{equation}
	n = m^2 = (p_1^{a_1} p_2^{a_2} ... p_k^{a_k})^2 = p_1^{2a_1} p_2^{2a_2} ... p_k^{2a_k}
\end{equation}
And so by the uniqueness of the FTA, this must be \textit{the} prime factorization of n. Now if $a_i, i \in [0, k]$ is odd then $2a_i$ is even. Similarly if $a_i$ is even then $2a_i$ is also even. Thus all the factors in the prime factorization of n must have even power. 

Now we w.t.s that given $gcd(a,b) = 1$ and xy is a perfect square $\Rightarrow$ x,y are perfect squares. Assume for contradiction that x is \textit{not} a perfect square. Then $\exists$ some $p_i$ in the prime factorization of x with an odd power. For xy to be a perfect square then $p_i$ must also divide y so that $p_i$ will have even power in the prime factorization of xy. This contradicts the assumption that $gcd(a,b) = 1$ and thus we say that x and y must \textit{both} be perfect squares. 


\end{document}








