\documentclass[a4paper, 11pt]{article}
\usepackage{geometry}
\geometry{letterpaper, margin=1in}
\usepackage{amsmath}
\usepackage{amssymb}  
\usepackage{amsthm}
\usepackage{ulem} 
\usepackage{graphicx}
\graphicspath{ {images/} }

\begin{document}
	%Header-Make sure you update this information!!!!
	\noindent
	\large\textbf{Homework 2} \hfill \textbf{John Waczak} \\
	\normalsize PH 431 \hfill  Date: \today \\
	Prof. Ren Guo \\ 

\section*{4.4.8}
\textit{List all of the cyclic subgroups of U(30).}\\

\noindent Recall that U(30) is the set of elements in $\mathbb{Z}_{30}$ that are relatively prime to 30. Thus: 
	\begin{equation*}
		U(30) = \{1,7,11,13,17,19,23,29\}
	\end{equation*}

We have that $1 = id$ for U(30) but $\langle 1 \rangle$ is a trivial subgroup. We can calculate the other cyclic subgroups by analyzing powers of each element of U(30). We have that: 
	\begin{align*}
		7& \\ 
		7^2& \mod(30) = 19 \\ 
		7^3& \mod(30) = 13 \\ 
		7^4& \mod(30) = 1 \\
		11& \\ 
		11^2& \mod(30) = 1 \\ 
		13& \\
		13^2& \mod(30) = 19 \\ 
		13^3& \mod(30) = 7 \\ 
		13^4& \mod(30) = 1 \\ 
		17 \\ 
		17^2& \mod(30) = 19 \\ 
		17^3& \mod(30) = 23 \\ 
		17^4& \mod(30) = 1 \\ 
		23 \\ 
		23^2& \mod(30) = 19 \\ 
		23^3& \mod(30) = 17 \\ 	
		23^4& \mod(30) = 1 \\ 
		29& \\ 
		29^2& \mod(30) = 1 \\ 	 
	\end{align*}
Thus from this information we can conclude that the cyclic subgroups of U(30) are: 
	\begin{align*}
		\langle 7 \rangle &= \langle 13 \rangle = \{ 1, 7, 13, 19\} \\ 
		\langle 11 \rangle &= \{ 1, 11 \} \\ 
		\langle 17 \rangle &= \langle 23 \rangle = \{1, 17, 19, 23 \} \\ 
		\langle 29 \rangle &= \{ 1, 29 \}
	\end{align*}
	
\section*{4.4.25}
\textit{Let p be prime and r be a positive integer. How many generators does $\mathbb{Z}_{p^r}$ have?} \\ 

\noindent Note that $\mathbb{Z}_{p^r}$ has exactly $p^r$ elements. Now, an element g of $\mathbb{Z}_{p^r}$ is a generator if $1 \leq g < p^r$ and $gcd(g, p^r)=1$. Since p is prime, the only possible values of $gcd(g, p^r)$ are $p, p^2, p^3, ..., p^r$. The only way $gcd{g, p^r}\neq 1$ is if $g = mp$ for some integer m. This set is: 
	\begin{equation*}
		\{p, 2p, 3p...pp, 2pp, 3pp...p^3, 2p^3, 3p^3...p^{r-1}, 2p^{r-1}, ...pp^{r-1}\}
	\end{equation*} 
We can see this set has $p^{r-1}$ elements and so the set of values with $gcd(g, p^r) = 1$ has $p^r-p^{r-1}$ elements. Thus by definition, $\mathbb{Z}_{p^r}$ has $p^r-p^{r-1}$ elements.\\ 

\section*{4.3.2.d}
\textit{evaluate $(1423)(34)(56)(1324)$.} \\ 

\noindent To evaluate this composition of cycles, I will first name each one and then list out the full mapping for each cycle as I'm still getting used to cycle notation. 
	\begin{align*}
		\delta = (1432) \quad \gamma = (34) \quad \beta = (56) \quad \alpha = (1324) 
	\end{align*}
And the mappings are: 
	\begin{align*}
		\delta(1) = 4 \quad \gamma(1) = 1 \quad \beta(1) = 1 \quad \alpha(1) = 3 \\ 
		\delta(2) = 3 \quad \gamma(2) = 2 \quad \beta(2) = 2 \quad \alpha(2) = 4 \\ 
		\delta(3) = 1 \quad \gamma(3) = 4 \quad \beta(3) = 3 \quad \alpha(3) = 2 \\ 
		\delta(4) = 2 \quad \gamma(4) = 3 \quad \beta(4) = 4 \quad \alpha(4) = 1 \\ 
		\delta(5) = 5 \quad \gamma(5) = 5 \quad \beta(5) = 6 \quad \alpha(5) = 5 \\ 
		\delta(6) = 6 \quad \gamma(6) = 6 \quad \beta(6) = 5 \quad \alpha(6) = 6 
	\end{align*}
Thus we have that the following is the mapping for the composition of cycles $\delta\gamma\beta\alpha$: 
	\begin{align*}
		\delta\gamma\beta\alpha(1) = \delta\gamma\beta(3) = \delta\gamma(3) = \delta(4) = 2 \\ 
		\delta\gamma\beta\alpha(2) = \delta\gamma\beta(4) = \delta\gamma(4) = \delta(3) = 1 \\ 
		\delta\gamma\beta\alpha(3) = \delta\gamma\beta(2) = \delta\gamma(2) = \delta(2) = 3 \\ 
		\delta\gamma\beta\alpha(4) = \delta\gamma\beta(1) = \delta\gamma(1) = \delta(1) = 4 \\ 
		\delta\gamma\beta\alpha(5) = \delta\gamma\beta(5) = \delta\gamma(6) = \delta(6) = 6 \\ 
		\delta\gamma\beta\alpha(6) = \delta\gamma\beta(6) = \delta\gamma(5) = \delta(5) = 5
	\end{align*}
From this mapping we can see that 3 and 4 are fixed and therefore this cycle is equivalent to: 
	\begin{equation*}
		\delta\gamma\beta\alpha = (12)(56)
	\end{equation*}

\section*{5.3.3.d} 
\textit{Express the following permutation as a product of transpositions and identify then as even or odd} \\ 

\noindent To begin we will first simplify the cycle as much as possible. 
	\begin{align*}
		\rho = (17254) \quad \tau = (1423) \quad \sigma = (154632) \\ \\ 
		\rho(1) = 7 \quad \tau(1) = 4 \quad \sigma(1) = 5 \\ 
		\rho(2) = 5 \quad \tau(2) = 3 \quad \sigma(2) = 1 \\ 
		\rho(3) = 3 \quad \tau(3) = 1 \quad \sigma(3) = 2 \\ 
		\rho(4) = 1 \quad \tau(4) = 2 \quad \sigma(4) = 6 \\ 
		\rho(5) = 4 \quad \tau(5) = 5 \quad \sigma(5) = 4 \\ 
		\rho(6) = 7 \quad \tau(6) = 6 \quad \sigma(6) = 3 \\ 
		\rho(7) = 2 \quad \tau(7) = 7 \quad \sigma(7) = 7 \\ \\ 
		\rho\tau\sigma(1) = \rho\tau(5) = \rho(5) = 4 \\ 
		\rho\tau\sigma(2) = \rho\tau(1) = \rho(4) = 1 \\ 
		\rho\tau\sigma(3) = \rho\tau(2) = \rho(3) = 3 \\ 
		\rho\tau\sigma(4) = \rho\tau(6) = \rho(6) = 6 \\ 
		\rho\tau\sigma(5) = \rho\tau(4) = \rho(2) = 5 \\ 
		\rho\tau\sigma(6) = \rho\tau(3) = \rho(1) = 7 \\ 
		\rho\tau\sigma(7) = \rho\tau(7) = \rho(7) = 2 \\ \\ 
		\rho\tau\sigma  = (14672)
	\end{align*}
Now that we have one cycle, we can easily decompose it as follows: 
	\begin{equation*}
		(14672) = (12)(17)(16)(14)(31)(13)(51)(15)
	\end{equation*}
Counting the number of 2-cycles gives that the original cycle must be an even permutation. 

\section*{5.3.4}
\textit{Find $(a_1a_2a_3...a_{n-1}a_n)^{-1}$. } \\ 

\noindent Let $ \sigma = (a_1a_2a_3...a_{n-1}a_n)$. I claim that $\sigma^{-1}=(a_na_{n-1}...a_3a_2a_1) = (a_1a_{n}a_{n-1}...a_3a_2)$ is the inverse to $\sigma$. Let $x = a_i$ then $\sigma(a_{i-1})= a_i$ and so $\sigma^{-1}(a_i)=a_{i-1}$ hence, the 'reverse order' of $\sigma^{-1}$. For $x = a_1$ we have the special case of $\sigma(a_k)=a_1$ and so by definition $\sigma^{-1}(a_1)=a_k$ if $x\neq a_i$ then $\sigma(x) = x$ which implies $\sigma^{-1}(x)=x$. Thus $\sigma^{-1}$ is the inverse for $\sigma$ because: 
	\begin{align*}
		\sigma^{-1}\sigma(a_{i-1}) = \sigma^{-1}(a_i) = a_{i-1} \\ 
		\sigma\sigma^{-1}(a_i) = \sigma(a_{i-1}) = a_i
	\end{align*}
So every element maps to itself, i.e. $\sigma\sigma^{-1} = \sigma^{-1}\sigma = id$

\section*{5.3.18}
\textit{Show $A_n$ is non-Abelian for $n\geq 4$.}\\

\noindent Note that it is sufficient to find two examples from $A_4$ that do not commute because if they are in $A_4$ they must also be in every $A_n$ with $r \geq 4$. Let us examine the even permutations $\alpha = (123) = (13)(12)$  and $\beta = (234)=(24)(23)$. 
	\begin{align*}
		\alpha \beta = (123)(234) = (12)(34) \\ 
		\beta \alpha = (234)(123) = (13)(24) \\ 
		\Rightarrow \alpha \beta \neq \beta \alpha 
	\end{align*}
Thus since $\alpha, \beta \in A_n, n\geq 4$ we have that $A_n$ is non-Abelian. 

\end{document}
















