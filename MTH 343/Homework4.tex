\documentclass[a4paper, 11pt]{article}
\usepackage{geometry}
\geometry{letterpaper, margin=1in}
\usepackage{amsmath}
\usepackage{amssymb}  
\usepackage{amsthm}
\usepackage{ulem} 
\usepackage{graphicx}
\graphicspath{ {images/} }

\begin{document}
%Header-Make sure you update this information!!!!
\noindent
\large\textbf{Homework 4} \hfill \textbf{John Waczak} \\
\normalsize MTH 343 \hfill  Date: \today \\
Prof. Ren Guo  \\

\section*{6.4.5b}
\textit{Find all the left and right cosets of $\langle 3\rangle$ in U(8).}\\

Recall $U(8) = \{1, 3, 5, 7\}$ and we have that $\langle 3 \rangle = {1, 3}$. Thus we have that: 
	\begin{align*}
		1 \langle 3 \rangle &= \langle 3 \rangle 1 = \{1 , 3\} \\ 
		3 \langle 3 \rangle &= \langle 3 \rangle 3 = \{3, 1\} \\ 
		5 \langle 3 \rangle &= \langle 3 \rangle 5 = \{5, 7\} \\ 
		7 \langle 3 \rangle &= \langle 3 \rangle 7 = \{7, 5\} 
	\end{align*}
Thus we have that the left cosets and right cosets are the same... i.e. $L_H = R_H = \{\{1,3\},\{5,7\}\}$ which we can see is a partition of U(8) as expected. 



\section*{6.4.5h}
\textit{Find all the left and right cosets of $H = \{(1), (123), (132)\}$ in $S_4$.} \\ 

Recall that $S_4$ is defined as: 
	\begin{align*}
		S_4 = \{&(1), (12), (13), (14), (23), (24), \\
		 &(34), (12)(34), (13)(24), (14)(23), (123), (124), \\
		 & (132), (134), (142), (143), (234), (243), \\
		 & (1234), (1243), (1324), (1342), (1423), (1432)\}
	\end{align*}
Now we need to look at gH where g is in $S_4$. Note that we know both the left and right cosets must have the same number of elements and the index H in $S_4$ is $24/3 = 8$ Thus we can stop once we get 8 unique cosets. 
	\begin{align*}
		(1)H 		&= \{(1)(1), (1)(123), (1)(132)\} \\ 
					&= \{(1), (123), (132)\} \\ 
		(12)H 		&= \{(12)(1), (12)(123), (12)(132)\} \\ 
					&= \{(12), (23), (12)\} \\ 
		(13)H		&= \{(13)(1), (13)(123), (13)(132)\} \\ 
					&= \{(13), (12), (23)\} \\ 
		(14)H 		&= \{(14)(1), (14)(123), (14)(132)\} \\ 
					&= \{(14), (1234), (1324)\} \\ 
		(23)H		&= \{(23)(1), (23)(123), (23)(132)\} \\ 
					&= \{(23), (13), (12)\} \\ 
		(24)H		&= \{(24)(1), (24)(123), (24)(132)\} \\ 
					&= \{(24), (1423), (1342)\} \\ 
		(34)H 		&= \{(34)(1), (34)(123), (34)(132)\} \\ 
					&= \{(34), (1243),(1432)\} \\ 
		(12)(34)H	&= \{(12)(34)(1), (12)(34)(123), (12)(34)(132)\} \\ 
					&= \{(12)(34), (243), (143)\} \\ 
		(13)(24)H	&= \{(13)(24)(1), (13)(24)(123), (13)(24)(132)\} \\ 
					&= \{(13)(24), (142), (234)\} \\ 
		(14)(23)H	&= \{(14)(23)(1), (14)(23)(123), (14)(23)(132)\} \\ 
					&= \{(14)(23), (134), (124)\}
 	\end{align*}
Thus we have found all of the left cosets. They form a partition of $S_4$: 
	\begin{align*}
		L_H = \{&\{(1), (123), (132)\}\\
				&\{(12), (23), (12)\} \\ 
				&\{(14), (1234), (1324)\} \\  
				&\{(24), (1423), (1342)\} \\ 
				&\{(24), (1423), (1342)\} \\ 
				&\{(12)(34), (243), (143)\} \\ 
				&\{(13)(24), (142), (234)\} \\ 
				&\{(14)(23), (134), (124)\}	\}
	\end{align*}t
Now we will do the same for the right cosets although we will find the partition is not the same as that created by $L_H$. 
	\begin{align*}
		H(1)		&= \{(1)(1), (123)(1), (132)(1)\} \\ 	
					&= \{(1), (123), (132)\} \\ 
		H(12)		&= \{(1)(12), (123)(12), (132)(12)\} \\ 
					&= \{(12), (13), (23)\} \\ 
		H(13) 		&= \{(1)(13), (123)(13), (132)(13)\} \\ 
					&= \{(13), (23), (12)\} \\ 
		H(14) 		&= \{(1)(14), (123)(14), (132)(14)\} \\ 
					&= \{(14), (1423), (1432)\} \\ 
		H(23) 		&= \{(1)(23), (123)(23), (132)(23)\} \\ 
					&= \{(23), (12), (13)\} \\ 
		H(24) 		&= \{(1)(24), (123)(24), (132)(24)\} \\ 
					&= \{(24), (1243), (1324)\} \\ 
		H(34) 		&= \{(1)(34), (123)(34), (132)(34)\} \\ 
					&= \{(34), (1234), (1342)\} \\ 
		H(12)(34) 	&= \{(1)(12)(34), (123)(12)(34), (132)(12)(34)\} \\ 
					&= \{(12)(34), (341), (234)\} \\ 
		H(13)(24) 	&= \{(1)(13)(24), (123)(13)(24), (132)(13)(24)\} \\ 
					&= \{(13)(24), (243), (124)\} \\ 
		H(14)(23) 	&= \{(1)(14)(23), (123)(14)(23), (132)(14)(23)\} \\ 
					&= \{(14)(23), (142), (143)\} 
 	\end{align*}
Thus we have found the right cosets of H in $S_4$. They form the partition: 
	\begin{align*}
		R_H = \{& \{(1), (123), (132)\} \\
				& \{(12), (13), (23)\} \\
				& \{(14), (1423), (1432)\} \\ 
				& \{(24), (1243), (1324)\} \\ 
				& \{(34), (1234), (1342)\} \\
				& \{(12)(34), (341), (234)\} \\ 
				& \{(13)(24), (243), (124)\} \\ 
				& \{(14)(23), (142), (143)\} \}
	\end{align*}

\section*{6.4.14}
\textit{given $g^n = e$ prove the order of g divides n.} \\

\noindent By definition of the order of an element g in the group G, the order is the smallest integer k such that $g^k = e$. Thus there are two cases we must consider: $n\neq k$ and $n = k$. \\

\noindent If $n=k$ then we have that n clearly divides itself. Thus the proposition is true for the first case. Now if $n \neq k$ then for some $q,r \in \mathbb{Z}$ the division algorithm tells us that $n = qk + r$. Thus the statement of the proposition becomes: $g^n = g^{qk+r} = g^{qk}g^r = e$. Now $g^{qk}=e$ as $g^{qk} = (g^{k})^q = e^q = e$. Therefore in $r = 0$ and so then we have $n = qk$ which means that k, the order of g divides n. \qed


\section*{6.4.19} 
\textit{Let H and K be subgroups of G. Prove that $gH \cap gK$ is a coset of $H \cap K$ in G.}\\ 

\noindent Suppose $gH \cap gK \neq \emptyset$. Now let $f \in gH \cap gK$. Then by definition of the intersection of two sets, we have that:
	\begin{equation*}
		f \in gH \quad \text{and} \quad f \in gK 
	\end{equation*}
\noindent This implies that $f = gh = gk$ for some $h \in H, k \in K$. Since G is a subgroup, $\exists g^{-1}$ such that: 
	\begin{align*}
		g^{-1}f &= g^{-1}gh = g^{-1}gk \\ 
		g^{-1}f &= h = k \\ 
		\Rightarrow g^{-1}f &\in H \cap K \\ 
		f &\in g(H \cap K) \\ 
	\end{align*}
i.e. f is an element of $g(H \cap K)$ which is a coset of $H \cap K$ in G. \qed


\section*{9.3.2} 
\textit{Prove that $\mathbb{C}^\star$ is isomorphic to the subgroup of $GL_2(\mathbb{R})$ consisting of matrices of the form: $\begin{pmatrix}
		a & b \\ 
		-b & a 
	\end{pmatrix} \forall a,b \in \mathbb{R} \quad  \text{s.t.} \quad a^2 +b^2 \neq 0$. }\\

\noindent Recall that $\mathbb{C}^\star$ is $\{\mathbb{C}\setminus\{0\}, \cdot\}$. I claim that the mapping $\phi:\mathbb{C}^\star \rightarrow S$, the subgroup of $GL_2(\mathbf{R})$ defined by: 
	\begin{equation*}
		\phi(\gamma + i\delta) = \begin{pmatrix}
			\gamma & \delta \\ 
			-\delta & \gamma 
		\end{pmatrix}
	\end{equation*}
is an isomorphism between the two groups. Clearly this function is a bijection as the inverse can be seen to be: 
	\begin{equation*}
		\phi_{-1}\begin{pmatrix}
			\gamma & \delta \\ 
			-\delta & \gamma
		\end{pmatrix} = \gamma + i \delta \in \mathbb{C}^\star
	\end{equation*}
Now all that is left to show is that for any $z_1, z_2 \in \mathbb{C}^\star$ we have that $\phi(z_1\cdot z_2) = \phi(z_1) \cdot \phi(z_2)$. Let $z_1 = \alpha + i\beta$ and $z_2 = a + ib$. We have that: 
	\begin{align*}
		\phi(z_1 \cdot z_2) &= \phi((\alpha + i\beta)(a + ib)) \\ 
							&= \phi((\alpha a - \beta b) + i (\alpha b + a \beta) \\ 
							&= \begin{pmatrix}
								\alpha a - \beta b & \alpha b + a \beta \\ 
								-\alpha b - a\beta & \alpha a + \beta b 
							\end{pmatrix} \\ 
		\phi(z_1)\cdot \phi(z_2) 	&= \phi(\alpha + i\beta) \cdot \phi(a + ib) \\ 
									&= \begin{pmatrix}
										\alpha & \beta \\ 
										-\beta & \alpha 
									\end{pmatrix} \cdot \begin{pmatrix}
										a & b \\ 
										-b & a 
									\end{pmatrix}\\ 
							&= \begin{pmatrix}
								\alpha a + \beta(-b) & \alpha b + \beta a \\ 
								-\beta a - \alpha b & -\beta b + \alpha a 
							\end{pmatrix} \\ 
							&= \begin{pmatrix}
								\alpha a - \beta b & \alpha b + a \beta \\ 
								-\alpha b - a \beta & \alpha a - \beta b 
								\end{pmatrix} \\
		\Rightarrow \phi(z_1 \cdot z_2) &= \phi(z_1)\cdot \phi(z_2)						
	\end{align*}
We have constructed a bijection $\phi$ that preserves group operations. Thus we have proved that $\mathbb{C}^\star \cong S$. 

\section*{9.3.3}
\textit{prove or disprove that $U(8)\cong \mathbb{Z}_4$}\\ 

Suppose that there exists an isomorphism $\phi: U(8) \rightarrow \mathbb{Z}_4$. Then by theorem 9.6 there must exist an inverse mapping $\phi^{-1}: \mathbb{Z}_4 \rightarrow U(8)$ since $\phi$ is a bijection. Now again by theorem 9.6 because $\phi^{-1}$ is an isomorphism, if $\mathbb{Z}_4$ is cyclic then U(8) must be cyclic. We know $\mathbb{Z}_4$ is cyclic with $\langle 1 \rangle$ the generator. We can test this by examining the powers of each element in U(8):
	\begin{align*}
		U(8) &= \{1, 3, 5, 7\} \\ 
		 1^n &= 1 \\ 
		 3^2 mod(8) &= 1 \\ 
		 5^2 mod(8) &= 1 \\ 
		 7^2 mod(8) &= 1 
	\end{align*}
From this we can see that none of the elements in $U(8)$ generate $U(8)$. Therefore, it can \textit{not} be cyclic and so we have a contradiction to our supposition. Therefore we conclude that $U(8) \ncong \mathbb{Z}_4$
\end{document}




























