\documentclass[a4paper, 11pt]{article}
\usepackage{geometry}
\geometry{letterpaper, margin=1in}
\usepackage{amsmath}
\usepackage{amssymb}  
\usepackage{amsthm}
\usepackage{ulem} 
\usepackage{graphicx}
\graphicspath{ {images/} }

\begin{document}
%Header-Make sure you update this information!!!!
\noindent
\large\textbf{Homework 5} \hfill \textbf{John Waczak} \\
\normalsize MTH 343 \hfill  Date: \today \\
Prof. Ren Guo  \\

\section*{9.3.15}
\textit{List all of the elements of $\mathbb{Z}_2 \times \mathbb{Z}_4$.} \\ 

\noindent Recall the definitions fo the following groups: 
	\begin{align*}
		\mathbb{Z}_4 &= \{0, 1, 2, 3\} \\ 
		\mathbb{Z}_2 &= \{0, 1\} \\ 
	\end{align*}
Thus, we create $\mathbb{Z}_2 \times \mathbb{Z}_4$ via the Cartesian product of the two sets: 
	\begin{align}
		\mathbb{Z}_2 \times \mathbb{Z}_4 = \{(0,0), (0, 1), (0,2), (0,3), (1,0), (1,1),(1,2), (1,3)\}
	\end{align}

\section*{9.3.16.b}
\textit{Find the order of $(6,15,4) \in \mathbb{Z}_{30}\times\mathbb{Z}_{45}\times\mathbb{Z}_{24}$.}\\

\noindent Recall Corollary 9.18: For $(g_1...g_n)\in \Pi_i G_i$ if $g_i$ has finite order $r_i$ then the order of $(g_1...g_n)$ is the least common multiple of $r_1,...r_n$. \\ 

\noindent So, we simply need to find the individual order of 6, 15, and 4 in order to determine the order of (6,15,4). Observe that: 
	\begin{align*}
		 6*5 \mod (30) &= 0 \\
		15*3 \mod (45) &= 0 \\ 
		4*6 \mod (24) &= 0  
	\end{align*}
So now that we have the order of each number in the tuple, the order of (6,15,4) is simply: 
	\begin{eqnarray}
		LCM(6,15,4) = 30 
	\end{eqnarray}
Thus the order of (6, 15, 4) is 30 by Corollary 9.18. \qed

\section*{9.3.32}
\textit{Prove that $U(5)\cong \mathbb{Z}_4$. Can you generalize this for U(p) where p is prime?}\\

\noindent Recall that $U(5) = \{1,2,3,4\}$ and $\mathbb{Z}_4 = \{0,1,2,3\}$. To prove that these two groups are isomorphic, we simply need to prove that U(5) is cyclic as both U(5) and $\mathbb{Z}_4$ have the same order (4).  
	\begin{align*}
		2 \mod(5) &= 2 \\ 
		2^2 \mod(5) &=  4 \\ 
		2^3 \mod(5) &= 8 \mod(5) = 3 \\ 
		2^4 \mod(5) &= 16 \mod(5) = 1 = e 
	\end{align*}
Thus 2 is a generator for U(5) and therefore U(5) is cyclic. U(5) is a cyclic group of order 4 and so by theorem 9.8, $U(5) \cong \mathbb{Z}_4$. \\ 

\noindent In order to extend this theorem to groups of the form $U(p)$ where p is prime, we need to be able to prove that $U(p)$ is cyclic so long a p is prime. First let's consider the type of elements in U(p). By definition this is all of the non-zero elements of $\mathbb{Z}_p$ that are relatively prime to p (i.e. k such that $gcd(k,p)=1$). Since p itself is prime it's only divisors are 1 and itself. Thus U(p) is necessarily all of the integers from 1 up to p-1: 
	\begin{align}
		U(p) = \{1,2,3...p-1\}
	\end{align}
It can be shown that U(p) is cyclic and so by theorem 9.8 it must be isomorphic to $\mathbb{Z}_{p-1}$. 

\section*{10.3.1.b}
\textit{Determine whether $H=\{(1),(123),(132)\}$ is a normal subgroup of $A_5$.}

Recall that $A_5$ is the subgroup of even permutations of $S_5$, the symmetric group of permutations on 5 letters. Observe that the cycle (134) is an element of $A_5$ as it is equivalent to (14)(13). Now for H to be a normal group we must have $gH=Hg \quad \forall g \in A_5$. The following will show that this is not true for the cycle in $A_5$ listed above: 
	\begin{align}
		(134)H &= \{(134)(1), (134)(123), (134)(132)\} \\ 
				&= \{(134), (124), (14)(23)\} \\ 
		H(134) &= \{(1)(134), (123)(134), (132)(134)\} \\ 
				&= \{(134), (234), (12)(34)\}  
	\end{align}
Thus $(134)H \neq H(134)$ and so H is not a normal subgroup of $A_5$. 
\section*{10.3.4.c}
\textit{Prove that U is normal in T}\\ 

\noindent By definition we have the following: 
	\begin{align}
		T &= \Bigg\{ \begin{pmatrix}
			a & b \\ 
			0 & c \\ 
		\end{pmatrix} : ac\neq 0, \quad a,b \in \mathbb{R}\Bigg\}\\ 
		U &= \Bigg\{ \begin{pmatrix}
			1 & x \\ 
			0 & 1 \\
		\end{pmatrix} : x \in \mathbb{R} \Bigg\}
	\end{align}
By theorem 10.3 we need to show that $\forall m \in T$, $mUm^{-1}=U$. Let: 
	\begin{align*}
		m &= \begin{pmatrix}
			a & b \\ 
			0 & c \\ 
		\end{pmatrix} \\ 
		\Rightarrow m^{-1} &= \frac{1}{ac}\begin{pmatrix}
			c & -b \\ 
			0 & a
		\end{pmatrix}
	\end{align*}
Now with this we have: 
	\begin{align*}
		mUm^{-1} &= \frac{1}{ac}\begin{pmatrix}
			a & b \\ 
			0 & c 
		\end{pmatrix} \begin{pmatrix}
			1 & x \\ 
			0 & 1 
		\end{pmatrix} \begin{pmatrix}
			c & -b \\ 
			0 & a 
		\end{pmatrix} \\ 
		&= \frac{1}{ac}\begin{pmatrix}
			a & b \\ 
			0 & c 
		\end{pmatrix}\begin{pmatrix}
			c & -b + ax \\ 
			0 & ac 
		\end{pmatrix} \\ 
		&= \frac{1}{ac}\begin{pmatrix}
			ac & -ab +a^2x + ab \\ 
			0 & ac 
		\end{pmatrix}\\
		&= \begin{pmatrix}
			1 & \frac{a}{c}x \\ 
			0 & 1 
		\end{pmatrix}
	\end{align*}
Now if we let $x' = \frac{a}{c}x$ Then we have shown: 
	\begin{align}
		mUm^{-1} &= \begin{pmatrix}
			1 & x' \\ 
			0 & 1 
		\end{pmatrix}
	\end{align}
Thus since x' is arbitrary in $\mathbb{R}$ we have shown that $mUm^{-1} = U$ and thus U is normal in T. \qed 

\section*{10.3.6}
\textit{If G is abelian, prove G/H is also abelian} \\ 

\noindent Recall that G is abelian if $\forall a,b \in G$, $ab = ba$. By definition G/H is the group of cosets of H in G under the operation $(aH)(bH) = abH$. Now let $\alpha,\beta \in G$. By definition of G/H, 
	\begin{equation*}
		(\alpha H)(\beta H) = (\alpha \beta)H 
	\end{equation*}
Since we are given that G is abelian, 
	\begin{align*}
		(\alpha \beta)H &= (\beta \alpha)H \\ 
						&= (\beta)H(\alpha)H \\
		\Rightarrow (\alpha)H(\beta)H &= (\beta)H(\alpha)H 				
	\end{align*}
Therefore, because G is abelian, G/H must also be abelian.  \qed

















\end{document}
