\documentclass[a4paper, 11pt]{article}
\usepackage{geometry}
\geometry{letterpaper, margin=1in}
\usepackage{amsmath}
\usepackage{amssymb}  
\usepackage{amsthm}
\usepackage{ulem} 
\usepackage{graphicx}
\graphicspath{ {images/} }

\begin{document}
%Header-Make sure you update this information!!!!
\noindent
\large\textbf{Homework 5} \hfill \textbf{John Waczak} \\
\normalsize MTH 343 \hfill  Date: \today \\
Prof. Ren Guo  \\

\section*{9.3.15}
\textit{List all of the elements of $\mathbb{Z}_2 \times \mathbb{Z}_4$.} \\ 

\noindent Recall the definitions fo the following groups: 
	\begin{align*}
		\mathbb{Z}_4 &= \{0, 1, 2, 3\} \\ 
		\mathbb{Z}_2 &= \{0, 1\} \\ 
	\end{align*}
Thus, we create $\mathbb{Z}_2 \times \mathbb{Z}_4$ via the Cartesian product of the two sets: 
	\begin{align}
		\mathbb{Z}_2 \times \mathbb{Z}_4 = \{(0,0), (0, 1), (0,2), (0,3), (1,0), (1,1),(1,2), (1,3)\}
	\end{align}

\section*{9.3.16.b}
\textit{Find the order of $(6,15,4) \in \mathbb{Z}_{30}\times\mathbb{Z}_{45}\times\mathbb{Z}_{24}$.}\\

\noindent Recall Corollary 9.18: For $(g_1...g_n)\in \Pi_i G_i$ if $g_i$ has finite order $r_i$ then the order of $(g_1...g_n)$ is the least common multiple of $r_1,...r_n$. \\ 

\noindent So, we simply need to find the individual order of 6, 15, and 4 in order to determine the order of (6,15,4). Observe that: 
	\begin{align*}
		 6*5 \mod (30) &= 0 \\
		15*3 \mod (45) &= 0 \\ 
		4*6 \mod (24) &= 0  
	\end{align*}
So now that we have the order of each number in the tuple, the order of (6,15,4) is simply: 
	\begin{eqnarray}
		LCM(6,15,4) = 30 
	\end{eqnarray}
Thus the order of (6, 15, 4) is 30 by Corollary 9.18. \qed

\section*{9.3.32}
\textit{Prove that $U(5)\cong \mathbb{Z}_4$. Can you generalize this for U(p) where p is prime?}\\

\noindent Recall that $U(5) = \{1,2,3,4\}$ and $\mathbb{Z}_4 = \{0,1,2,3\}$. To prove that these two groups are isomorphic, we simply need to prove that U(5) is cyclic as both U(5) and $\mathbb{Z}_4$ have the same order (4).  
	\begin{align*}
		2 \mod(5) &= 2 \\ 
		2^2 \mod(5) &=  4 \\ 
		2^3 \mod(5) &= 8 \mod(5) = 3 \\ 
		2^4 \mod(5) &= 16 \mod(5) = 1 = e 
	\end{align*}
Thus 2 is a generator for U(5) and therefore U(5) is cyclic. U(5) is a cyclic group of order 4 and so by theorem 9.8, $U(5) \cong \mathbb{Z}_4$. \\ 

\noindent In order to extend this theorem to groups of the form $U(p)$ where p is prime, we need to be able to prove that $U(p)$ is cyclic so long a p is prime. First let's consider the type of elements in U(p). By definition this is all of the non-zero elements of $\mathbb{Z}_p$ that are relatively prime to p (i.e. k such that $gcd(k,p)=1$). Since p itself is prime it's only divisors are 1 and itself. Thus U(p) is necessarily all of the integers from 1 up to p-1: 
	\begin{align}
		U(p) = \{1,2,3...p-1\}
	\end{align}
For U(p) to be cyclic, it remains to find a generator for U(p). While in most cases there are multiple generators, the only option that has a clear chance of being the generator for every p is the element (p-1). Fermat's Little Theorem (6.19) gives us that: 
	\begin{equation}
		a^{p-1} \equiv 1 \mod(p) 
	\end{equation}



\end{document}
