\documentclass[a4paper, 11pt]{article}
\usepackage{geometry}
\geometry{letterpaper, margin=1in}
\usepackage{amsmath}
\usepackage{amssymb}  
\usepackage{amsthm}
\usepackage{ulem} 
\usepackage{graphicx}
\usepackage{enumitem} 
\graphicspath{ {images/} }


\newtheorem*{theorem}{Theorem}


\begin{document}
%Header-Make sure you update this information!!!!
\noindent
\large\textbf{Gauss-Bonnet (Local)} \hfill \textbf{John Waczak} \\
\normalsize MTH 435 \hfill  Date: \today \\
Dr. Christine Escher \\

\section*{6A}
\subsection*{b. Questions}
	\begin{enumerate}
		\item I think I followed this section pretty well although the details of the proof of the theorem were a little hairy. I am curious how the jump from signed angle to interior angle was made. \\ 
		\item To get rid of the $\int \kappa_g$, when using geodesics triangles is that because they have 0 geodesic curvature? 
		\item The Green's theorem that is used is just the one dimensional case of Stoke's theorem right? I was imagining $\oint \vec{F}\cdot d\vec{\ell} = \int (\nabla \times \vec{F}) dV$. 
	\end{enumerate}
\subsection*{c. Reflections}
		I think this section was pretty straight forward and reasonable. I will need to go through the proof of the local theorem slowly to make sure I understand what's going on. I was a little confused by the $X$ and $Y$ functions he used. 
\subsection*{d. Time}
I took roughly 0.5 hour(s) to read this section. 

\end{document}