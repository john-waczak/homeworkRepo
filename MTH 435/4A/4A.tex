\documentclass[a4paper, 11pt]{article}
\usepackage{geometry}
\geometry{letterpaper, margin=1in}
\usepackage{amsmath}
\usepackage{amssymb}  
\usepackage{amsthm}
\usepackage{ulem} 
\usepackage{graphicx}
\graphicspath{ {images/} }

\begin{document}
%Header-Make sure you update this information!!!!
\noindent
\large\textbf{Parallel Transport} \hfill \textbf{John Waczak} \\
\normalsize MTH 435 \hfill  Date: \today \\
Dr. Christine Escher \\

\section*{4A}
\subsection*{b. Questions}
	\begin{enumerate}
		\item I think I'm understanding this definition for the covariant derivative but I just want to clarify. The covariant derivative is just the projection to the tangent plane of S of the derivative of the restriction of a vector field on S to a curve $\gamma$ in S, correct? The definition for parallel then means that the component of the derivative of the velocity vector $v'$ in the tangent plane is 0. That signifies that the angle of the vector field with respect to the tangent plane of the surface does not change, right? 
		\item I am finding the third Algebraic Property on page 280 a little confusing. 
		\item Can you go over the definition of parallel transport? I'm having trouble parsing through that definition. 
		\item Can you go over Lemma 5.51 and 5.52? I'm having trouble following the derivation especially with the switching between covariant and regular derivatives. 
	\end{enumerate}
\subsection*{c. Reflections}
		Overall, I thought these sections were fine but I feel like I'm starting to get lost in notation. I think I need some practice with the Christoffel symbols and covariant derivatives so I can wrap my head about what kinds of maps all of these functions are. 
\subsection*{d. Time}
I took roughly 1 hour(s) to read this section. 

\end{document}