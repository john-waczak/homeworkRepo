\documentclass[a4paper, 11pt]{article}
\usepackage{geometry}
\geometry{letterpaper, margin=1in}
\usepackage{amsmath}
\usepackage{amssymb}  
\usepackage{amsthm}
\usepackage{ulem} 
\usepackage{graphicx}
\usepackage{enumitem} 
\graphicspath{ {images/} }


\newtheorem*{theorem}{Theorem}


\begin{document}
%Header-Make sure you update this information!!!!
\noindent
\large\textbf{Gauss-Bonnet (Global)} \hfill \textbf{John Waczak} \\
\normalsize MTH 435 \hfill  Date: \today \\
Dr. Christine Escher \\

\section*{6A}
\subsection*{b. Questions}
		I think I understand the section pretty well but I'm a little confused about the $\sum \alpha_i$ which the Tapp says is the sum of all vertices of all boundary components of R. I thought this was supposed to refer to external angle as in the local Gauss Bonnet so that was a bit confusing. Secondly, I think I followed the proof for how the geodesic and Gaussian curvatures were summed over the triangulation but again I had a bit of difficulty with how the $A$ section on page 350 was expanded.  
	
\subsection*{c. Reflections}
		I think this section is very interesting and I'm excited to try and put some triangulations on some surfaces. Since this is true for any Regular region of a Regular Surface will we get the same result if we take a triangulation over a small region as opposed to over the entire surface -- for example with the sphere could we just triangulate over say one octant and get the same answer as if we had triangulated the whole thing? 
\subsection*{d. Time}
I took roughly 1 hour(s) to read this section. 

\end{document}