\documentclass[a4paper, 11pt]{article}
\usepackage{geometry}
\geometry{letterpaper, margin=1in}
\usepackage{amsmath}
\usepackage{amssymb}  
\usepackage{amsthm}
\usepackage{ulem} 
\usepackage{graphicx}
\graphicspath{ {images/} }

\begin{document}
%Header-Make sure you update this information!!!!
\noindent
\large\textbf{Complete Surfaces} \hfill \textbf{John Waczak} \\
\normalsize MTH 435 \hfill  Date: \today \\
Dr. Christine Escher \\

\section*{5A}
\subsection*{b. Questions}
	\begin{enumerate}
		\item I am a little confused about how to use the (2) part of propositions/definition 5.34. Is this why the punctured plane is incomplete? I could have a geodesic heading towards the origin so when you extend the interval I to $\mathbb{R}$ you have a problem... 
		\item Can we go over all of the definitions for closed, compact, and complete surfaces? I'm getting a little lost in all of the definitions. 
	\end{enumerate}
\subsection*{c. Reflections}
		OVerall this section was pretty short and to the point. I think I will need to carefully review the proof of the final Hopf-Rinow theorem but it seems pretty intuitive given our discussion of geodesics so far. Of course this makes sense for our classic examples of the plane (where geodesics are just straight lines) and the sphere (great circles).
\subsection*{d. Time}
I took roughly 0.5 hour(s) to read this section. 

\end{document}