\documentclass[a4paper, 11pt]{article}
\usepackage{geometry}
\geometry{letterpaper, margin=1in}
\usepackage{amsmath}
\usepackage{amssymb}  
\usepackage{amsthm}
\usepackage{ulem} 
\usepackage{graphicx}
\graphicspath{ {images/} }

\begin{document}
%Header-Make sure you update this information!!!!
\noindent
\large\textbf{Exponential Map} \hfill \textbf{John Waczak} \\
\normalsize MTH 435 \hfill  Date: \today \\
Dr. Christine Escher \\

\section*{2A}
	\subsection*{b. Questions}
		\begin{itemize}
			\item	On page 258 in definition 5.13 why is the exponential map given with $\gamma_v(1)$? On page 249 (the example referenced below the definition) nearly the same map is given with $\gamma_v(t)$ instead. 

			\item Can you explain a little more why our normal neighborhood is restricted to $\epsilon \leq \pi$ for the cylinder? I would have thought the issue occurs once we've gone $2\pi$ all the way around the circumference of the cylinder. 
			
			\item Can you explain the pictures in figure 5.7 a little more. I think I kind of understand what's going on but I could use some clarification. 
			
			\item Does the exponential map always give longitudes or is that just because in figure 5.6 the author chose the top of the hemisphere? 
			
			\item Are $\mu, \sigma$ just surfaces patches for normal/polar coordinates made using the exponential map? 
			
			\item Can we go through some of the final proofs for this section? It seems like every single one of these relies on the results of some exercise we haven't completed yet.
		\end{itemize}	
	
	\section*{c. Reflections}
	There was a whole lot going on in this section and I think I'm definitely going to need some time to digest all of this. It was a little frustrating when trying to follow proofs of important properties like Propositions 5.20, 5.21, 5.24, etc. to find that they all required me to have finished some exercise already. It seemed like he did this more in this section than usual. I think we might need to take some extra time for this section. 
	
	\subsection*{d. Time}
	I took roughly 1.5 hours to read this section. 

\end{document}