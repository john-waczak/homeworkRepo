\documentclass[a4paper, 11pt]{article}
\usepackage{geometry}
\geometry{letterpaper, margin=1in}
\usepackage{amsmath}
\usepackage{amssymb}  
\usepackage{amsthm}
\usepackage{ulem} 
\usepackage{graphicx}
\graphicspath{ {images/} }

\begin{document}
%Header-Make sure you update this information!!!!
\noindent
\large\textbf{Exponential Map} \hfill \textbf{John Waczak} \\
\normalsize MTH 435 \hfill  Date: \today \\
Dr. Christine Escher \\

\subsection*{Tapp 5.21} 
	\textit{Prove Corollary 5.24. If $f:S\rightarrow \tilde{S}$ is an isometry between regular surfaces, and $\gamma$ is a geodesic in $S$, then $f\circ\gamma$ is a geodesic in $\tilde{S}$.}
	
\subsection*{Tapp 5.22}
	\textit{Explicitly describe the surface patch for normal polar coordinates when $S = S^2$, $p = (0,0,1)$, $e_1 = (1,0,0)$, and $e_2 = (0,1,0)$.}
	
\subsection*{Tapp 5.25}
	\textit{In Fig. 5.4 on page 252, the purple geodesic is asymptotic to the light-blue latitudinal curve.
		\begin{enumerate}
			\item Prove that a geodesic on a surface of revolution could not be asymptotic to a latitudinal curve unless the latitudinal curve is itself a geodesic.
			\item Rigorously justify the assertions in the discussion of Fig. 5.4
			\item Let $\gamma:\mathbb{R}\rightarrow S$ be a geodesic in the paraboloid 
				$$ S = \{(x,y,z)\in\mathbb{R}^3 | z = x^2 + y^2\} $$
			Prove that the height function, $h(x,y,z)=z$, composed with $\gamma$ has exactly one critical point $t_0$; it is decreasing on $(-\infty, t_0)$ and increasing on $(t_0, \infty)$. 
		\end{enumerate}
	} 
\end{document}