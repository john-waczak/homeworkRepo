\documentclass[usletter, 12pt]{article}
\usepackage{amsmath}
\usepackage{enumitem}
\usepackage{graphicx}
\graphicspath{ {images/} }


\begin{document}


\begin{enumerate}[leftmargin=0em, label=\textbf{\arabic*}.]
  \setcounter{enumi}{2} % remove this if you want to add other solutions 
  \item \textbf{Solution}:\\
    \begin{enumerate}[leftmargin=2em, label=(\textbf{\alph*})]
    \item We wish to calculate the necessary velocity changes $\Delta v$ and
      $\Delta v'$ in order to achieve the orbit shown in the above figure.
      Recall from class that the total energy of an orbiting body is related to
      the length of the semi-major axis by
      \begin{equation}
        E = -\frac{k}{2a}
      \end{equation}
      The total energy $E$ is also defined as the sum of the kinetic ($T$) and
      potential ($U(r)$) energies so that we must have
      \begin{equation}
        -\frac{k}{2a} = E = \frac{1}{2}mv_1^2-\frac{k}{R}
      \end{equation}
      Solving this equation for $v_1$, yields the total velocity of the
      satellite for the smaller circular orbit. That is,
      \begin{equation}
        v_1 = \sqrt{\frac{k}{mR}}
      \end{equation}
      The transfer orbit is an ellipse whose semi-major axis is determined from
      figure 1 to be
      \begin{equation}
        2a_t = R+R'
      \end{equation}
      We can now use this to solve for the speed of the satellite in the
      elliptical transfer orbit at the point where it changes from the green
      circular orbit. That is,
      \begin{equation}
        E_t = -\frac{k}{R+R'}= \frac{1}{2}m v_t^2-\frac{k}{R}
      \end{equation}
      Note that here we are using $R$ as distance from the sun in agreement
      with the green orbit. This results in a speed
      \begin{equation}
        v_{t1}=\sqrt{\frac{2k}{mR}\left( \frac{R'}{R+R'} \right)}
      \end{equation}
      Therefore, the necessary change in speed for the satellite to leave the
      circular orbit and enter the yellow elliptical transfer orbit is just 
      \begin{equation}
        \Delta v = v_{1t}-v_1
      \end{equation}

      Similarly, we can solve for $\Delta v'$ by finding the speed of the two
      orbits at the point where they overlap. They are
      \begin{align}
        v_2 = \sqrt{\frac{k}{mR'}}
        v_{t2} = \sqrt{\frac{2k}{mR'}\left( \frac{R}{R+R'} \right)}
      \end{align}
      The necessary change in velocity is therefore,
      \begin{equation}
        \Delta v' = v_2-v_{t2}
      \end{equation}
      
    \item The total time to make the transfer $T_t$ is a half-period of the
      transfer orbit. Using Kepler's third law, the time is found to be
      \begin{equation}
        T_t = \pi \sqrt{\frac{k}{m}}a_t^{3/2} = \pi\sqrt{\frac{k}{m}}\left( \frac{R+R'}{2} \right)^{3/2}
      \end{equation}
      
    \end{enumerate}
\end{enumerate}
\end{document}

