\documentclass[a4paper, 11pt]{article}
\usepackage{geometry}
\geometry{letterpaper, margin=1in}
\usepackage{graphicx}
\graphicspath{ {images/} }

\usepackage{amsmath}
\usepackage{amssymb}  
\usepackage{amsthm}
\usepackage{ulem}

\usepackage{enumitem}


\usepackage{pdfpages} % for including full pdf pages

\usepackage{empheq}

\usepackage{listings}
\usepackage{hyperref}

%format to allow bolded theorems, corollaries, etc...
\newtheorem*{theorem}{Theorem}
\newtheorem*{corollary}{Corollary}
\newtheorem*{lemma}{Lemma}
\newtheorem*{definition}{Definition}
\newtheorem*{Example}{Example} 
\newtheorem*{Remark}{Remark}

% stop typing \mathbb a thousand times 
\newcommand{\R}{\mathbb{R}}
\newcommand{\C}{\mathbb{C}}
\newcommand{\F}{\mathbb{F}}
\newcommand{\E}{\mathbb{E}}
\newcommand{\M}{\mathbb{M}}
\newcommand{\sphere}{\mathbb{S}}

% commands for bra-ket notation
\newcommand{\bra}[1]{\ensuremath{\left\langle#1\right|}}
\newcommand{\ket}[1]{\ensuremath{\left|#1\right\rangle}}
\newcommand{\bracket}[2]{\ensuremath{\left\langle #1 \middle| #2 \right\rangle}}
\newcommand{\matrixel}[3]{\ensuremath{\left\langle #1 \middle| #2 \middle| #3 \right\rangle}}
\newcommand{\expectation}[1]{\ensuremath{\left\langle #1 \right\rangle}}

% vector stuff
\newcommand{\basis}[1]{\hat{\mathbf{e}}_#1}
\newcommand{\unit}[1]{\hat{\boldsymbol{#1}}}
\newcommand{\bvec}[1]{\vec{\boldsymbol{#1}}}
\newcommand{\threevec}[2]{\begin{pmatrix} #1 \\ #2 \end{pmatrix}}

% change margins for solution
\newenvironment{solution}{%
	\begin{list}{}{%
			\setlength{\topsep}{0pt}%
			\setlength{\leftmargin}{0.5cm}%
			\setlength{\rightmargin}{0.5cm}%
			\setlength{\listparindent}{\parindent}%
			\setlength{\itemindent}{\parindent}%
			\setlength{\parsep}{\parskip}%
		}%
		\item[]}{\end{list}}




\begin{document}
\noindent
\begin{center}
  \centering
  \huge\textbf{Simplifying the Radial Equation} 
\end{center}
\par\noindent\rule{\textwidth}{0.4pt} \\\\


\noindent In the following activity, we will simplify the radial orbit equation
so that we can solve for the shape of the orbits.\\

\noindent Currently, the radial equation is
\begin{equation}
  \mu \ddot{r} = -\frac{d}{dr}V(r) + \frac{\ell^2}{\mu r^3}
\end{equation}


\begin{enumerate}[leftmargin=0em, label=\textbf{\arabic*}.]
\item Simplify $\dot{\phi}\frac{d}{d\phi}$
  \vspace{10em}
\item Using this expression, rewrite $\dot{r}$ as a derivative with respect
  to $\phi$.
  \vspace{10em}
\item Repeat the process, and re-express $\ddot{r}$ as derivatives with
  respect to $\phi$
  \vspace{10em}
\item Perform the variable substitution $r =\frac{1}{u}$ to re-expression $\dot{r}$ and $\ddot{r}$
  \vspace{10em}
\item Using $r=\frac{1}{u}$, find $\frac{d}{dr}$ in terms of $u$.
  \vspace{7em}
\item Use your expressions for $\ddot{r}$ and $\frac{d}{dr}$ to re-express the
  radial equation in terms of $u(\phi)$
  \vspace{10em}
\item Assuming a potential of the form $V(r)= -\frac{k}{r}$, solve your
  differential equation to find $u(\phi)$
  \vspace{20em}
\item Re-express your solution in terms of $r(\phi)$
  
  
  
  
\end{enumerate}




 
\end{document}






























