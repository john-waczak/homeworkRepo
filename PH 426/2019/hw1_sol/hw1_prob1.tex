\documentclass{article}
\usepackage{amsmath}

\begin{document}

\noindent\textbf{(a)}

\noindent \textit{Sensemaking:} Our final solution must be a polynomial of the
form
\begin{equation}
  f(z) = a_0 + a_1 (z-1) + a_2 (z-1)^2+ a_3 (z-1)^3+...
\end{equation}
where  we must specify the values for the coefficients $\{a_n\}$. If the
recurrence relation allows for 1 solution, we will include the first four non-zero
terms. If the recurrence relation indicates two possible solutions, we will
write out the first five non-zero terms for each such solution $f_0(z)$ and $f_1$(z)\vspace{3em}\\ 

\noindent\textit{Solution}: 
\noindent We wish to find the power series expansion centered around $z=1$ for a
differential equation whose recurrence relation is
\begin{equation}
  a_{n+1}=\frac{1}{n+1}a_n
\end{equation}
Because the recurrence relation does not skip indices ($n\to n+1$), knowing a single
coefficient enables us to use (2) to solve for \textbf{all other coefficients}.
This indicates that there is one possible solution. As per the instructions, we
must find the first four non-zero terms. Take $a_0$ to be the first coefficient.
Then, equation (2) gives:

\begin{align}
  n=0 &:\qquad a_1 = a_0= \frac{1}{1!}a_0 \\
  n=1 &:\qquad a_2 = \frac{1}{2}a_1 = \frac{1}{2!}a_0 \\
  n=2 &:\qquad a_3 = \frac{1}{3}a_2 = \frac{1}{3!}a_0
\end{align}

\begin{align}
  f(z) &\approx a_0 + \frac{a_0}{1!}(z-1)+\frac{a_0}{2!}(z-1)^2+\frac{1}{3!}(z-1)^3 \nonumber\\
  &= a_0\Bigg(1+(z-1)+\frac{(z-1)^2}{2!}+\frac{(z-1)^3}{3!}\Bigg)
\end{align}
In the final line, I have factored out the $a_0$ to make it clear that this
value will be determined by our initial conditions. Take a good look inside the
parenthesis in (6). Do you recognize which function this power series is for? \\


\noindent\textbf{(b)}\\

\noindent \textit{Sensemaking:} Same as (a) except the expansion will be centered
around zero. That is,
\begin{equation}
  f(z) = a_0+a_1z+a_2z^2+a_3z^3+ ...
\end{equation}
\\

\noindent For this problem, we wish to expand our solution around the point
$z=0$ for a differential equation that results in a recurrence relation given by
\begin{equation}
  a_{n+2} = -\frac{(5-n)(6+n)}{(n+2)(n+1)}a_n
\end{equation}
Because this recurrence relation \textit{does} skip indices ($n\to n+2$), we
expect two possible solutions. The first solution comes from starting with
$a_0$. The second comes from starting with $a_1$. We must calculate 5 non-zero
terms for each of these scenarios. Equation (8) gives
\begin{align}
  &\qquad a_0 = a_0 \\
  &\qquad a_1 = a_1 \\
  n=0 :&\qquad a_2=-15a_0 \\
  n=1 :&\qquad a_3= -\frac{14}{3}a_1\\
  n=2 :&\qquad a_4=-2a_2 = 30a_0\\
  n=3 :&\qquad a_5=-\frac{9}{10}a_3= \frac{21}{5}a_1 \\
  n=4 :&\qquad a_6=-\frac{1}{3}a_4 = -10a_0\\
  n=5 :&\qquad a_7=0 \\
  n=6 :&\qquad a_8=\frac{3}{14} = -\frac{15}{7}a_0
\end{align}
Interestingly, the series involving $a_1$ terminates after $n=3$. This comes
from the $(5-n)$ term in the numerator of (8). Putting these together, the
general solution is
\begin{equation}
  f(z) = a_0\left(1-15z^2+30z^4-10z^6-\frac{15}{7}z^8+...\right) + a_1\left(z-\frac{14}{3}z^3+\frac{21}{5}z^5\right)
\end{equation}
Note that the solution involving $a_1$ is not a power series. We started by
assuming a power series solution and have found a polynomial with a finite
number of terms that exactly solves the differential equation. \\






\noindent\textbf{(c)}\\
\noindent \textit{Sensemaking:} Same as (a) except the expansion will be centered
around zero. That is,
\begin{equation}
  f(z) = a_0+a_1z+a_2z^2+a_3z^3+ ...
\end{equation}
\\
We are faced with a similar scenario to part(b). We wish to write the power
series solution centered around $z=0$ given the recurrence relation
\begin{equation}
  a_{n+2}= -\frac{(3-n)}{(n+2)(n+1)}a_n
\end{equation}

This recurrence relation skips indices ($n\to n+2$) so we expect two
distinct solutions for $a_0$ and $a_1$. Using the relation, gives

\begin{align}
  &\qquad a_0 = a_0 \\
  &\qquad a_1 = a_1 \\
  n=0 :&\qquad a_2=-\frac{3}{2}a_0 \\
  n=1 :&\qquad a_3= -\frac{1}{3}a_1\\
  n=2 :&\qquad a_4=-\frac{1}{2}a_2=\frac{3}{4}a_0\\
  n=3 :&\qquad a_5=0 \\
  n=4 :&\qquad a_6=\frac{1}{30}a_4 = \frac{1}{40}a_0\\
  n=5 :&\qquad a_7=0 \\
  n=6 :&\qquad a_8=\frac{3}{56}a_6 = \frac{3}{2240}a_0
\end{align}
so that the general solution to the differential equation is
\begin{equation}
  f(z)= a_0\left(1-\frac{3}{2}z^2+\frac{3}{4}z^4+\frac{1}{40}z^6+\frac{3}{2240}z^8+...\right) + a_1\left(z-\frac{1}{3}z^3\right)
\end{equation}
Again, the series involving $a_1$ terminated! 












\end{document}






























