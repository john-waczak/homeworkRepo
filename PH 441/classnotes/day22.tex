\documentclass[a4paper, 11pt]{article}
\usepackage{geometry}
\geometry{letterpaper, margin=1in}
\usepackage{amsmath}
\usepackage{amssymb}  
\usepackage{amsthm}
\usepackage{ulem} 
\usepackage{graphicx}
\usepackage{cancel} 
\usepackage{enumitem} 
\graphicspath{ {images/} }


\newtheorem*{theorem}{Theorem}
\newtheorem*{corollary}{Corollary}
\newtheorem*{lemma}{Lemma}
\newtheorem*{definition}{Definition}

\begin{document}
	%Header-Make sure you update this information!!!!
	\noindent
	\large\textbf{Thermal Physics - PH441} \hfill \textbf{John Waczak} \\
	\normalsize Day 22 \hfill  Date: \today \\
	
\subsection*{Heat capacity for Fermi-gas} 
	Recall that the heat capacity is given by $C_V = \big(\frac{\partial U}{\partial T}\big)_V$. Thus we can find this using our density of states
		\begin{align*}
			U &= \int D(\varepsilon)\varepsilon f(\varepsilon)d\varepsilon \\ 
			\Rightarrow C_V &= \int D(\varepsilon)\varepsilon \Big(\frac{\partial f(\varepsilon)}{\partial T}\Big)_\varepsilon d\varepsilon \\ 
			f(\varepsilon) &= \frac{1}{e^(\varepsilon-\mu)/kT + 1} \\ 
			\frac{\partial f}{\partial T} &= \frac{-e^{\beta(\varepsilon-\mu)}}{(e^{\beta(\varepsilon-\mu)}+1)^2}\Big[\frac{(\varepsilon-\mu)}{kT^2}-\beta\frac{\partial \mu}{\partial T}\Big]
		\end{align*}
	where the $\partial_T \mu$ term goes to zero if we assume that $\mu$ isn't changing much with temperature. Thus we have $\mu=\varepsilon_f$ as well. So,  
		\begin{align*}
			\partial_T f(\varepsilon) &= \frac{1}{\Big(e^{\beta(\varepsilon-\varepsilon_f)}+1\Big)\Big(e^{-\beta(\varepsilon-\varepsilon_f)}+1\Big)}\frac{\varepsilon-\varepsilon_f}{kT^2}
		\end{align*}
\end{document}