\documentclass[a4paper, 11pt]{article}
\usepackage{geometry}
\geometry{letterpaper, margin=1in}
\usepackage{amsmath}
\usepackage{amssymb}  
\usepackage{amsthm}
\usepackage{ulem} 
\usepackage{graphicx}
\graphicspath{ {images/} }
\usepackage{tikz} 

\newcommand{\dbar}{{\mkern3mu\mathchar'26\mkern-12mu d}} 


\begin{document}
%Header-Make sure you update this information!!!!
\noindent
\large\textbf{Thermal Physics - PH441} \hfill \textbf{John Waczak} \\
\normalsize Day 8\hfill  Date: \today \\

\subsection*{Free Energy}
	From Monday we have that $U = \sum\limits_\mu P_\mu E_\mu = - \Big(\frac{\partial \ln(Z)}{\partial\beta}\Big)_V$. Here is another way to solve for the Free Energy. 
		\begin{align*}
			\Rightarrow d\ln Z &= -U d\beta + \frac{\partial \ln Z}{\partial V}dV \\
			\text{let } \xi &= \frac{\partial \ln Z}{\partial V} \\ 
			d(U\beta) &= Ud\beta + \beta dU \\ 
			\Rightarrow -Ud\beta &= \beta dU - d(U\beta) \\ 
			d\ln Z &= \beta dU - d(U\beta) + \xi dV \\ 
			dU &= \frac{1}{\beta} d(\ln Z - U\beta) + \xi dV \\ 
				&= TdS - pdV = Td(k(\ln Z - U\beta)) + \xi dV \\ 
			\Rightarrow \frac{S}{k} &= \ln Z - \frac{U}{kT} \\ 
			-kT\ln Z &= U - TS = F \\ 
		\end{align*}
  
	  \noindent Question: What is F if g $\mu$states all with energy $E_0$? 
		  \begin{align*}
			  Z &= \sum\limits_\mu e^{-\beta E_\mu} = ge^{-\beta E_0} \\ 
			  F &= -kT\ln(Z) =-kT\ln(ge^{-\beta E_0}) = E_0 - kT\ln(g)
		  \end{align*}
	  
	  \noindent Note: $F=-kT\ln Z$ is a fine place \textit{to start} as opposed to beginning with $U - - \frac{\partial \ln Z}{\partial \beta}$. \\ 
	  
	  \noindent Turning the equation around we have that $Z = e^{-\beta F}$ which can be very useful. 

\subsection*{Pressure} 
	Recall that $p = -\frac{\partial U}{\partial V}$ at fixed S from the thermodynamic identity. \\ 
	
	\noindent \textit{Question:} How do I fix the entropy?\\
	 
		\noindent Recall that $dU = \dbar Q - \dbar W $ Thus, if we thermally isolate the system, $\dbar Q = 0 \Rightarrow dS = 0$ since $T\neq 0$. Another way to think about it is from $S=-k\sum\limits_\mu P_\mu \ln P_\mu $. So if we don't let the probabilities change then $dS = 0$. \\
	  
	 Thus, now we have: 
		 \begin{align*}
			 p &= -\frac{\partial U}{\partial V} = -\sum\limits_\mu P_\mu \frac{dE_\mu}{dV} \quad \text{since probabilities are fixed} 
		 \end{align*}
		 
\subsection*{What do we do with F?} 
	We have two important definitions: 
		\begin{enumerate}
			\item $F = -kT\ln Z $ 
			\item $F = U - TS \Rightarrow dF = -SdT -pdV$ 
		\end{enumerate}
	This allows us to say that $p = -\frac{\partial F}{\partial V}$ and $S = -\frac{\partial F}{\partial T}$ \\
	
	\noindent What is the physics meaning of the Free-energy? Usually you talk about keeping either T or V fixed. For example if you keep the temperature fixed, then F is equal to the work i.e. \textit{the Helmholtz Free energy describes the available work}. Another thing you can say is that it is the energy that is naturally a function of temperature and volume. The internal energy U given by $dU = TdS - pdV$ is naturally a function of entropy and volume. 
\end{document}





























