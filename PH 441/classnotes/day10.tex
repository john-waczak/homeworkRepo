\documentclass[a4paper, 11pt]{article}
\usepackage{geometry}
\geometry{letterpaper, margin=1in}
\usepackage{amsmath}
\usepackage{amssymb}  
\usepackage{amsthm}
\usepackage{ulem} 
\usepackage{graphicx}
\graphicspath{ {images/} }
\usepackage{tikz} 
\begin{document}
%Header-Make sure you update this information!!!!
\noindent
\large\textbf{Thermal Physics - PH441} \hfill \textbf{John Waczak} \\
\normalsize Day 10 \hfill  Date: \today \\

\subsection*{Simple harmonic oscillator(s)}	
Recall the that the energy of a single, simple harmonic oscillator is given by $E_n = (n+\frac{1}{2})\hbar\omega$. Typically to solve for this kind of thing we look for normal modes in a differential equation with specified boundary conditions. Normal modes are nice because we can view them as non-interacting.	\\ 

\noindent First, let's find our partition function: 
	\begin{align*}
		Z 	&= \sum\limits_{n=0}^\infty e^{-\beta(n+1/2)\hbar\omega} \\ 
			&= e^{-\beta\hbar\omega/2}\sum\limits_{n=0}^\infty e^{-\beta \hbar \omega n}\\
		\text{let } \xi &= e^{-\beta\hbar\omega} \\ 
		\text{let } \Xi &= \sum\limits_{n=0}^\infty e^{-\beta\hbar\omega n}\\
			\Xi &= \sum\limits_{n=0}^\infty \xi^n \\ 
		\text{Note: this is}& \text{ the geometric series} \\ 
			\Xi &= \frac{1}{1-\xi} \\ 
		\Rightarrow Z &= \frac{e^{-\beta\hbar\omega/2}}{1-e^{-\beta \hbar \omega}}
	\end{align*}
	
\noindent Now that we have our partition function $Z$ the next logical thing is to calculate the Helmholtz free energy.
	\begin{align*}
		F &= -kT\ln Z \\ 
			&= -kT \ln \Big(\frac{e^{-\beta\hbar\omega/2}}{1-e^{-\beta \hbar \omega}}\Big) \\ 
			&= \frac{kT\beta\hbar\omega}{2}+kT\ln (1-e^{-\beta\hbar\omega}) \\
				&= \frac{\hbar\omega}{2}+kT\ln (1-e^{-\beta\hbar\omega})
	\end{align*}
	
\noindent Now that we have F we can find the entropy as usual. 
	\begin{align*}
		S 	&= -\Big(\frac{\partial F}{\partial T}\Big)_V\\
			&= -k\ln(1-e^{-\beta\hbar\omega})-\frac{kT(-e^{-\beta\hbar\omega})}{1-e^{-\beta\hbar\omega}}\Big(\frac{-\hbar\omega}{kT}\Big)\\
			&= \frac{\hbar\omega}{T}\frac{e^{-\beta\hbar\omega}}{1-e^{-\beta\hbar\omega}}-k\ln(1-e^{-\beta\hbar\omega})
	\end{align*}
	
\noindent Now recall that $U = \langle E \rangle = \langle(n+1/2)\hbar\omega\rangle =(\langle n\rangle + \frac{1}{2})\hbar\omega$. Considering this, solve for U. 
	\begin{align*}
		U 	&= F + TS \\ 
			&= \frac{\hbar\omega}{2}+kT\ln(1-e^{-\beta\hbar\omega}) + \hbar\omega\frac{e^{-\beta\hbar\omega}}{1-e^{-\beta\hbar\omega}}-kT\ln(1-e^{-\beta\hbar\omega}) \\ 
			&= \frac{\hbar\omega}{2}+\hbar\omega\frac{e^{-\beta\hbar\omega}}{1-e^{-\beta\hbar\omega}} \\ 
		\text{Furthermore: } \langle n \rangle &= \frac{e^{-\beta\hbar\omega}}{1-e^{-\beta\hbar\omega}} = \frac{1}{e^{\beta\hbar\omega}-1}
	\end{align*}
	
\noindent Now let's consider the high and low temperature limits of this
	\begin{align*}
		\text{High temp} &\Rightarrow \beta\hbar\omega << 1 \\ 
		\langle n \rangle &= (e^{\beta\hbar\omega}-1)^{-1} \\ 
			&= \frac{1}{1+\beta\hbar\omega + ... - 1} \\ 
			&= \frac{kT}{\hbar\omega}\quad \text{"Equipartition result"}\\
		\text{Low temp} &\Rightarrow \beta\hbar\omega >> 1 \\ 
		\langle n \rangle &= \frac{e^{-\beta\hbar\omega}}{1-e^{-\beta\hbar\omega}} \\ 
			&\approx e^{-\beta\hbar\omega}(1+e^{-\beta\hbar\omega}) \quad \text{using } (1+z)^p \\ 
			&= e^{-\beta\hbar\omega}+e^{-2\beta\hbar\omega}
	\end{align*}
	
	
	
	
	
	
	
	
	
	
	
	
	
	
	
	
	
	
	
\end{document}