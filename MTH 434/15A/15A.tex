\documentclass[a4paper, 11pt]{article}
\usepackage{geometry}
\geometry{letterpaper, margin=1in}
\usepackage{amsmath}
\usepackage{amssymb}  
\usepackage{amsthm}
\usepackage{ulem} 
\usepackage{graphicx}
\graphicspath{ {images/} }

\begin{document}
%Header-Make sure you update this information!!!!
\noindent
\large\textbf{Normal Curvature} \hfill \textbf{John Waczak} \\
\normalsize MTH 434 \hfill  Date: \today \\
Dr. Christine Escher \\

\section*{15A}
	\subsection*{b. Questions}
	I am still confused as to the difference between the Gaussian curvature and the mean curvature. I guess, more specifically, what is the motivation for the difference between these two and how do they relate to the surface visually. I'm still trying to connect to the idea of curve-curvature and the osculating plane/circle. Also I was a little confused at the $|v|=1$ requirement that kept popping up. For example why is it that in part (4) of definition 4.9 that we need $|v|=1$ in order to define the normal curvature? 
	
	\section*{c. Reflections}
	I think this section followed pretty clearly from the 4.1 reading. My big takeaway was that if we want to find the the Gaussian curvature all we need to do is find an orientation, construct the Weingarten map and then diagonalize it. That seemed more or less pretty straight forward to me. 
	\subsection*{d. Time}
	I took roughly 30 minutes to reread this section. 

\end{document}