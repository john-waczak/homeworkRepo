\documentclass[a4paper, 11pt]{article}
\usepackage{geometry}
\geometry{letterpaper, margin=1in}
\usepackage{amsmath}
\usepackage{amssymb}  
\usepackage{amsthm}
\usepackage{ulem} 
\usepackage{graphicx}
\graphicspath{ {images/} }

\begin{document}
%Header-Make sure you update this information!!!!
\noindent
\large\textbf{Gaussian Curvature Again} \hfill \textbf{John Waczak} \\
\normalsize MTH 434 \hfill  Date: \today \\
Dr. Christine Escher \\

\section*{18A}
	\subsection*{b. Questions}
	After rereading this section again the only question I still have is for the explanation of the loops in figure 4.10. I think they are just using the sign Gaussian Curvature to determine if the map reverses or preserves the orientation of the loop. Is that correct?  
	
	\section*{c. Reflections}
	I found looking at proposition 4.14 a second time was really interesting as I think I have a better mental picture of what's going on with the intersection of the neighborhoods of p with TpS now. 
	\subsection*{d. Time}
	I took roughly 30 minutes to reread this section. 

\end{document}