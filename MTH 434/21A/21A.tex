\documentclass[a4paper, 11pt]{article}
\usepackage{geometry}
\geometry{letterpaper, margin=1in}
\usepackage{amsmath}
\usepackage{amssymb}  
\usepackage{amsthm}
\usepackage{ulem} 
\usepackage{graphicx}
\graphicspath{ {images/} }

\begin{document}
%Header-Make sure you update this information!!!!
\noindent
\large\textbf{Minimal Surfaces} \hfill \textbf{John Waczak} \\
\normalsize MTH 434 \hfill  Date: \today \\
Dr. Christine Escher \\

\section*{21A}
	\subsection*{b. Questions}
	 
	 My first question is sort of related--- Is there any way to describe these minimal surface problems with some kind of variational principle? The section talked a lot about soap films which from the "surface tension" perspective I'm sure obey some kind of calculus of variations law like in Lagrangian dynamics. It seems like we are really close to something along these lines by defining minimum surfaces in terms of having zero mean curvature anywhere. \\
	 
	 Beyond that, I think I understood the section pretty well, I had some issues visualizing what was going on with proposition 4.22 but I think after staring at figure 4.16 for a while I have a better understanding of what's going on. Is there a way to use the mean curvature field to construct a map that takes a non-minimal surface to a minimal surface? The examples showed how we can check if a surface is a minimal surface but can we always make a map that takes us from the former to the latter?  Is there anything special about classes of surfaces that would all be sent to the same minimal surface under such a map? 
	
	\section*{c. Reflections}
	I really enjoyed this section. In particular,  I found the soap film idea really interesting and am curious how we can use minimal surfaces to classify sets of surfaces.
	
	\subsection*{d. Time}
	I took roughly 30 minutes to read this section. 

\end{document}