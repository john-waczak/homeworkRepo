\documentclass[a4paper, 11pt]{article}
\usepackage{geometry}
\geometry{letterpaper, margin=1in}
\usepackage{amsmath}
\usepackage{amssymb}  
\usepackage{amsthm}
\usepackage{ulem} 
\usepackage{graphicx}
\graphicspath{ {images/} }

\begin{document}
%Header-Make sure you update this information!!!!
\noindent
\large\textbf{Parametrized Curves} \hfill \textbf{John Waczak} \\
\normalsize MTH 434 \hfill  Date: \today \\
Dr. Christine Escher \\

\section*{1A}
	\subsection*{b. Questions}
	Overall, this section was very much a review of prior knowledge from calculus and linear algebra. There were a couple of sections I had to read closely and reread a few times, in particular: 
		\begin{enumerate}
			\item \textit{Proposition and Definition 1.13}
			It took me a while to realize what was meant by writing $\mathbf{x} = \mathbf{x}^{\parallel}+\mathbf{x}^{\perp}$. Clearly the parallel component is the projection but for some reason when discussing vectors in $\mathbb{R}^n$ I was expecting to see n components even though this was simply suggesting that any vector $\mathbf{x}$ can be decomposed into a sum of parallel and perpendicular vectors with reference to as second vector $\mathbf{y}$. 
			
			\item \textit{Lemma 1.12 Schwarz Inequality}
			I was expecting to see the more familiar expression $|\langle u, v\rangle|^2 \leq \langle u,u \rangle \cdot \langle v,v \rangle$. I think I would have been less bothered had the book's definition used the double bar $||\mathbf{x}||$ to mean norm because it's weird to think of the norm of an inner product which returns a scalar (although I guess the norm of a scalar $\alpha$ is simply $\sqrt{\alpha^2}$ which is a definition for absolute value...)
			
			\item \textit{Example 1.9 Shortest path between two points}
			It took me a few rereads to think about why it is sufficient to show $L \ge d$ where $d$ is the straight line distance and not $L > d$. 
			
			\item \textit{Inner product}
			Something I always find interesting is using the inner product to define \textit{angle} in higher dimensional space. 
		\end{enumerate}
	
	\subsection*{c. Reflections}
	I first read closely through the section in order to familiarize myself with the material. As I read I circled things I had questions about with pencil. I took note of the examples but didn't work them out in detail and paid close attention to the included proofs. Afterwards I went back through with a highlighter and marked parts I thought were important that weren't already bolded or boxed in the text. 
	
	
	\subsection*{d. Time}
	It took me approximately an hour and a half to read through the section thoroughly and another half hour to go back and highlight what I thought to be important / interesting information. 

\end{document}

