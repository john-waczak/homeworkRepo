\documentclass[a4paper, 11pt]{article}
\usepackage{geometry}
\geometry{letterpaper, margin=1in}
\usepackage{amsmath}
\usepackage{amssymb}  
\usepackage{amsthm}
\usepackage{ulem} 
\usepackage{graphicx}
\graphicspath{ {images/} }

\begin{document}
%Header-Make sure you update this information!!!!
\noindent
\large\textbf{Isometries} \hfill \textbf{John Waczak} \\
\normalsize MTH 434 \hfill  Date: \today \\
Dr. Christine Escher \\

\section*{11A}
	\subsection*{b. Questions}
	I thought this section was pretty straight forward. I had not seen the alternative definition of the cross product, $|x\times y| = \sqrt{|x|^2|y|^2-\langle x,y\rangle^2}$, before so that was cool and I think the idea of intrinsics and forms is pretty interesting. I was a little confused by the example given for the cylinder surface patch. I don't see why the final lines follows from the fact that $d\sigma_q$ preserves orthonormal bases. 
	
	\section*{c. Reflections}
	I am curious what the precise definition of form is and what other kinds of forms exist other than the norm-squared. I particularly found the last section on the volume of interior interesting as I also initially expected the volume might be intrinsic. 
	
	
	\subsection*{d. Time}
	I took roughly 30 minutes to reread this section. 

\end{document}