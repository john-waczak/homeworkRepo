\documentclass[a4paper, 11pt]{article}
\usepackage{geometry}
\geometry{letterpaper, margin=1in}
\usepackage{amsmath}
\usepackage{amssymb}  
\usepackage{amsthm}
\usepackage{ulem} 
\usepackage{graphicx}
\graphicspath{ {images/} }

\begin{document}
%Header-Make sure you update this information!!!!
\noindent
\large\textbf{Curvature} \hfill \textbf{John Waczak} \\
\normalsize MTH 434 \hfill  Date: \today \\
Dr. Christine Escher \\

\section*{2A}
	\subsection*{b. Questions}
	The only part I found confusing in this section was in the derivation of the formula for curvature. On page 25 the author begins by analyzing the reparametrization of the curve $\tilde{\gamma}$. I see how they used the chain and product rules cited in the first two lines but on the third, the author writes:
		\begin{align*}
			\tilde{\mathbf{a}}^{\perp} &= 0 + \phi'(t)^2\mathbf{a}^{\perp}(\phi(t))
		\end{align*}
	I think this step comes from the fact that the velocity vector must point in the direction of the curve and so does not contribute to $\tilde{\mathbf{a}}^{\perp}$. Then the only part left is that $\tilde{\mathbf{a}}=\phi'(t)^2\mathbf{a}$ and so again, we only care about the perpendicular component leaving us with the above equation that's later used in the derivation of curvature. 
	\subsection*{c. Reflections}
	This was a short section so I chose to read through it once without taking notes to get familiar with the material. Then I reread and highlighted important parts and wrote down questions I had. I found the geometric interpretation of fitting a circle to the curve as a way to think about curvature very helpful. 
	
	
	\subsection*{d. Time}
	It took me approximately half an hour to read through the section two times and take note of everything I thought was important.

\end{document}
