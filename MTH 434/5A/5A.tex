\documentclass[a4paper, 11pt]{article}
\usepackage{geometry}
\geometry{letterpaper, margin=1in}
\usepackage{amsmath}
\usepackage{amssymb}  
\usepackage{amsthm}
\usepackage{ulem} 
\usepackage{graphicx}
\graphicspath{ {images/} }

\begin{document}
%Header-Make sure you update this information!!!!
\noindent
\large\textbf{Curvature} \hfill \textbf{John Waczak} \\
\normalsize MTH 434 \hfill  Date: \today \\
Dr. Christine Escher \\

\section*{4A}
	\subsection*{b. Questions}
	I didn't think this section was too confusing. I think the fundamental theorems make great visual sense because if we are given a curvature and torsion for a space curve, surely moving it by a rigid motion should not affect the properties of that curve. I imagine this will be useful because if we can show that a curve is of a certain form (circle, helix, parabola, etc...) we now can say that the coordinate axes are arbitrary and we are always allowed to redefine them so long as the curvature and torsion are unaffected. 
	\subsection*{c. Reflections}
	As I accidentally read the entire section for part 4A, I was able to pay a bit closer attention to the details the second time around. 
	
	\subsection*{d. Time}
	I took roughly 10 minutes to read this section. 

\end{document}