\documentclass[a4paper, 11pt]{article}
\usepackage{geometry}
\geometry{letterpaper, margin=1in}
\usepackage{amsmath}
\usepackage{amssymb}  
\usepackage{amsthm}
\usepackage{ulem} 
\usepackage{graphicx}
\graphicspath{ {images/} }

\begin{document}
%Header-Make sure you update this information!!!!
\noindent
\large\textbf{Surface Area} \hfill \textbf{John Waczak} \\
\normalsize MTH 434 \hfill  Date: \today \\
Dr. Christine Escher \\

\section*{17A}
	\subsection*{b. Questions}
	The only question I can think of at the moment is regarding the example of the area of as sphere. He mentions some technicalities of dealing with setting the bounds and excluding the $\epsilon$ argument. I am a little confused as to why that's necessary or I guess when it would actually be necessary to make that limit argument when performing these integrals. Can you explain the technical detail that's happening here? 
	
	\section*{c. Reflections}
	I think I get the idea of this section. My understanding is that they are saying that we can figure out how an area in S is distorted by the diffeomorphism f by computing this area distortion since $||d\sigma||$ gives the approximate ratio of the new area to the area in the preimage of the coordinate chart. This makes intuitive sense as the length of a cross product vector gives the area of the parallelogram formed by the original two vectors so if we find this area distortion we are essentially figuring out how an infinitesimal area changes under f.  
	\subsection*{d. Time}
	I took roughly 30 minutes to reread this section. 

\end{document}