\documentclass[a4paper, 11pt]{article}
\usepackage{geometry}
\geometry{letterpaper, margin=1in}
\usepackage{amsmath}
\usepackage{amssymb}  
\usepackage{amsthm}
\usepackage{ulem} 
\usepackage{graphicx}
\graphicspath{ {images/} }

\begin{document}
%Header-Make sure you update this information!!!!
\noindent
\large\textbf{Gauss Map} \hfill \textbf{John Waczak} \\
\normalsize MTH 434 \hfill  Date: \today \\
Dr. Christine Escher \\

\section*{14A}
	\subsection*{b. Questions}
	This section seemed pretty straight forward and made a good analogy to the discussion of the curvature of curves. I had a question about normal fields that I didn't think about during the reading for that section. Tapp mentions that we can think of the unit Normal field as a map from the surface to the unit sphere S2. Would it be reasonable then to suppose that the image N(S) is the entire unit sphere S2 iff S is a closed surface? That could be interesting to try and prove. For the Weingarten map example for the sphere on page 198, am I reading this correctly that the book is getting the Gaussian curvature to be $\frac{1}{r^2}$ because the determinant of a scalar multiplied by an $n\times n$ matrix is the scalar to the nth power times the determinant of the matrix? Lastly, is there some way to relate the Gaussian curvature to the radius of the sphere that best fits the surface at that point? I was trying to think of this like we did for the curvature of a curve relating to the osculating circle.  
	
	\section*{c. Reflections}
	I'm curious about how we are going to relate this Gaussian curvature to the local fundamental forms we developed in the previous section. 
	
	\subsection*{d. Time}
	I took roughly 30 minutes to reread this section. 

\end{document}