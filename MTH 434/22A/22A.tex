\documentclass[a4paper, 11pt]{article}
\usepackage{geometry}
\geometry{letterpaper, margin=1in}
\usepackage{amsmath}
\usepackage{amssymb}  
\usepackage{amsthm}
\usepackage{ulem} 
\usepackage{graphicx}
\graphicspath{ {images/} }

\begin{document}
%Header-Make sure you update this information!!!!
\noindent
\large\textbf{Minimal Surfaces} \hfill \textbf{John Waczak} \\
\normalsize MTH 434 \hfill  Date: \today \\
Dr. Christine Escher \\

\section*{22A}
	\subsection*{b. Questions}
	 
	 Am I correct that the $S_t = \{p + t\phi(p)N(p) | p \in S\}$ is the surface created by shifting every point by some factor $\phi(p)$ in the normal direction? I guess I'm having a little trouble understanding proposition 4.22. Otherwise I think I understand this section pretty well. It seems like all of the examples relied on showing the surface patches were conformal. Is this our only tool to really test these surfaces to find if they are minimal? 
	
	\section*{c. Reflections}
	I feel about the same as when I first read this section. I'm sure I will have more questions once I've done some sample problems and discussed the section in class.
	
	
	\subsection*{d. Time}
	I took roughly 30 minutes to read this section. 

\end{document}