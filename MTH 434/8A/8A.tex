\documentclass[a4paper, 11pt]{article}
\usepackage{geometry}
\geometry{letterpaper, margin=1in}
\usepackage{amsmath}
\usepackage{amssymb}  
\usepackage{amsthm}
\usepackage{ulem} 
\usepackage{graphicx}
\graphicspath{ {images/} }

\begin{document}
%Header-Make sure you update this information!!!!
\noindent
\large\textbf{Regular Surfaces} \hfill \textbf{John Waczak} \\
\normalsize MTH 434 \hfill  Date: \today \\
Dr. Christine Escher \\

\section*{4A}
	\subsection*{b. Questions}
	After rereading this section I found I had some more questions I did not previously address in HW 7A. Particularly when discussing multiple examples such as the cylinder and sphere, Tapp examines the functions $\sigma(u,v)\to (x(u,v), y(u,v), z(u,v))$ which to my knowledge are candidates for surface patches and says they cover the object up to some set of points. For example, with the cylinder, he says that the function $\sigma(u,v) = (\cos(u), \sin(u), v)$ is only a valid surface patch for $x\neq -1$. I do not understand where this comes from but I am imagining it has something to do with the "degeneracy" mentioned earlier in the chapter. //
	
	\noindent My second question is more of a request. Near the end of the chapter, Tapp defines a regular value of f. We now have a regular curve, regular surface, and a regular value. Can you clarify the differences and similarities between these ideas? 
	
	\section*{c. Reflections}
	Once again I personally found the information regarding the Jacobian to be the most interesting. In this chapter the discussion of how having a Jacobian of Rank 2 guarantees that two linearly independent vectors in $\mathbb{R}^2$ map to two linearly independent vectors in $\mathbb{R}^3$ was very neat.  
	\subsection*{d. Time}
	I took roughly 30 minutes to reread this section. 

\end{document}