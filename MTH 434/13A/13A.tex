\documentclass[a4paper, 11pt]{article}
\usepackage{geometry}
\geometry{letterpaper, margin=1in}
\usepackage{amsmath}
\usepackage{amssymb}  
\usepackage{amsthm}
\usepackage{ulem} 
\usepackage{graphicx}
\graphicspath{ {images/} }

\begin{document}
	%Header-Make sure you update this information!!!!
	\noindent
	\large\textbf{First Fundamental Form} \hfill \textbf{John Waczak} \\
	\normalsize MTH 434 \hfill  Date: \today \\
	Dr. Christine Escher \\
	
\section*{13A}
	\subsection*{b. Questions}
	I think that this section followed pretty well from what we did in class on Isometries. Is there a way to somehow relate the first fundamental form in local coordinates to Isometries? Would that just be an alternative definition of a local Isometry? Also I was wondering if there is any particular reason for the E,F,G notation or were those letters just chosen arbitrarily from the alphabet? When we talk about local properties in Geometry are we typically thinking about differentials? 
	
	\section*{c. Reflections}
	This section was really interesting!  I particularly found the very last paragraph that referenced fig 3.11 neat as it really showed how these functions encode information about the surfaces. 
	
	
	\subsection*{d. Time}
	I took roughly 30 minutes to reread this section. 
	
\end{document}