\documentclass[a4paper, 11pt]{article}
\usepackage{geometry}
\geometry{letterpaper, margin=1in}
\usepackage{amsmath}
\usepackage{amssymb}  
\usepackage{amsthm}
\usepackage{ulem} 
\usepackage{graphicx}
\graphicspath{ {images/} }

\begin{document}
%Header-Make sure you update this information!!!!
\noindent
\large\textbf{Space Curves} \hfill \textbf{John Waczak} \\
\normalsize MTH 434 \hfill  Date: \today \\
Dr. Christine Escher \\

\section*{3A}
	\subsection*{b. Questions}
	There were two questions I had after reading this section. Is it possible to abstract the unit binormal vector to n dimensions? Even though you wont have the cross product as long as you can find your normal and tangent vectors you could just Gramm-Schmidt your way to n vectors. Would this lose the geometric meaning? 
	My second question isn't really a question. Rather I just got lost a bit during the derivation in the final page of the chapter involving the Taylor expansions and could use some help clarifying how that was accomplished. 
	
	\subsection*{c. Reflections}
	I found this section interesting- particularly the concept of torsion. However in trying to get used to all of these new unit vector I am still confused as to why we don't just refer to $\mathbf{t} \mathbf{n}$ as $\hat{\mathbf{v}}, \hat{\mathbf{a}}^{\perp}$. It seems to me like thinking of these vectors as just the unit vectors of velocity and acceleration would be much easier to remember. 
	
	\subsection*{d. Time}
	It took me approximately 20 an hour to read through the section two times and take note of everything I thought was important. I would like to re-read the final section which involved the Taylor expansion a few times. 

\end{document}
