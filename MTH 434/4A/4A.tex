\documentclass[a4paper, 11pt]{article}
\usepackage{geometry}
\geometry{letterpaper, margin=1in}
\usepackage{amsmath}
\usepackage{amssymb}  
\usepackage{amsthm}
\usepackage{ulem} 
\usepackage{graphicx}
\graphicspath{ {images/} }

\begin{document}
%Header-Make sure you update this information!!!!
\noindent
\large\textbf{Curvature} \hfill \textbf{John Waczak} \\
\normalsize MTH 434 \hfill  Date: \today \\
Dr. Christine Escher \\

\section*{4A}
	\subsection*{b. Questions}
	I didn't think this section was too confusing. The only question I have is regarding definition 1.62. Is a positively oriented basis the same thing as a left handed basis? I had some exposure to rigid motions in MTH 343. 
	\subsection*{c. Reflections}
	This section makes a lot of intuitive sense as the properties of the curvature and torsion should not depend on how the curve is situated in $\mathbb{R}^n$. Moving the entire trace from one location to another or rotating it should have no effect on how the curve bends and twists. 
	
	\subsection*{d. Time}
	I took roughly 15 minutes to read this section. 

\end{document}