\documentclass[a4paper, 11pt]{article}
\usepackage{geometry}
\geometry{letterpaper, margin=1in}
\usepackage{amsmath}
\usepackage{amssymb}  
\usepackage{amsthm}
\usepackage{ulem} 
\usepackage{graphicx}
\graphicspath{ {images/} }

\begin{document}
%Header-Make sure you update this information!!!!
\noindent
\large\textbf{Orientable Surfaces} \hfill \textbf{John Waczak} \\
\normalsize MTH 434 \hfill  Date: \today \\
Dr. Christine Escher \\

\section*{12A}
	\subsection*{b. Questions}
	I think the notion of an orientable surface makes sense. We mentioned this briefly during MTH 254 and 255 so the given definition seems reasonable. The idea of the normal field and tying that to the orientation makes sense. One question I had is what would be the interior and exterior for a plane? Also, the book uses "connected oriented surfaces" in proposition 3.52. What does it mean for two surface to be connected? The last bit about the Mobius strip was very interesting but I got a bit lost when they defined the parametrization.  I could use some extra explanation for how this was derived. 
	
	\section*{c. Reflections}
	I read through the section once and then went back again to make sure I could interpret the pictures and diagrams given as the idea of orientation and normal fields seems very visual to me. 
	
	
	\subsection*{d. Time}
	I took roughly 30 minutes to reread this section. 

\end{document}