\documentclass[a4paper, 11pt]{article}
\usepackage{geometry}
\geometry{letterpaper, margin=1in}
\usepackage{amsmath}
\usepackage{amssymb}  
\usepackage{amsthm}
\usepackage{ulem} 
\usepackage{graphicx}
\graphicspath{ {images/} }

\begin{document}
%Header-Make sure you update this information!!!!
\noindent
\large\textbf{Second Fundamental Form} \hfill \textbf{John Waczak} \\
\normalsize MTH 434 \hfill  Date: \today \\
Dr. Christine Escher \\

\section*{19A}
	\subsection*{b. Questions}
	 The first thing I think I can use some clarification on is the fake spheres example. I think I understand the idea-- to show that because we are looking at local properties we can construct strange pathological examples to demonstrate how this pullback of the second fundamental form does lose some global information. The bit I'm confused about is near the end of page 222 where they explain how they are choosing the different I intervals for the t parameter. \\ 
	 
	 I could also use some clarification on proposition 4.21. I think I can follow the proof but I got a little confused by the geometric explanation given in the last sentence of page 223. 
	
	\section*{c. Reflections}
	I thought this section was pretty clear and I definitely appreciated the two examples with the sphere and the graph. I also found the strategy explained in 4.14 pretty interesting for solving for the principal curvatures once we have K and H. I had not previously considered that strategy since we were mostly interested in going the other way before-- using the principal curvatures to find the Gaussian and Mean curvatures. 
	
	
	\subsection*{d. Time}
	I took roughly 30 minutes to read this section. 

\end{document}