\documentclass[a4paper, 11pt]{article}
\usepackage{geometry}
\geometry{letterpaper, margin=1in}
\usepackage{amsmath}
\usepackage{amssymb}  
\usepackage{amsthm}
\usepackage{ulem} 
\usepackage{graphicx}
\usepackage{enumitem} % use for making lettered list 
\usepackage{bbm} % use for making the 1 identity operator EX: \mathbbm{1}
\usepackage{subfig} 
\graphicspath{ {images/} }

% format to allow bolded theorems, corollaries, etc... 
\newtheorem*{theorem}{Theorem}
\newtheorem*{corollary}{Corollary}
\newtheorem*{lemma}{Lemma}
\newtheorem*{definition}{Definition}
\newtheorem*{Example}{Example} 
\newtheorem*{Remark}{Remark}

% stop typing \mathbb a thousand times 
\newcommand{\R}{\mathbb{R}}
\newcommand{\C}{\mathbb{C}}
\newcommand{\F}{\mathbb{F}}
\newcommand{\Mat}[2]{\mathcal{M}_{#1\times#2}}

% change margins for solution
\newenvironment{solution}{%
	\begin{list}{}{%
			\setlength{\topsep}{0pt}%
			\setlength{\leftmargin}{1.5cm}%
			\setlength{\rightmargin}{1.5cm}%
			\setlength{\listparindent}{\parindent}%
			\setlength{\itemindent}{\parindent}%
			\setlength{\parsep}{\parskip}%
		}%
		\item[]}{\end{list}}

\begin{document}
%Header-Make sure you update this information!!!!
\noindent
\large\textbf{Homework 7} \hfill \textbf{John Waczak} \\
\normalsize MTH 443 \hfill  Date: \today \\
Dr. Schmidt \hfill Worked with: Garrett Jepson 
\par\noindent\rule{\textwidth}{0.4pt} \\\\

\noindent 1.) Let $T$ be a linear operator defined on $\C^4$ by $Te_1 =0$, $Te_2=-5e_1$, $Te_3=5e_1$, and $Te_4 = 2e_2 + 5e_3$. Determine a Jordan basis for $\C^4$ with respect to $T$. \\

\begin{solution}
  \noindent In order to find a Jordan basis with respect to $T$, let us consider the matrix representation of $T$ in the standard basis
  \begin{equation*}
    \mathcal{B} = \{e_1, e_2, e_3, e_4\}
  \end{equation*}
  From the definition of how $T$ acts on each of these vectors we find
  \begin{equation*}
    [T]_\mathcal{B} = \begin{pmatrix}
      0 & -5 & 5 & 0 \\
      0 & 0 & 0 & 2 \\
      0 & 0 & 0 & 5 \\
      0 & 0 & 0 & 0 \end{pmatrix}
  \end{equation*}
  From this we can see that the characteristic polynomial for $T$ is
  \begin{equation*}
    c_T[x] = x^4
  \end{equation*}
  This polynomial splits over $\C$ and gives a single eigenvalue of $\lambda = 0$ with algebraic multiplicity $4$. To find the corresponding eigenspace, we look for solutions to $[T]_\mathcal{B}-0\;\mathcal{I}_{\C^4}=0$ i.e.
  \begin{align*}
    \begin{pmatrix}
      0 & -5 & 5 & 0 \\
      0 & 0 & 0 & 2 \\
      0 & 0 & 0 & 5 \\
      0 & 0 & 0 & 0 \end{pmatrix}&\begin{pmatrix} x \\ y \\ z \\ t \end{pmatrix} = \begin{pmatrix} 0 \\ 0 \\ 0 \\ 0 \end{pmatrix} \\
    \Rightarrow x &= \text{anything} \\
    y &= z \\
    t &= 0 \\ 
  \end{align*}
  From this we see that the eigenspace for $\lambda=0$ is given by
  \begin{equation*}
    \mathcal{E}_0 = \operatorname{span}\left\{ \begin{pmatrix}0 \\ 1 \\ 1 \\ 0 \end{pmatrix}, \begin{pmatrix}1 \\ 0 \\ 0 \\ 0 \end{pmatrix}\right\}
  \end{equation*}
  From class we know that $\mathcal{E}_0 \subseteq \mathcal{K}_0$ where $\mathcal{K}_0$ is the generalized eigenspace for $\lambda = 0 $. Therefore, each of the basis vectors we have identified for $\mathcal{E}_0$ constitutes a Jordan chain with one element. Now we want to \textit{grow} these chains so that we have 4 linearly independent vectors. Growing the chain means looking for vectors that satisfy $([T]_\mathcal{B}-0\;\mathcal{I}_{\C^4})^2v = 0$ Let's try a vector that lies in the range of $T$ and not in the null space. An easy choice would be $e_4$. Calculation yields
  \begin{align*}
    [T]_\mathcal{B}\begin{pmatrix}0\\0\\0\\1\end{pmatrix} &= \begin{pmatrix}0\\2\\5\\0\end{pmatrix}\\
        [T]_\mathcal{B}^2\begin{pmatrix}0\\0\\0\\1\end{pmatrix} &= [T]_\mathcal{B}\begin{pmatrix}0\\2\\5\\0\end{pmatrix}\\
          &=\begin{pmatrix}0\\0\\0\\0\end{pmatrix}
  \end{align*}
  Therefore we have that $e_4\in\ker(T^2)$ which gives us a new Jordan Chain of length 2:
  \begin{equation*}
    C = \left\{ 2e_2+5e_3,\;\; e_4\right\}
  \end{equation*}
  Thus we conclude that the Jordan basis for $\C^4$ with respect to $T$ is
  \begin{equation*}
    \mathcal{J} = \left\{ e_2+e_3,\; e_1,\; 2e_2+5e_3,\; e_4\right\}
  \end{equation*}
  Writing $\mathcal{J}$ as a matrix and taking the determinant yields $\det([\mathcal{J}]) = -3 \neq 0$ which confirms that these vectors are linearly independent and therefore do form a basis for $\C^4$. If we compose this change of basis transformation with $[T]_\mathcal{B}$ we find
  \begin{equation*}
    \mathcal{J}^{-1}[T]_\mathcal{B}\mathcal{J} = \begin{pmatrix}
      0 & 0 & 0 & 0 \\
      0 & 0 & 1 & 0 \\
      0 & 0 & 0 & 1 \\
      0 & 0 & 0 & 0
      \end{pmatrix}
  \end{equation*}
 which is in Jordan Canonical form as desired. \\
\end{solution}

\noindent 2.) Let $V$ be a finite dimensional $\C$-vector space and let $T$ be a linear operator on $V$. Assume that $c_T[x]=x^{10}$ and $N_T = R_T$. Does this information determine the Jordan canonical form of T? If so, give this form and if not, give examples to this effect.\\

\begin{solution}
  \noindent We see that the characteristic polynomial splits over $\C$ into 10 factors of $(x-0)$. Therefore, by theorem (Jordan Mania class notes) we have that there exists a Jordan basis for T and therefore the matrix of $T$ is similar to a matrix in Jordan Canonical Form. Furthermore, because the degree of $c_T[x]$ is 10 we have that $\dim(V)=10$. Therefore, by the Rank-Nullity theorem, we have that
  \begin{align*}
    10 &= \dim(N_T) + \dim(R_T) \\
    &= \dim(N_T) + \dim(N_T) \\
    \Rightarrow \dim(N_T) &= \dim(R_T) = 5
  \end{align*}

  \noindent $c_T[x]$ has a single root at 0 meaning that $T$ has a single eigenvalue of $\lambda=0$ with algebraic multiplicity 10. Based on the above calculation, we therefore have that the eigenspace $\mathcal{E}_0=\ker(T-0\;\mathcal{I}) = N_T$ is spanned by 5 linearly independent vectors. Because each of these vectors is in the kernel of $T$, each constitutes a Jordan chain of one element. We now wish to extend these chains so that they have two elements. Therefore, we are looking for vectors $b_i$ satisfying $b_i\in\ker(T-0\;\mathcal{I})^2$ or
  \begin{align*}
    T^2 b_i = T(Tb_i) = 0
  \end{align*}
  From this we see that we must have $Tb_i\in\ker{T}$ and therefore we can write 
  \begin{equation*}
    Tb_i = \sum_i^5 \alpha_i v_i
  \end{equation*}
  At this point nothing prevents us from choosing the $b_i$ such that the summation simplifies to something nice. 
  \begin{equation*}
    Tb_i = v_i
  \end{equation*}
  The $v_i$ are linearly independent and therefore each of the $b_i$ are distinct because linear operators are functions and therefore send each input to exactly one output. We now have that the Jordan chains become
  \begin{equation*}
    \mathcal{J}_i = \{ v_i, b_i\}
  \end{equation*}
  The matrix representation of each of these distinct chains forms a Jordan block. These are
  \begin{align*}
    B_i = T\mathcal{J}_i = \begin{pmatrix} 0 & 1 \\ 0 & 0 \end{pmatrix}
  \end{align*}
  Now we have everything we need to construct the JCF of T. We have 5, 2x2 Jordan blocks with the above form. If $\mathcal{B}_\mathcal{J}=\bigcup\limits_i^5\mathcal{J}_i$, then
  \begin{equation*}
    [T]_{\mathcal{B}_\mathcal{J}}=\begin{pmatrix}
    0 & 1 & 0 & 0 & 0 & 0 & 0 & 0 & 0 & 0 \\
    0 & 0 & 0 & 0 & 0 & 0 & 0 & 0 & 0 & 0 \\
    0 & 0 & 0 & 1 & 0 & 0 & 0 & 0 & 0 & 0 \\
    0 & 0 & 0 & 0 & 0 & 0 & 0 & 0 & 0 & 0 \\
    0 & 0 & 0 & 0 & 0 & 1 & 0 & 0 & 0 & 0 \\
    0 & 0 & 0 & 0 & 0 & 0 & 0 & 0 & 0 & 0 \\
    0 & 0 & 0 & 0 & 0 & 0 & 0 & 1 & 0 & 0 \\
    0 & 0 & 0 & 0 & 0 & 0 & 0 & 0 & 0 & 0 \\
    0 & 0 & 0 & 0 & 0 & 0 & 0 & 0 & 0 & 1 \\
    0 & 0 & 0 & 0 & 0 & 0 & 0 & 0 & 0 & 0 \\
    \end{pmatrix}
  \end{equation*}

  \noindent In finding this JCF we assumed that each chain could be grown to a chain with 2 elements. This is just an assumption and without more information it is not clear whether or not each chain can be grown or should stop at 2 elements. From class we know that the dimension of the eigenspace is the number of chains of a fixed eigenvalues. Thus it is conceivable that we could also find a JCF in which we have three 3x3 blocks and a 1x1 block, or perhaps a 5x5 block, two 2x2 blocks and a 1x1 block. So long as there are 5 blocks and the total size is 10, the JCF is valid. \\

  \noindent Therefore, we conclude that we are not given enough information to explicitly determine the JCF of the linear operator $T$. 

\end{solution}

\noindent 3.) Determine the Jordan canonical form of the linear operator $T$ on $\C^n$ defined by $Te_i = \sum\limits_i^n e_i$\\
\begin{solution}
  \noindent If we take the basis for $T$ to be the standards basis,
  \begin{equation*}
    \mathcal{B} = \{e_1, e_2, ... , e_n\},
  \end{equation*}
  then the above definition indicates that the matrix of $T$ in the basis $\mathcal{B}$ is the nxn matrix of ones. That is,
  \begin{equation*}
    [T]_\mathcal{B} = \begin{pmatrix}
      1 & 1 & 1 & \cdots & 1 \\
      1 & 1 & 1 & \cdots & 1 \\
      \vdots & \vdots & \vdots & \ddots & \vdots \\
      1 & 1 & 1 & \cdots & 1 
    \end{pmatrix}
  \end{equation*}
  We want to find the Jordan Canonical Form of this matrix. To do this, we first consider the characteristic polynomial of $T$ given by
  \begin{align*}
    c_T[x] &= \det([T]_\mathcal{B}-\lambda\mathcal{I})\\
    &= \begin{vmatrix}
      1-x & 1 & 1 & \cdots & 1 \\
      1 & 1-x & 1 & \cdots & 1 \\
      \vdots & \vdots & \vdots & \ddots & \vdots \\
      1 & 1 & 1 & \cdots & 1-x
      \end{vmatrix}\\
  \end{align*}
  For the case of n=2, we have $c_T[x] = (1-x)^2-1 = (x-2)x$. For the case of $n=3$, we have $c_T[x]= 3x^2-x^3 = (3-x)x^2$. Therefore, by an induction, the characteristic polynomial for the n=n cases is
  \begin{equation*}
    c_T[x] = (x-n)x^{n-1}
  \end{equation*}
  Therefore, equating $c_T[x]$ to zero yields the eigenvalues $\lambda = n$ with algebraic multiplicity one, and $\lambda=0$ with algebraic multiplicity $n-1$. Now consider the eigenvalue equation for the first eigenvalue. We have
  \begin{equation*}
    Tv = nv
  \end{equation*}
  for an eigenvector n. If $v=\sum_i\alpha_ie_i$ then the above equation becomes
  \begin{equation*}
    \sum_i\alpha_i = n\alpha_j
  \end{equation*}
  For each $j$. This is solved if we allow $\alpha_i=\alpha_j$ $\forall i,j$ Therefore we have that the vector $v_1 = \sum_i e_i$ spans the eigenspace $\mathcal{E}_n$. For the other eigenvalue of $\lambda=0$ we have that this reduces to finding the kernel of T. That is, we are looking for solutions to
  \begin{align*}
    Tv &= 0v \\
    \Rightarrow \sum_i \alpha_i &= 0
  \end{align*}
  This leads to the condition that
  \begin{equation*}
    \sum_{i\neq j} \alpha_i= -\alpha_j
  \end{equation*}
  Which means that we have $n-1$ free parameters. Therefore, the eigenspace $\mathcal{E}_0$ is spanned by $n-1$ vectors. Each of these vectors constitutes a Jordan chain of size 1 so that overall we have a single chain from $\mathcal{E}_n$ and $n-1$ chains from $\mathcal{E}_0$. Therefore if we take the ordering of our Jordan basis to be the $\lambda=0$ eigenvectors first, our JCF of T will actually be diagonalized, i.e.
  \begin{equation*}
    [T]_\text{JCF} = \begin{pmatrix}
      0 & 0 & 0 & \cdots & 0 \\
      0 & 0 & 0 & \cdots & 0 \\
      \vdots & \vdots & \vdots & \ddots & \vdots \\
      0 & 0 & 0 & \cdots & n
    \end{pmatrix}
  \end{equation*}
  In terms of the original transformation, this is
  \begin{equation*}
    \begin{cases}
      Tb_i = 0; & i<n \\ 
      Tb_i = nb_i; & i=n\\
    \end{cases}
  \end{equation*}
  where the $b_i$ are the elements of the Jordan basis described above. \\
\end{solution}





\noindent 4.) (543) 


\end{document}



















