\documentclass[a4paper, 11pt]{article}
\usepackage{geometry}
\geometry{letterpaper, margin=1in}
\usepackage{amsmath}
\usepackage{amssymb}  
\usepackage{amsthm}
\usepackage{ulem} 
\usepackage{graphicx}
\usepackage{enumitem} % use for making lettered list 
\usepackage{bbm} % use for making the 1 identity operator EX: \mathbbm{1}
\usepackage{subfig} 
\graphicspath{ {images/} }

% format to allow bolded theorems, corollaries, etc... 
\newtheorem*{theorem}{Theorem}
\newtheorem*{corollary}{Corollary}
\newtheorem*{lemma}{Lemma}
\newtheorem*{definition}{Definition}
\newtheorem*{Example}{Example} 
\newtheorem*{Remark}{Remark}

% stop typing \mathbb a thousand times 
\newcommand{\R}{\mathbb{R}}
\newcommand{\C}{\mathbb{C}}
\newcommand{\F}{\mathbb{F}}
\newcommand{\Mat}[2]{\mathcal{M}_{#1\times#2}}

\begin{document}
%Header-Make sure you update this information!!!!
\noindent
\large\textbf{Class notes} \hfill \textbf{John Waczak} \\
\normalsize MTH 443 \hfill  Last Updated: \today \\
Dr. Schmidt
\par\noindent\rule{\textwidth}{0.4pt}	
	
\section*{Notation Comments} 
	\begin{Remark}
		The notation $L_A:\F^n \to F^m$ when $A\in M_n(\F)$ has the letter $L$ to indicate left multiplication by A on column vectors. 
	\end{Remark}

	\begin{Remark}
		Given any linear operator $T:V\to V$, and finite ordered bases $B, C$ for $V$. The matrix of $T$ with respect to $B$ and $C$ is denoted $[T]_B^C$. In particular, 
			\begin{equation}
				[Id_v]_C^B = \Big([Id_v]_b^C\Big)^{-1}
			\end{equation}
		From this, 
			\begin{align}
				[T]_C^C &= [Id_v]_B^C\; [T]_B^B\; [Id_v]_C^B \\ 
					&= Q^{-1}\;[T]_B^C\; Q
			\end{align}
		where $Q = [Id_v]_C^B$ is the change of basis matrix. 
	\end{Remark}

\section*{Cosets}
	If U is a subspace of V and $v\in V$ then the left coset of U in V represented by v is
		\begin{equation}
			v+U = \{v+u|u\in U\}
		\end{equation}
	The \textbf{set of left cosets} of U in V is 
		\begin{equation}
			V/U = \{v+U | v \in V\}
		\end{equation}
	Note that if $v\in V$ and $u \in U$ then $v+U = (v+u)+U$. Naively, we could hope that 
		\begin{align*}
			V/U \times V/U &\to V/U  \\ 
			(v_1+U, v_2+U) &\mapsto (v_1+v_2)+U
		\end{align*}
	actually defines a function. We have to be sure that when you choose some $v_1', v_2'$ that the resulting coset is the same... i.e. that we need to check that this really is a function for which inputs have exactly one output. That is, 
		\begin{align*}
			(v_1+U, v_2+U) \mapsto v_1+v_2+U
		\end{align*}
	is well-defined, in the sense that the right hand side value is independent of choice of coset representatives of the initial cosets. Here, if $v_1' = v_1+u_1$, $v_2' = v_2+u_2$ with $u_i \in U$. Now 
		\begin{equation*}
			v_1' + v_2' + U = \Big[(v_1+u_1)+(v_2+u_2)\Big]+ U
		\end{equation*}
	Thus,
		\begin{align*}
			v_1'+v_2'+U &= (v_1+u_1+v_2) + (u_2 + U) \\ 
				&= (v_1+v_2+u_1) + U \\ 
				&= v_1+v_2 + U
		\end{align*}
	That is, since addition on V is Abelian, every subgroup U is normal and thus the naive formula does give a well-defined function. We now check if 
		\begin{align*}
			\F \times V/U &\to V/U \\ 
			(\lambda, v+U)&\mapsto \lambda v + U
		\end{align*}
	is a well-defined function (\textbf{IT IS}). So the family of cosets of subspace U in vector space V is itself a vector space over $\F$. 
	
	\begin{lemma}
		Suppose U is a vector subspace of V, and B is a basis of U. Let $C = B\cup B'$ be any basis of V extending B. Then, $\{v+U | v \in B'\}$ is a basis of our quotient vector space $V/U$. 
	\end{lemma}
	\begin{proof}
		Suppose $\sum_i \lambda_i (v_i+U) = 0_{V/U}$ for some $\lambda_1,..., \lambda_n \in \F$. and $v_1, ... , v_n \in B'$. Since $0_{V/U} = 0_v + U = U$ is our zero vector, thus
			\begin{align*}
				\Big(\sum_i^n \lambda_i v_i\Big) + U = U
			\end{align*} 
		This holds if and only if 
			\begin{align*}
				\sum_i^n \lambda_i v_i \in U
			\end{align*}
		However, the $v_i\in B'$ and hence are linearly independent of the $sp(B)$. Therefore, this linear combination can only be $0_V\in U$. But $C$ is a basis and thus all of the $\lambda_i=0$. Note if $U=V$ then $V/U$ is only $\{0_v + U\}$ and one uses logical statements. 
	\end{proof}


\end{document}






























