\documentclass[a4paper, 11pt]{article}
\usepackage{geometry}
\geometry{letterpaper, margin=1in}
\usepackage{amsmath}
\usepackage{amssymb}  
\usepackage{amsthm}
\usepackage{ulem} 
\usepackage{graphicx}
\usepackage{enumitem} % use for making lettered list 
\usepackage{bbm} % use for making the 1 identity operator EX: \mathbbm{1}
\usepackage{subfig} 
\graphicspath{ {images/} }

% format to allow bolded theorems, corollaries, etc... 
\newtheorem*{theorem}{Theorem}
\newtheorem*{corollary}{Corollary}
\newtheorem*{lemma}{Lemma}
\newtheorem*{definition}{Definition}
\newtheorem*{Example}{Example} 
\newtheorem*{Remark}{Remark}

% stop typing \mathbb a thousand times 
\newcommand{\R}{\mathbb{R}}
\newcommand{\C}{\mathbb{C}}
\newcommand{\F}{\mathbb{F}}
\newcommand{\Mat}[2]{\mathcal{M}_{#1\times#2}}

% change margins for solution
\newenvironment{solution}{%
	\begin{list}{}{%
			\setlength{\topsep}{0pt}%
			\setlength{\leftmargin}{1.5cm}%
			\setlength{\rightmargin}{1.5cm}%
			\setlength{\listparindent}{\parindent}%
			\setlength{\itemindent}{\parindent}%
			\setlength{\parsep}{\parskip}%
		}%
		\item[]}{\end{list}}

\begin{document}
%Header-Make sure you update this information!!!!
\noindent
\large\textbf{Homework 2} \hfill \textbf{John Waczak} \\
\normalsize MTH 443 \hfill  Date: \today \\
Dr. Schmidt
\par\noindent\rule{\textwidth}{0.4pt} \\

\noindent(1). (a) Clearly state under what conditions the range and null space of a linear transformation T are the same set. 
	\begin{solution}
		\noindent Consider a general linear transformation $T:V\to W$ where $V$ and $W$ are $\F$-vector spaces. In order for the range of T to be equal to its null space, we must have the following
			\begin{enumerate}
				\item $\operatorname{rank}(T) = \operatorname{nullity}(T)$
				\item $\dim(V)$ is even 
				\item W = V 
			\end{enumerate}
	\end{solution}

\noindent(b) Prove your assertion 
	\begin{solution}
		\begin{proof}
			Recall from class that the Dimension Theorem (2.3) asserts 
				\begin{equation*}
					\operatorname{rank}(T) + \operatorname{nullity}(T) = \dim(V) 
				\end{equation*}
		\end{proof}
	\end{solution} 
\noindent(c) Give an example
	\begin{solution}
		\noindent\textbf{Example:} Consider the linear transformation $T:\F^2\to\F^2$ given by
			\begin{equation*}
				T(x,y) = (0, x) 
			\end{equation*}
		Acting this transformation on the canonical basis vectors $e_1, e_2$ generates the matrix representation for $T$ denoted $A_T \in \mathcal{M}_{2\times2}$. 
			\begin{equation*}
				A_T = \begin{pmatrix}
					0 & 0 \\ 
					1 & 0 
				\end{pmatrix}
			\end{equation*}
		Then solutions to the equation $Ax=0$ are easily shown by row operations to be of the form
			\begin{equation*}
				\begin{pmatrix} 0 \\ \lambda \end{pmatrix}, \quad \forall \lambda\in\F
			\end{equation*}
		Therefore the null space of this linear transformation is
			\begin{equation*}
				\ker(T) = \{(x,y)\in \F^2| T(x,y) = (0,0)\} = \operatorname{span}(e_2)
			\end{equation*}
	\end{solution}

\end{document} 


























