\documentclass[a4paper, 11pt]{article}
\usepackage{geometry}
\geometry{letterpaper, margin=1in}
\usepackage{amsmath}
\usepackage{amssymb}  
\usepackage{amsthm}
\usepackage{ulem} 
\usepackage{graphicx}
\usepackage{enumitem} % use for making lettered list 
\usepackage{bbm} % use for making the 1 identity operator EX: \mathbbm{1}
\usepackage{subfig} 
\graphicspath{ {images/} }

% format to allow bolded theorems, corollaries, etc... 
\newtheorem*{theorem}{Theorem}
\newtheorem*{corollary}{Corollary}
\newtheorem*{lemma}{Lemma}
\newtheorem*{definition}{Definition}
\newtheorem*{Example}{Example} 
\newtheorem*{Remark}{Remark}

% stop typing \mathbb a thousand times 
\newcommand{\R}{\mathbb{R}}
\newcommand{\C}{\mathbb{C}}
\newcommand{\F}{\mathbb{F}}
\newcommand{\Mat}[2]{\mathcal{M}_{#1\times#2}}

% change margins for solution
\newenvironment{solution}{%
	\begin{list}{}{%
			\setlength{\topsep}{0pt}%
			\setlength{\leftmargin}{1.5cm}%
			\setlength{\rightmargin}{1.5cm}%
			\setlength{\listparindent}{\parindent}%
			\setlength{\itemindent}{\parindent}%
			\setlength{\parsep}{\parskip}%
		}%
		\item[]}{\end{list}}

\begin{document}
%Header-Make sure you update this information!!!!
\noindent
\large\textbf{Homework 5} \hfill \textbf{John Waczak} \\
\normalsize MTH 443 \hfill  Date: \today \\
Dr. Schmidt \hfill Worked with: Garrett Jepson 
\par\noindent\rule{\textwidth}{0.4pt} \\\\

\noindent 1.) Let $\mathcal{B} = (b_1,...,b_n)$ be an n-tuple of elements of $\F^n$. Let $M\in\mathcal{M}_n(\F)$ be the matrix whose j-th column is $b_j$. Show that $\mathcal{B}$ is an ordered basis of $\F^n$ if and only if $\det(M)\neq 0$.\\

\begin{solution}
  \noindent $(\rightarrow)$ Assume that $\mathcal{B}$ is an ordered basis. We must show that $\det(M)\neq 0$. As $\mathcal{B}$ is an ordered basis, its elements are linearly independent. That is no column of the matrix $M$ whose columns are $b_j\in\mathcal{B}$ can be expressed as a linear combination of the other columns of $M$. This means that $M$ reduced to the identity matrix. In order to use this information to calculate the determinant, we must recall the following theorems from the text:\\

  \noindent \textbf{Theorem 4.5} If $A\in\mathcal{M}_n(\F)$ and $B$ is a matrix obtained by switching any two rows of a, then
  \begin{equation*}
    \det(B) = -\det(A)
  \end{equation*}

  \noindent \textbf{Theorem 4.6} If $A\in\mathcal{M}_n(\F)$ and B is a matrix obtained by adding a multiple of one row of A to another row of A. Then,
  \begin{equation*}
    \det(B) = \det(A)
  \end{equation*}

  \noindent Therefore, because $\mathcal{B}$ is a basis, the columns of $M$ are linearly independent and the matrix can therefore be row reduced in, say, $k$ moves such that
  \begin{align*}
    det(M) &= (-1)^k\det(\text{Id}_\F) \\
    &= (-1)^k\cdot 1 \neq 0 \\ 
  \end{align*}

  \noindent From this we can see that if $\mathcal{B}$ is an ordered basis then $\det(M)\neq 0$. \\

  \noindent$(\leftarrow)$ Assume that $\det(M)\neq 0$. Now, assume for contradiction that the $b_j\in\mathcal{B}$ do not form an ordered basis. It must be true that $\exists i\in\{1,...,n\}$ such that $b_i$ is a linear combination of the vectors of a subset of $\mathcal{B}$. Recall the following theorem from the text:

  \noindent \textbf{Theorem 4.8} For any $A\in\mathcal{M}_n(\F)$, $\det(A^t) = \det(A)$. \\

  \noindent Consider the matrix $M^t$ in which the previous $b_j\in\mathcal{B}$ columns of $M$ have become rows. For this new matrix $M^t$, we have that row $b_i^t$ is a linear combination of some number of other rows. However, in class we proved that the determinant is zero if any two rows are linearly dependent. Thus, we have that
  \begin{equation*}
      \det{M^t} = 0 = \det{M}
  \end{equation*}
  This contradicts the hypothesis and therefore we have that if $\det(M)\neq 0$, then $\mathcal{B}$ is an ordered basis.  This completes the proof. \qed \\ 
\end{solution}

\noindent 2.) Let $V$ be an $\R$-vector space of dimension 2 and let $T$ be a linear operator on $V$. Suppose $[T]_\mathcal{B} = \begin{pmatrix} 1 & -1 \\ 2 & 2 \end{pmatrix}$, for some basis $\mathcal{B}$. Determine all $T$-invariant subspaces of $V$.\\

\begin{solution}
  \noindent Recall that a subspace $W$ of $V$ is called $T$-invariant if for all $w\in W$, $T(w)\in W$. From class, we saw that $\{0_V\}$ and $V$ are certainly $T$-invariant subsapces for any linear operator as $T(0_V) = 0_V$ and $T(v)\in V$  by definition of a linear operator. Certainly if $V$ has dimension 2 there can be no other 2-dimensional subspace than $V$ itself as any such space must also contain it's span. Thus, the only other possible $T$-invariant subspaces must have dimension 1. To solve for such subspaces, consider a vector $v$ such that $\operatorname{span}(v)$ is a $T$ invariant subspace. In order for this to work, $T$ must send $v$ to $\operatorname{span}(v)$, that is,
  \begin{equation*}
    T(v) = \lambda v
  \end{equation*}
  That is, the other T-invariant subspaces must be the eigenspaces of T. The matrix $[T]_\mathcal{B}$ has a characteristic polynomial
  \begin{equation*}
    P_T(x) = (1-x)(2-x)+2
  \end{equation*}
  Setting this to zero gives
  \begin{align*}
    (1-x)(2-x)+2 &= 2-x-2x+x^2+2 \\
    &= x^2-3x+4 = 0 
  \end{align*}
  Thus the eigenvalues of this operator are
  \begin{align*}
    \lambda_{1,2} &= \frac{3\pm\sqrt{9-16}}{2} \\
    &= \frac{3\pm i\sqrt{7}}{2}
  \end{align*}
  Therefore, we can see that there can't be any such $T$-invariant subspaces of V because there are no real eigenvalues of $T$. Such complex eigenvalues would correspond to vectors in $\C^2$ to which our $\R$-vector space $V$ is not isomorphic. Perhaps such subspaces could be allowed if we had $V$ was a $\C$-vector space. \\ 


  \noindent The only other $T$-invariant subspaces we have encountered before are the range $T(V)$ and the kernel $\ker(V)$. One can easily verify that $\ker(T) = \{0_V\}$ because the columns of $[T]_\mathcal{B}$ are linearly independent. By rank nullity, we have that the dimension of the image is 2 and therefore must also span all of $V$. Thus we can conclude that the only two $T$-invariant subspaces of $V$ are $\{0_V\}$ and $V$. \\
\end{solution}


\noindent 3.) (543)\\

\noindent 4.) Give an example of a continuous function $v:\R\to\R^3$ such that $v(t_1), v(t_2), v(t_3)$ form an $\R$-basis for $\R^3$ whenever $t_1, t_2, t_3$ are distinct points of $\R$.\\

\begin{solution}
  \noindent First, let's consider possible solutions for a function $f:\R\to\R^2$ with similar properties to aid in our construction. Certainly we can not have a constant function as $f(t_1)=f(t_2)$ $\forall t_1\neq t_2 \in \R$. Thus we can check the naive next step:
  \begin{equation*}
    f(t) = (1, t)^t
  \end{equation*}
  In order for pairs of $t_1, t_2$ to form a basis for $\R^2$ we need that $det[f(t_1)\; f(t_2)]\neq 0$ Fortunately this determinant is
  \begin{equation*}
    \det\begin{pmatrix}1 & 1 \\ t_1 & t_2 \end{pmatrix} = t_1 - t_2
  \end{equation*}
  which certainly isn't zero if $t_1, t_2$ are distinct. Thus we have that for any two points of $\R$ f applied to these points forms a basis for $\R^2$. \\

  \noindent Based on this information we can consider how we should add a third component try and extend the property of the function f to $\R^3$. One simple way could be to increase the final component for successive values of $t$. Thus we shall consider the function
  \begin{equation*}
    v(t) = (1, t, e^t)^t
  \end{equation*}
  Certainly this function $v(t)$ is continuous as each of its components consists of a continuous function. If we take $t_1, t_2, t_3 \in \R$ to be three distinct points, we must have that
  \begin{equation*}
    \det[v(t_1)\; v(t_2)\; v(t_3)] \neq 0
  \end{equation*}
  expanding this out gives
  \begin{equation*}
    t_2 e^{t_3}-e^{t_2}t_3-t_1e^{t_3}+e^{t_1}t_3+t_1e^{t_2}-e^{t_1}t_2 \neq 0
  \end{equation*}
  The functions $t, e^t$ are both strictly increasing and $e^t > t$ $\forall t\in\R$. It is not clear to me how to show that this determinant function is always nonzero but I think it is sufficient to show how the function behaves for an example from each of $t_1<t_2<t_3<0$, $\;\;\;$   $t_1 < 0$, $t_3>0$ and $t_1<t_2<t_3$, and lastly $0<t_1<t_2<t_3$ as the behavior of the component functions is well understood. \\

  \noindent For the case of $t_1 = -10$, $t_2 = -5$, $t_3=-1$ we have that the determinant is
  \begin{equation*}
    det[v(t_1)\;\; v(t_2)\;\; v(t_3)] =\frac{4 - 9 e^5 + 5 e^9}{e^{10}} \neq 0
  \end{equation*}

  \noindent For the case of $t_1=-1$, $t_2=0$, $t_3=1$ we have
  \begin{equation*}
    det[v(t_1)\;\; v(t_2)\;\; v(t_3)] = -2 + \frac{1}{e}+e \neq 0
  \end{equation*}

  \noindent and finally for the case of $t_1=1$, $t_2=10$, $t_3=20$ we have that
  \begin{equation*}
    det[v(t_1)\;\; v(t_2)\;\; v(t_3)] =e (10 - 19 e^9 + 9 e^{19})\neq 0
  \end{equation*}
Therefore, I believe I have found an example of such a function. 

\end{solution}



\end{document}



















